\chapter{Centre of Mass}
\label{chapter:cm}


Finding an object's centre of mass is important, because
\begin{itemize}
\item The laws of motion are formulated by treating an objects as point
  masses (for real-life objects, we let the forces apply to the centre of
  mass)
\item Objects can have \emph{rotational} motion in addition to
  \emph{translational} motion as well (we will examine that a bit more in a
  very-important topic later)
\end{itemize}


\section{Finding the Centre of Mass}

The \textbf{centre of mass} (``CM'') is the \emph{weighted average of the
masses in a system.} The ``system'' may be:
\begin{itemize}
\item A collection of individual particles
\item A continuous distribution of mass with constant density. In this case,
  CM is also the geometric centre (called the \textbf{centroid}) of the object
\item A continuous distribution of mass with varying density
\item If the masses are inside a \emph{uniform} gravitational field, then the
  CM is also its \textbf{centre of gravity} (``CG'')
\end{itemize}

Let's begin with two point equal masses $m$ along the $x$-axis, located at
$x_1$ and $x_2$, as shown in Fig.~\ref{fig:2-equal-masses}.
\begin{figure}[ht]
  \centering
  \begin{tikzpicture}[scale=.95]
    \draw[axes] (-1,0)--(8,0) node[right]{$x$};
    \draw[mass] (2,0) circle (.18) node{$m$} node[above=5]{$x=x_1$};
    \draw[mass] (6,0) circle (.18) node{$m$} node[above=5]{$x=x_2$};
    \draw[thick] (0,.15)--+(0,-.3) node[below]{$O$};
    \fill (4,0) circle (.05) node[above]{cm}
     node[below=5]{$x=x_\text{cm}$};
  \end{tikzpicture}
  \caption{Two equal masses along the $x$-axis.}
  \label{fig:2-equal-masses}
\end{figure}
Even for someone without much experience in mathematics and physics, we should
be able to at least correctly \emph{guess} that the centre of mass is at the
half-way point between the masses, i.e.\ arithmetic average of the position of
the two masses:
\begin{equation*}
  x_\text{cm}=\frac{x_1+x_2}2
\end{equation*}
We can imagine the $x$-axis as a literal rod, and that the masses are stuck on
this rod. The point where the masses can be balanced is, of course, at the mid
point. The equation above can also be rewritten as a weighted average, although
this form is still rather unremarkable:
\begin{equation*}
  x_\text{cm}=\frac{mx_1+mx_2}{2m}
\end{equation*}
The mass $m$ of each point mass is the ``weight'' of the sum (pardon me for the
language here), and the denominator is now $2m$ which is the total mass of the
system.

What if one of the masses are increased to $2m$, as shown in
Fig~\ref{fig:unequal-masses}. This is still not a difficult problem; you can
still \emph{guess} the right answer without knowing much about physics or
mathematics, or the equation for centre of mass. 
\begin{figure}[ht]
  \centering
  \begin{tikzpicture}
    \draw[axes] (-4,0)--(4,0) node[right]{$x$};
    \draw[mass] (-2.5,0) circle (.18) node{$m$} node[above=5]{$x=x_1$};
    \draw[mass] (2.5,0) circle (.25) node{$2m$} node[above=5]{$x=x_2$};
    \fill (5/6,0) circle (.05) node[below]{cm};
  \end{tikzpicture}
  \caption{Centre of mass of two unequal mass}
  \label{fig:unequal-masses}
\end{figure}
The centre of mass is no longer half way between the two masses, but now
$\frac13$ the total distance from the larger masses. We can show this using the
weighted average that we used before
\begin{equation*}
  x_\text{cm}=\frac{mx_1+(2m)x_2}{m+2m}
\end{equation*}
Generalizing this equation for $N$ number of masses along the $x$ axis, we
have:
\begin{equation}
  x_\text{cm}=\frac{\sum_{i=1}^Nm_ix_i}{\sum_{i=1}^N m_i}
\end{equation}
The weighted average concept can now be extended to cases when there are
multiple masses in two or more dimensions, as shown in
Fig~\ref{2d-many-masses}.
\begin{figure}[ht]
  \centering
  \begin{tikzpicture}
    \draw[axes] (-3,0)--(3,0) node[right]{$x$};
    \draw[axes] (0,-1.5)--(0,1.5) node[above]{$y$};
    \draw[mass] (-1.3,1) circle (.4) node{$m_1$};
    \draw[mass] (-1.5,-.5) circle (.3) node{$m_2$};
    \draw[mass] (1,.3) circle (.25) node{$m_3$};
    \draw[mass] (0,.3) circle (.2) node{$m_4$};
    \draw[mass] (2,-1) circle (.25) node{$m_5$};
  \end{tikzpicture}
  \caption{Centre of mass of many objects in 2D}
  \label{2d-many-masses}
\end{figure}
The centre of mass is defined for discrete number of masses with the weighted
average:
\begin{important-equation}
  \bm r_\text{cm}=\frac{\sum \bm r_i m_i}{\sum m_i}
  \label{eq:cm-summation}
\end{important-equation}
The centre of mass would be a function of time, i.e.\
$\bm r_\text{cm}=\bm r_\text{cm}(t)$ if any of the masses are in motion. We can
break $\bm r$ into its components:
\begin{equation*}
  x_\text{cm}=\frac{\sum m_ix_i}{\sum m_i}\quad\quad
  y_\text{cm}=\frac{\sum m_iy_i}{\sum m_i}\quad\quad
  z_\text{cm}=\frac{\sum m_iz_i}{\sum m_i}
\end{equation*}



%\begin{frame}{An Example}
%  \textbf{Example:} Consider the following masses and their coordinates
%  which make up a ``discrete mass'' rigid body''
%  \begin{align*}
%    m_1&=\SI{5.0}{\kg} &\bm x_1&=3.0\iii-2.0\kkk\\
%    m_2&=\SI{10.0}{\kg}&\bm x_2&=-4.0\iii+2.0\jjj+7.0\kkk\\
%    m_3&=\SI{1.0}{\kg}&\bm x_3&=10.0\iii-17.0\jjj+10.0\kkk
%  \end{align*}
%  What are the coordinates for the centre of mass of this system?


\subsection{Continuous Mass Distribution}

When the number of masses approaches infinity, this becomes a continuous
distribution of mass. Taking the limit of masses $N\rightarrow\infty$, the
summation in Eq.~\ref{eq:cm-summation} becomes an integral:
\begin{important-equation}
  \bm r_\text{cm}=\frac{\int\bm r\dl m}{\int\dl m}
  \label{eq:cm-integral}
\end{important-equation}
The infinitesimal mass $\dl m$ is calculated based on the type of the problem
that we are solving. For one-dimensional problems, we can relate $\dl m$ to the
\textbf{linear mass density} ($\gamma$), which has a unit of
\emph{kilograms per metre} (\si{\kilo\gram\per\metre}):
\begin{equation}
  \gamma(x) = \diff mx\quad\rightarrow\quad \dl m =\gamma(x)\dl x
\end{equation}
For an object with uniform density, $\gamma=M/L$ is a constant. For
two-dimensional problem, the \textbf{surface mass density} ($\sigma$) is mass
per area of the material, which has a unit of \emph{kilograms per metre squared}
(\si{\kilo\gram\per\metre\squared}):
\begin{equation}
  \sigma(x,y)=\diff mA\quad\rightarrow\quad \dl m =\sigma(x,y)\dl A
\end{equation}
In Cartesian space with an orthogonal $x$ and $y$ axes, $\dl A=\dl x\dl y$.
Finally, for three-dimensional problem, \textbf{volume density}, or just
\textbf{density} ($\rho$), is just defined as the mass per unit volume, with a
unit of \emph{kilograms per metre cubed} (\si{\kilo\gram\per\metre\cubed}):
\begin{equation}
  \rho(x,y,z)=\diff mV\quad\rightarrow\quad \dl m =\rho(x,y,z)\dl V
\end{equation}
The densities do not have to be constant.

Let's start with a trivial example, that you definitely already know the
answer to.
\begin{example}
  A uniform thin rod of length $L$ with mass $M$ lies along the $x$-axis with
  the left edge located at $x=0$. What is the $x$ coordinate of its centre of
  mass?
  \begin{center}
    \begin{tikzpicture}[thick]
      \draw[axes] (0,0)--(8.5,0) node[right]{$x$};
      \draw (0,0)--+(0,-.3) node[below]{$O$};
      \draw (8,0)--+(0,-.3) node[below]{$L$};
      \draw[mass] (0,0) rectangle (8,.2);
    \end{tikzpicture}
  \end{center}
  \textbf{Solution:} Most students with common sense should know that the
  answer would be $x_\text{cm}=L/2$, but we would nevertheless use
  Eq.~\ref{eq:cm-integral} to confirm that. The integral will from $x=0$ to
  $x=L$, with a constant linear mass density of $\gamma=M/L$:
  \begin{equation*}
    x_\text{cm}=\frac{\int_0^Lx\dl m}M=\frac{\int_0^Lx\gamma\dl x}M=
    \frac{\gamma}M\int_0^Lx\dl x=\frac{\gamma}M\frac{L^2}2=\boxed{\frac L2}
  \end{equation*}
\end{example}
We will continue with a similar, but slightly more difficult case:
\begin{example}
  A thin rod of length $L$ lies along the $x$-axis with the left edge located
  at $x=0$. The rod's linear mass density varies linearly along the $x$-axis:
  $\gamma=ax$ What is the $x$ coordinate of its centre of mass?
  \begin{center}
    \begin{tikzpicture}[thick]
      \draw[axes] (0,0)--(8.5,0) node[right]{$x$};
      \draw (0,0)--+(0,-.3) node[below]{$O$};
      \draw (8,0)--+(0,-.3) node[below]{$L$};
      \shade[left color=white,right color=cyan!50] rectangle (8,.2);
      \draw rectangle (8,.2);
    \end{tikzpicture}
  \end{center}
  \textbf{Solution:} Like the previous example, we will integrate from $x=0$ to
  $x=L$, but since we have non-constant linear mass density, we also have to
  integrate to find the total mass in the denominator:
  \begin{equation*}
    x_\text{cm}
    =\frac{\int_0^Lx\dl m}{\int_0^L\dl m}
    =\frac{\int_0^Lx(\gamma\dl x)}{\int_0^L(\gamma\dl x)}
    =\frac{\int_0^Lax^2\dl x}{\int_0^L(ax\dl x)}
    =\frac{\frac{\cancel{a}L^3}3}{\frac{\cancel{a}L^2}2}
    =\boxed{\frac 23L}
  \end{equation*}
  On first glance, you may think that this is just a math problem, but think
  about what object would have a mass distribution that varies linearly! In
  fact, what we are solving for is the centroid of a triangle:
  \begin{center}
    \vspace{-.2in}
    \begin{tikzpicture}
      \draw[axes] (0,0)--(8.5,0) node[right]{$x$};
      \draw[axes] (0,-.5)--(0,.5) node[above]{$y$};
      \draw[mass] (0,0)--(8,.3)--(8,-.3)--cycle;
      \draw[thick] (5.33,0) circle (.15);
      \fill (5.48,0) arc (0:90:.15)--(5.33,0);
      \fill (5.18,0) arc (180:270:.15)--(5.33,0);
    \end{tikzpicture}
  \end{center}
  \vspace{-.2in}which is $1/3$ the height of the triangle from the base, or in
  this case, a distance of $2L/3$ from the top of the triangle.
\end{example}


\begin{example}
  A triangular plate is placed in a Cartesian coordinate system with two of its
  edges along the $x$ and $y$-axis. The length of the edges along the axes are
  $a$ and $b$ respectively. Assuming that the surface area density $\sigma$ is
  uniform, determine the coordinate of its centre of mass.
  \begin{center}
    \begin{tikzpicture}[scale=.8]
      \draw[axes] (0,0)--(4,0) node[right]{$x$};
      \draw[axes] (0,0)--(0,5) node[above]{$y$};
      \draw[fill=lightgray] (0,0)--(3,0) node[midway,below]{$a$}
      --(0,4)--cycle node[midway,left]{$b$};

      \draw[fill=red] (1,0)--(1.2,0)--(1.2,2.4)--(1,2.67)
      node[midway,right]{$\dl m=y\dl x$}--cycle;
    \end{tikzpicture}
  \end{center}
\end{example}


\subsection{Centroid}

For an object with a uniform mass distribution (i.e.\ constant density, either
$\gamma$, $\sigma$ or $\rho$), the centre of mass is also its geometric centre,
called the \textbf{centroid}. Some 2D examples are shown in
Fig.~\ref{fig:centroids}. Generally, integration for these two-dimensional
shapes is straightforward, but often tedious.
\begin{figure}[ht]
  \centering
  \begin{subfigure}{.3\textwidth}
    \centering
    \begin{tikzpicture}[scale=.45]
      \draw[mass] rectangle (7,4);
      \draw[<->] (-.3,0)--+(0,4) node[midway,left=-1]{$a$};
      \draw[<->] (0,-.3)--+(7,0) node[midway,below=-1]{$b$};
      
      \draw[<->] (3.5,0)--+(0,2) node[midway,right=-1]{$\frac a2$};
      \draw[<->] (0,2)--+(3.5,0) node[midway,above=-1]{$\frac b2$};

      \fill[red] (3.5,2) circle (.08) node[right]{cm};
    \end{tikzpicture}
    \caption{Rectangular Area}
  \end{subfigure}
  \begin{subfigure}{.3\textwidth}
    \centering  
    \begin{tikzpicture}[scale=.45]
      \draw[mass] (0,0)--(5.5,4)--(8,0)--(0,0);
      \draw (5.5,0)--+(0,4);
      \draw[<->] (-.3,0)--+(0,4) node[midway,fill=black!2]{$h$};
      \draw[<->] (0,-.3)--+(5.5,0) node[midway,below=-1]{$b$};
      \draw[<->] (5.5,-.3)--+(2.5,0) node[midway,below=-1]{$a$};
      
      \draw[<->] (11.5/3,0)--+(0,4/3) node[midway,left=-1]{$\frac h3$};
      \draw[<->] (11.5/3,4/3)--(8,4/3) node[midway,above=-1]{
        \small $\frac{b+2a}3$};
      \draw (8,0)--(8,1.5);
      \draw (-.5,4)--(6,4);
      \fill[red] (11.5/3,4/3) circle (.08) node[above]{cm};
    \end{tikzpicture}
    \caption{Triangular Area}
  \end{subfigure}
  \begin{subfigure}{.3\textwidth}
    \centering  
    \begin{tikzpicture}[scale=.45]
      \draw[mass] (0,0)--(0,4.5)--(7,3)--(7,0)--(0,0);
      \draw[<->] (-.3,0)--+(0,4.5) node[midway,left=-1]{$a$};
      \draw[<->] (7.3,0)--+(0,3) node[midway,right=-1]{$b$};
      \draw[<->] (0,-.3)--+(7,0) node[midway,below=-1]{$L$};
      
      \draw[<->] (3.3,0)--+(0,1.9) node[midway,right=-1]{
        $\frac{a^2+ab+b^2}{3(a+b)}$};
      \draw[<->] (0,1.9)--+(3.3,0) node[midway,above=-1]{
        $\frac{L(a+2b)}{2(a+b)}$};

      \fill[red] (3.3,1.9) circle (.08) node[right]{cm};
    \end{tikzpicture}
    \caption{Trapezoidal Area}
  \end{subfigure}
  \begin{subfigure}{.3\textwidth}
    \centering  
    \begin{tikzpicture}[scale=.5]
      \draw[mass] circle (2.5);
      \draw[->,rotate=-50] (0,0)--(2.5,0) node[midway,above=-1]{$r$};
      \fill[red] circle (.08) node[left]{cm};
    \end{tikzpicture}
    \caption{Circular Area}
  \end{subfigure}
  \begin{subfigure}{.3\textwidth}
    \centering  
    \begin{tikzpicture}[scale=.5]
      \draw[black!2] circle (2.5);
      \draw[mass] (2.5,0) arc (0:180:2.5)--(2.5,0);
      \draw[dash dot] (0,0)--(0,2.7);
      \draw[<->] (0,0)--+(0,1.1) node[midway,right=-1]{$\frac{4r}{3\pi}$};

      \draw[->,rotate=-45] (0,0)--(-2.5,0) node[midway,above=-1]{$r$};
      \fill[red] (0,1.1) circle (.08) node[above]{cm};
    \end{tikzpicture}
    \caption{Semi-Circular Area}
  \end{subfigure}
  \begin{subfigure}{.3\textwidth}
    \centering  
    \begin{tikzpicture}[scale=.5]
      \draw[black!2] circle (2.5);
      \draw[mass] (0,0)--(0,2.5) arc (90:180:2.5)--(0,0);
      \draw (-1.1,-.5)--(-1.1,1.1)--(.5,1.1);
      
      \draw[<->|] (.3,1.1)--(.3,0) node[midway,right=-1]{
        \small$\frac{4r}{3\pi}$};
      \draw[<->|] (-1.1,-.3)--(0,-.3) node[midway,below=-1]{
        \small$\frac{4r}{3\pi}$};
      \draw[->,rotate=-45] (0,0)--(-2.5,0) node[midway,above=-1]{$r$};
      \fill[red] (-1.1,1.1) circle (.08) node[left]{cm};
    \end{tikzpicture}
    \caption{Quarter-Circular Area}
  \end{subfigure}
  \caption{Centroid of some 2D shapes}
  \label{fig:centroids}
\end{figure}


\subsection{Compound Shapes}
\label{sec:compound-shape}

For compound shapes, integration or summation may be difficult, even for simple
shapes like the cross section of a T-beam shown in
Fig~\ref{fig:compound-shape}. In such a case, the centre of mass is the
weighted average of the centres of mass of each component. For the T-beam, we
can separate the beam into the horizontal and the vertical parts, and evaluate
the centres of mass of each part, shown as $\bm r_1$ and $\bm r_2$ in
Fig.~\ref{fig:cm-of-each-part}, then find the weighted average of these masses.


\begin{figure}[ht]
  \centering
  \begin{subfigure}{.3\textwidth}
    \begin{tikzpicture}[scale=.7]
      \draw[axes] (0,0)--(4.5,0) node[right]{$x$};
      \draw[axes] (0,0)--(0,4.5) node[left]{$y$};
      \draw[mass] (0,4)--(4,4)--(4,3)--(2.5,3)
      --(2.5,0)--(1.5,0)--(1.5,3)--(0,3)--(0,4);
      \draw (0,4) rectangle (4,3) node[midway]{$m_1$};
      \node at (2,1.5) {$m_2$};
    \end{tikzpicture}
    \caption{A compound shape}
    \label{fig:compound-shape}
  \end{subfigure}
  \begin{subfigure}{.3\textwidth}
    \begin{tikzpicture}[scale=.7]
      \draw[lightgray,fill=gray!20,thick] (0,4)--(4,4)--(4,3)--(2.5,3)--(2.5,0)
      --(1.5,0)--(1.5,3)--(0,3)--(0,4);
      \draw[lightgray,dotted,thick] (1.5,3)--(2.5,3);
      \draw[axes] (0,0)--(4.5,0) node[right]{$x$};
      \draw[axes] (0,0)--(0,4.5) node[left]{$y$};
      \fill[blue!70!black] (2,3.5) circle (.08)
      node[left]{$m_1$}
      node[right]{$\bm r_1$};
      \fill[blue!70!black] (2,1.5) circle (.08)
      node[left]{$m_2$}
      node[right]{$\bm r_2$};
    \end{tikzpicture}
    \caption{Treating each part as a point mass}
    \label{fig:cm-of-each-part}
  \end{subfigure}
  \begin{subfigure}{.3\textwidth}
    \begin{tikzpicture}[scale=.7]
      \draw[lightgray,thick] (0,4)--(4,4)--(4,3)--(2.5,3)
      --(2.5,0)--(1.5,0)--(1.5,3)--(0,3)--(0,4);
      \draw[axes] (0,0)--(4.5,0) node[right]{$x$};
      \draw[axes] (0,0)--(0,4.5) node[left]{$y$};
      \fill[blue!30] (2,3.5) circle (.08) node[right]{$\bm r_1$};
      \fill[blue!30] (2,1.5) circle (.08) node[right]{$\bm r_2$};
      \fill[red!70!black] (2,2.64) circle (.08) node[right]{$\bm r_\text{cm}$};
    \end{tikzpicture}
  \end{subfigure}
  \caption{Finding the centre of mass of a compound shape}
  \label{fig:compound-shape-cm}
\end{figure}


%\begin{frame}{A Difficult Example to Try at Home}
%  Not typically an AP problem, this example shows how we can use integral to
%  find the centre of mass for something very complicated.
%  \begin{columns}
%    \column{.6\textwidth}
%    \textbf{Example 3:} Find the $x$-coordinate of the centre of mass in the
%    shape bound by the two functions shown on the right.
%
%    \column{.4\textwidth}
%    \begin{tikzpicture}[scale=3]
%      \draw[->](0,0)--(1.25,0) node[right]{ $x$};
%      \draw[->](0,0)--(0,1.25) node[above]{ $y$};
%      \draw[fill=green!40]
%      plot[smooth,samples=40,domain=0:1] (\x,{\x*\x*\x})--
%      plot[smooth,samples=40,domain=1:0] (\x,{\x^(.5)});
%      
%      \draw[red!70,thick]  plot[smooth,samples=40,domain=0:1] (\x,{\x*\x*\x});
%      \draw[blue!70,thick] plot[smooth,samples=40,domain=0:1] (\x,{\x^(.5)});
%      \node at (.4,.85){\textcolor{blue!70}{ $y=\sqrt{x}$}};
%      \node at (.9,.3){\textcolor{red!70}{ $y=x^3$}};
%    \end{tikzpicture}
%  \end{columns}


\subsection{Symmetric Configurations}
  \begin{itemize}
  \item Any plane of symmetry, mirror line, axis of rotation, point of inversion
    \emph{must} contain the centre of mass.
  \item Caveat: only works if the density distribution is also symmetric
  \item Again: if density is uniform, CM is also geometric centre (centroid)
  \end{itemize}



\subsection{``Negative Mass'': A Mathematical Trick}
When there is a ``hole'' in the geometry, we can treat the hole as having
negative mass density. Obviously, negative masses don't exist, so this is
really just a trick. For example, to find the centre of mass of the shape
shown in Fig.~\ref{fig:hole}, it is nearly impossible to use the compound-shape
technique in Sec.~\ref{sec:compound-shape}. But we note that the overall shape
(a circle) and the cut out (also a circle) are both simple shapes.
\begin{figure}[ht]
  \centering
  \begin{tikzpicture}
    \draw[thick,fill=lightgray] circle (2);
    \draw[thick,fill=black!2] (0,1) circle (1);
    \fill circle (.04);
    \draw[axes,rotate=45] (0,0)--(-2,0) node[midway,below]{$r$};
    \fill (0,1) circle (.04);
    \draw[axes,rotate around={-45:(0,1)}] (0,1)--(1,1) node[right]{$r/2$};
    \draw[axes](-2.4,0)--(2.4,0) node[right]{$x$};
    \draw[axes](0,-2.4)--(0,2.4) node[above]{$y$};
  \end{tikzpicture}
  \caption{A circle with a hole cut into it}
  \label{fig:hole}
\end{figure}

This is how we would think of it:
\begin{figure}[ht]
  \centering
  \begin{tikzpicture}[scale=.6]
    \draw[thick,fill=lightgray] circle (2);
    \draw[thick,fill=black!2] (0,1) circle (1);
    \draw circle (.04);
    \draw[axes,rotate=45] (0,0)--(-2,0) node[midway,below]{$r$};
    \fill (0,1) circle(.04);
    \draw[axes,rotate around={-45:(0,1)}] (0,1)--(1,1) node[right]{$r/2$};
    
    \draw[thick,fill=lightgray] (6,0) circle (2) node{\large$A$};
    \draw[thick] (11,1) circle (1) node{\large$B$};
    \node at (3,0) {\huge=};
    \node at (9,0) {\huge -};
  \end{tikzpicture}
\end{figure}
%  \item Let the origin of the coordinate system to located at the centre of $A$
%  \item Based on symmetry: $x_\text{cm}=0$; only have to find $y$-coordinate.
%  \end{itemize}
\begin{equation*}
  y_\text{cm}
  =\frac{\sum y_i m_i}{\sum m_i}
  =\frac{m_A(0) + m_B (r/2)}{m_A+m_B}
  =\frac{-\sigma\pi\left(r/2\right)^2(r/2)}
  {\sigma\pi r^2-\sigma\pi\left(r/2\right)^2}
  =\frac{-r}6
\end{equation*}




\section{Momentum and Centre of Mass}

If the components of this system of masses is in motion, the position of the
centre of mass will also evolve in time as well. We can find out the velocity
of the centre of mass by taking the time derivative of $\bm r_\text{cm}$:
\begin{equation}
  \bm v_\text{cm}=\diff{\bm r_\text{cm}}t
  =\frac1m\diff{}t\left(\int\bm r\dl m\right)
  =\frac1m\int\diff{\bm r}t\dl m
  =\frac{\int\bm v\dl m}m
\end{equation}
Or, in a form that is similar to Eq.~\ref{eq:cm-integral}, the velocity of the
centre of mass is the weighted sum/integral of the velocities of the distribution of
mass:
\begin{important-equation}
  \bm v_\text{cm} = \frac{\sum_{i=1}^N\bm m_iv_i}m
  \quad\text{or}\quad
  \bm v_\text{cm} = \frac{\int\bm v\dl m}m
\end{important-equation}
where $m$ is the total mass of the system. We can also rearrange the equation
for the velocity of the centre of mass to relate it to momentum, because the
term $\int\bm v\dl m$ is the net momentum of the mass distribution
$\bm p_\text{net}$:
\begin{equation}
  \bm v_\text{cm} = \frac{\int\bm v\dl m}m
  \quad\longrightarrow\quad
  \bm p_\text{net}=m\bm v_\text{cm}
  \label{eq:mv-cm}
\end{equation}
During a collision, there is no change in the net momentum,
%\footnote{Because
%there are are no external forces},
the centre of mass will continue to move at the same velocity before/after the
collision, as if the collision never occurred.
%During a collision
%\footnote{As we have studied in conservation of momentum in
%Physics 12 and in our previous class}, there are no external forces,
%therefore the velocity of the CM remains constant.
Consider this 1D inelastic collision in between two masses:
\begin{figure}[ht]
  \centering
  \begin{tikzpicture}[scale=.8,thick]
    \begin{scope}[violet]
      \draw (1,0) rectangle (2,1) node[midway]{$m_1$};
      \draw (4,0) rectangle (5,1) node[midway]{$m_2$};
      \draw[vector] (.8,1.3)--(2.5,1.3) node[right]{$v_1$};
      \draw[vector] (4,1.3)--(5,1.3) node[right]{$v_2$};
      \node[above] at (3,1.5) {Before Collision};
    \end{scope}
    \draw[vector] (2.7,.5)--(3.7,.5) node[above]{$v_\text{cm}$};
    \begin{scope}[orange]
      \draw (10,0) rectangle (11,1) node[midway]{$m_1$};
      \draw (11,0) rectangle (12,1) node[midway]{$m_2$};
      \draw[vector] (10.2,1.3)--(11.8,1.3) node[right]{$v'$};
      \node[above] at (11,1.5) {After Collision};
    \end{scope}
    \draw (0,0)--(14,0);
    \draw (2.7,.5) circle (.15);
    \fill (2.7,.5)--(2.85,.5) arc (0:90:.15)--cycle;
    \fill (2.7,.5)--(2.55,.5) arc (180:270:.15)
    node[below=-2] (c){\scriptsize cm\par}--cycle; 

    \draw (11,.5) circle (.15);
    \fill (11,.5)--(11.15,.5) arc (0:90:.15)--cycle;
    \fill (11,.5)--(10.85,.5) arc (180:270:.15)
    node[below=-2] {\scriptsize cm\par}--cycle; 

    \node[text width=2.05in,draw,violet,below right] (a) at (-.5,-.6)
         {Using the definition of the velocity of the CM, we find
           that \emph{before} the collision:
           \vspace{-.1in}
           \begin{displaymath}
             v_\text{cm} = \frac{\sum m_iv_i}{\sum m_i}
             =\frac{m_1v_1+m_2v_2}{m_1+m_2}
           \end{displaymath}\par
         };
         \draw[axes,violet] (a)--(c);
         
         \node[text width=2.05in,draw,orange,below left] at (14.5,-.6)
              {Using momentum conservation, we find that the final
                velocity \emph{after} the collision is:
                \vspace{-.08in}
                \begin{displaymath}
                  v'=\frac{m_1v_1+m_2v_2}{m_1+m_2}=v_\text{cm}
                \end{displaymath}
                \par
              };
  \end{tikzpicture}
\end{figure}



\section{Acceleration of the Centre of Mass}

Take time derivative of the equation for $\bm v_\text{cm}$ gives us the
acceleration of the centre of mass:
\begin{equation}
  \bm a_\text{cm}=\diff{\bm v_\text{cm}}t
  =\frac1m\diff{}t\left(\int\bm v\dl m\right)
  =\frac1m\int\diff{\bm v}t\dl m
  =\frac{\int\bm a\dl m}m
\end{equation}
where $m$ is the total mass. We can, of course, write this in a form that is
similar to Eq.~\ref{eq:cm-integral}, the acceleration of the centre of mass is
the weighted sum/integral of the accelerations of discrete number of masses, or
a distribution of mass:
\begin{important-equation}
  \bm a_\text{cm} = \frac{\sum_{i=1}^N\bm m_ia_i}m
  \quad\text{or}\quad
  \bm a_\text{cm} = \frac{\int\bm a\dl m}m
\end{important-equation}
We can now follow the same procedure, and multiply both sides by the total mass
$m$:
\begin{equation*}
  m\bm a_\text{cm}=\sum_{i=1}^Nm_i\bm a_i
  \quad\text{or}\quad
  m\bm a_\text{cm}=\int\bm a\dl m
\end{equation*}
On the right hand side, each term in the summation (or integral if it is a
continuous distribution of mass) is the net force on the individual mass $m_i$,
so the right hand side is just the net force on the system:
\begin{equation*}
  m_i\bm a_i=\bm F_i
  \quad\longrightarrow\quad
  \sum_{i=1}^Nm_i\bm a_i=\sum_{i=1}^N\bm F_i=\bm F_\text{net}
\end{equation*}
In other words, the net force on an entire system of objects is the total mass
of the system times the acceleration at the centre of mass, which is the
second law of motion:
\begin{equation}
  \bm F_\text{net}=m\bm a_\text{cm}
\end{equation}
We can see that when a net force is applied to an object (or a system of
objects), the object's (or the system's)  acceleration is evaluated at its
centre of mass. We can also get this same conclusion by taking a time
derivative of Eq.~\ref{eq:mv-cm}, i.e\ applying the 2nd law of motion to a
discrete/continuous distribution of mass. Since the system mass is constant,
this equation reduces to:
\begin{equation*}
  \diff{\bm p_\text{net}}t =
  \diff{}t (m\bm v_\text{cm})
  =m\diff{\bm v_\text{cm}}t
  \quad\longrightarrow\quad
  \bm F_\text{net}=m\bm a_\text{cm}
\end{equation*}


\section{First and Second Laws of Motion}
%  \begin{itemize}
%  \item Newton was right all along by treating all objects as point masses
%    located at the CM
%  \end{itemize}

\begin{definition}
  \textbf{First Law:} An object will remain in its state of rest or uniform
    motion, until a net external force is applied to it.
\end{definition}
The first law, called the \emph{law of inertia}, states that when the sum
of all the forces acting on an object, or a system of objects is balanced,
now there are \emph{four} equations that are equivalent:
\begin{important-equation}
  \bm F_\text{net}=\bm 0
  \quad\leftrightarrow\quad
  \bm p_\text{net}=\text{constant}
  \quad\leftrightarrow\quad
  \bm v_\text{cm}=\text{constant}
  \quad\leftrightarrow\quad
  \bm a_\text{cm}=\bm 0
\end{important-equation}



\begin{important-equation}
  \bm F_\text{net}=\diff{\bm p_\text{net}}t=m\bm a_\text{cm}
\end{important-equation}
