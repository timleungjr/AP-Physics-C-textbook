\chapter{Capacitors}

%\section{Parallel-Plate Capacitors}

%\section{Electric Field and Electric Potential Difference}
Recall that electric field $\bm E$ and electric potential difference $V$ can be
expressed using the fundamental theorem of calculus:
%electrostatic force ($\bm F_q$) and
%electric potential energy ($U_q$) can be expressed using definition of
%mechanical work and the fundamental theorem of calculus:
%\begin{equation}
%  \Delta U_q=-\int\bm F_q\cdot\dl\bm r\quad\quad
%  \bm F_q=-\nabla U_q=-\diffp{U_q}r\hat r
%\end{equation}
%Dividing both sides of the equations by $q$, we get the relationship between
%electric field ($\bm E$), electric potential ($V$) and electric potential
%difference ($\Delta V$):
\begin{equation*}
  \Delta V=-\int\bm E\cdot\dl\bm r\quad\quad\longleftrightarrow\quad\quad
  \bm E=-\nabla V%=-\diffp Vr\hat{\bm r}}
\end{equation*}
This relationship holds regardless of the charge configuration. However, for
infinite parallel plates, shown in Fig.~\ref{fig:parallel-plates},
\begin{figure}[ht]
  \centering
  \begin{tikzpicture}
    \draw[thick] rectangle (8,.2);
    \draw[thick] (0,1.4) rectangle (8,1.6);
    \foreach \x in {.4,.8,1.2,...,7.6}{
      \draw[axes] (\x,1.4)--(\x,.7);
      \draw[thick] (\x,1.4)--(\x,.2);
    }
    \foreach\x in {.8,1.6,2.4,...,7.2}{
      \node at (\x,1.78) {$+$};
      \node at (\x,-.28) {$-$};
    }
    
    \draw[axes] (0,1.5)..controls(-.15,1.2)..(-.2,.7);
    \draw[thick] (-.2,.8)..controls(-.15,.4)..(0,.1);
    
    \draw[axes] (8,1.5)..controls(8.15,1.2)..(8.2,.7);
    \draw[thick] (8.2,.8)..controls(8.15,.4)..(8,.1);
  \end{tikzpicture}
  \caption{Electric field between parallel plates}
  \label{fig:parallel-plates}
\end{figure}
the electric field between the plates is uniform, and the relationship between
electric field and potential difference simplifies to:
\begin{equation*}
  \boxed{E=\frac Vd} \quad\text{or}\quad
  V=Ed
\end{equation*}




\section{Parallel-Plate Capacitors}
\textbf{Capacitors} is a device that stores energy in an electric field. The
simplest form of a capacitor is a set of closely spaced parallel plates:
\begin{figure}[ht]
  \centering
  \pic{.5}{../capacitors-calculus/cap19}
\end{figure}
When the plates are connected to a battery, the battery transfer charges to
the plates until the voltage $V$ equals the battery terminals. After that,
one plate has charge $+Q$; the other has $-Q$.

As we have seen already, the (uniform) electric field between two parallel
plates is proportional to the charge density $\sigma$, which is the charge
$Q$ divided by the area of the plates $A$:
\begin{equation}
  E=\frac{\textcolor{red}{\sigma}}{\epsilon_0}=
  \frac{\textcolor{red}{Q}}{\textcolor{red}{A}\epsilon_0}
\end{equation}
Substituting this into the relationship between the plate voltage $V$ and
electric field, we find the relationship between the charges across the
plates and the voltage:
\begin{equation}
  \Delta V=\textcolor{blue} Ed=
  \frac{\textcolor{blue}{Q}d}{\textcolor{blue}{A\epsilon_0}}
  \quad\longrightarrow\quad
  \boxed{Q=\left[\frac{A\epsilon_0}d\right]\Delta V}
\end{equation}
Since area $A$, distance of separation $d$ and the vacuum permittivity
$\epsilon_0$ are all constants, the relationship between charge $Q$ and
voltage $\Delta V$ is \emph{linear}. The constant is called the
\textbf{capacitance} $C$, defined as:
\begin{equation}
  \boxed{C=\frac Q{\Delta V}}
\end{equation}
For parallel plates:
\begin{equation}
  \boxed{C=\frac{A\epsilon_0}d\quad\text{parallel plate}}
\end{equation}
The unit for capacitance is a \textbf{farad} (named after Michael Faraday),
where $\SI1\farad=\SI1{\coulomb\per\volt}$.




\section{Storage of Electrical Energy}
When charging up a capacitor, imagine positive charges moving from the
negatively charged plate to the positively charged plate
\begin{center}
  \pic{.45}{../capacitors-calculus/slide14}
\end{center}
%Initially neither plates are charged, so moving the first charge takes very
%little work; as the electric field builds, more and more work needs to be
%done
In the beginning---when the plates aren't charged---moving an infinitesimal
charge $\dl q$ across the plates, the infinitesimal work done $\dl U$ is very
small and related to the capacitance by:
\begin{equation}
  \dl U=V\dl q=\frac qC\dl q
\end{equation}
As the electric field begins to form between plates, more and more work
is required to move the charges.




To fully charge the plates, the total work $U_c$ is the integral:
\begin{equation}
  U_c=\int\dl U=\int_0^Q\frac qC\dl q=\frac12\frac{Q^2}C
\end{equation}
The work done is stored as a potential energy inside the capacitor. There are
different ways to express $U_c$ using definition of capacitance:
\begin{equation}
  \boxed{U_c=\frac12\frac{Q^2}C=\frac12QV=\frac12CV^2}
\end{equation}



\section{Cylindrical Capacitor}
Not all capacitors are parallel plates. Cylindrical capacitors are also
popular. One such capacitor is shown schematically in
Fig.~\ref{fig:cylindrical-capacitor}.
\begin{figure}[ht]
  \centering
  \begin{subfigure}{.3\linewidth}
    \centering
    \begin{tikzpicture}[scale=.65]
      \fill[magenta!30,opacity=.2] (-2,8)--(-2,0) arc (180:0:2.5 and 1)--(3,8)
      arc (0:180:2.5 and 1);
      \draw[magenta,thick] (-2,8)--(-2,0);
      \draw[magenta,thick] (3,0)--(3,8) arc (0:180:2.5 and 1);
      
      \fill[violet,opacity=.1] (-1,8)--(-1,0) arc (180:0:1.5 and .5) -- (2,8)
      arc (0:180:1.5 and .5) -- (-1,8);
      \draw[dash dot,thick,violet] (-1,8)--(-1,0) arc (180:0:1.5 and .5)--
      (2,8) arc (0:180:1.5 and .5);

      \fill[cyan!50, opacity=.6] (0,8)--(0,0) arc (180:0:.5 and .2) -- (1,8)
      arc (0:180:.5 and .2) -- (0,8);
      \draw[cyan,thick] (0,0)--(0,8) arc (180:0:.5 and .2)--(1,0);
      
      \fill[cyan,opacity=.8] (0,8)--(0,0) arc (180:360:.5 and .2) -- (1,8)
      arc (0:-180:.5 and .2) -- (0,8);

      \fill[violet,opacity=.1] (-1,8)--(-1,0) arc (180:360:1.5 and .5)--(2,8)
      arc (0:-180:1.5 and .5) -- (-1,8);
      \draw[dash dot,thick,violet] (-1,8) arc (180:360:1.5 and .5);

      \fill[magenta!30] (-2,8)--(-2,0) arc (180:360:2.5 and 1)
      --(3,8) arc (0:-180:2.5 and 1);
      \draw[magenta] (-2,8)--(-2,0) arc (180:360:2.5 and 1)
      --(3,8) arc (0:-180:2.5 and 1);

      \draw[|<->|] (3.4,8)--(3.4,0) node[midway,fill=white]{$\ell$};
      \draw[axes,rotate around={10:(.5,8)}](.5,8)--(.95,8) node[right]{$a$};
      \draw[axes,rotate around={-30:(.5,8)}](.5,8)--(2.15,8) node[below]{$b$};
    \end{tikzpicture}
  \end{subfigure}
  \begin{subfigure}{.4\linewidth}
    \centering
    \begin{tikzpicture}[scale=1.2]
      \draw[thick,magenta,fill=magenta!30] circle (1.5);
      \draw[thick,magenta,fill=white] circle (1.45);
      \foreach\theta in {0,30,...,359}
      \draw[vector,orange,rotate=\theta] (.5,0)--(1.45,0);
      \draw[thick,cyan,fill=cyan!50] circle (.5);
      \draw[axes,rotate=30] (0,0)--(.5,0) node[midway,below]{$a$};
      \draw[axes,rotate=110] (0,0)--(1.45,0) node[midway,right]{$b$};
      \node[orange] at (-.2,-1){$\bm E$};
      \draw[violet,thick,dash dot] circle (1);
      \draw[thick,<-,violet] (-.707,-.707)--+(-1,-1)
      node[below=-3]{Gaussian Surface};
    \end{tikzpicture}
  \end{subfigure}
  \caption{A cylindrical capacitor}
  \label{fig:cylindrical-capacitor}
\end{figure}
The capacitor consists of an inner cylinder of radius $a$, and an outer shell
of radius $b$. For the analysis, we will give the inner cylinder of a charge
$+Q$. The outer shell therefore has a charge of $-Q$. The capacitor has length
$\ell$ which is much larger than the inner and outer radii, i.e.\
$\ell\gg a$, $\ell\gg b$. Inside the capacitor, there is an  electric field in
the radial direction, but outside of the capacitor, there is no electric field.

Using Gauss's law, we can find the electric field between the cylinders, by
placing a cylindrical Gaussian surface of radius $a<r<b$ and length $\ell$
between the cylinders: 
\begin{equation*}
  \oint\bm E\cdot\dl\bm A=EA=2\pi r\ell E=\frac Q{\epsilon_0}
\end{equation*}
which gives the expression:
\begin{equation}
  E=\frac Q{2\pi r\ell\epsilon_0}%\quad\text{or}\quad
  %E=\frac\lambda{2\pi r\epsilon_0}
\end{equation}
%where $\lambda=Q/\ell$ is the (constant) linear charge density.
Once we have the electric field strength in terms of charge, we can relate the
electric field to the magnitude of the potential difference (voltage) across
the plates:
\begin{equation}
  \Delta V= \int_a^b E\dl r=\frac Q{2\pi\ell\epsilon_0}\int_b^a\frac{\dl r}r
  =\frac Q{2\pi\ell\epsilon_0}\ln\left[\frac ba\right]
\end{equation}
Like parallel plates, the relationship between voltage and charge is still
linear, but in this case, the capacitance is defined as:  
\begin{equation}
  C=\frac Q{\Delta V}=\frac{2\pi\ell\epsilon_0}{\ln(b/a)}
  \quad\longrightarrow\quad
  \boxed{\frac C\ell=\frac{2\pi\epsilon_0}{\ln(b/a)}\;\;\text{cylindrical}}
\end{equation}
The capacitance is generally expressed by $C/\ell$ (with unit
\si{\farad\per\metre}). Like the parallel-plate capacitor, the capacitance of
the cylindrical capacitor also only depends on the geometry (i.e.\ the radii
$a$ and $b$) and the permittivity.


\section{Spherical Capacitor}

\begin{figure}[ht]
  \centering
  \begin{subfigure}{.35\linewidth}
    \centering
    \begin{tikzpicture}
      \shade[ball color=cyan] circle (.7);
      \draw[thick,cyan] circle (.7);
      \shade[ball color=magenta,opacity=.3] circle (2);
      \draw[thick,magenta] circle (2);
      \draw[axes,rotate=100] (0,0)--(.7,0) node[midway,right]{$a$};
      \draw[axes,rotate=-130] (0,0)--(2,0) node[midway,below]{$b$};
    \end{tikzpicture}
  \end{subfigure}
  \begin{subfigure}{.35\linewidth}
    \centering
    \begin{tikzpicture}
      \draw[thick,magenta,fill=magenta!30] circle (2);
      \draw[thick,magenta,fill=white] circle (1.95);
      \draw[thick,cyan,fill=cyan!30] circle (.7);
      \draw[axes,rotate=100] (0,0)--(.7,0) node[midway,right]{$a$};
      \draw[axes,rotate=-120] (0,0)--(1.95,0) node[midway,below]{$b$};
      \foreach \theta in {15,45,...,360}{
        \draw[vector,orange,rotate=\theta] (.7,0)--(1.95,0);
      }
      \draw[thick] (0,-2)--(0,-2.5)--(2.5,-2.5)
      to[battery,l_=$\mathcal E$] (2.5,0)--(2.05,0) arc (0:-180:.08)--(.7,0);
    \end{tikzpicture}
  \end{subfigure}
  
  \caption{A spherical capacitor}
\end{figure}
Regardless of the geometry of the capacitor, the electric field will always
be proportional to the charge, i.e.:
\begin{equation*}
  E\propto Q
\end{equation*}
and therefore the voltage will always be proportional to charge as well.




\section{Dielectric Constants}
\begin{figure}[ht]
  \centering
  \pic{.5}{../capacitors-calculus/Figure_20_05_05a}
\end{figure}

\begin{itemize}
\item Capacitors (both parallel-plate and cylindrical) are very common in
  electric circuits, but the vacuum between the plates is not very effective
\item Instead, a non-conducting \textbf{dielectric} material is inserted
  between the plates
\item When the plates are charged, the electric field of the plates
  polarizes the dielectric.
\item The polarization  produces an electric field that opposes the field
  from the plates, therefore reduces the effective voltage, and increasing
  the capacitance
\end{itemize}

If electric field without dielectric is $E_0$, then $E$ in the dielectric is
reduced by $\kappa$, the \textbf{dielectric constant}:
\begin{equation}
  \boxed{\kappa=\frac{E_0}E}
\end{equation}
The capacitance of the plates with the dielectric is now amplified by the
same factor $\kappa$:
\begin{equation}
  \boxed{C=\kappa C_0}
\end{equation}
We can also view the dielectric as something that increases the
\emph{effective permittivity}:
\begin{equation}
  \boxed{\epsilon=\kappa\epsilon_0}
\end{equation}


\begin{table}[ht]
  \centering
  \begin{tabular}{l|l}
    \rowcolor{pink}
    \textbf{Material} & $\kappa$ \\ \hline
    Air         & \num{1.00059} \\
    Bakelite    & \num{4.9} \\
    Pyrex glass & \num{5.6} \\
    Neoprene    & \num{6.9} \\
    Plexiglas   & \num{3.4} \\
    Polystyrene & \num{2.55} \\
    Water (\SI{20}\celsius) & \num{80} 
  \end{tabular}
  \caption{Dielectric constants of commonly used materials}
\end{table}



\section{Notes About Storage of Electric Energy}
The work done (i.e.\ the energy stored in the capacitor) is inversely
proportional to the capacitance:
\begin{equation}
  \dl U=V\dl q=\frac qC\dl q
\end{equation}
\begin{itemize}
\item The presence of a dielectric \emph{increases} the capacitance; therefore
  the work (and potential energy stored) to move the charge $\dl q$
  \emph{decreases} with the dielectric constant $\kappa$
\item After the capacitor is charged, removing the dielectric material from
  the capacitor plates will require additional work.
\end{itemize}



\section{Capacitors in Electric Circuits}
Capacitors are an important part of an electric circuits because it acts as
an energy storage device. We can therefore connect it to a resistor, as shown
in Fig.~\ref{fig:very-basic-RC}.
\begin{figure}[ht]
  \centering
  \begin{tikzpicture}[scale=.85]
    \draw[thick] (0,0) to[C=$C$] (0,3)--(3,3) to[R=$R$] (3,0)--(0,0);
  \end{tikzpicture}
  \caption{A basic circuit with a capacitor acting as an energy source}
  \label{fig:very-basic-RC}
\end{figure}
In this configuration, the capacitor acts like a voltage source that drives a
current. The energy stored in the capacitor can then be dissipated as heat
through the resistor.  Unlike a battery, which stores energy \emph{chemically},
the voltage decreases as it drives a current, and the charge across the
capacitor plates decreases. This type of circuit is called an $RC$ circuit, and
is discussed in depth in Chapter~\ref{chapter:circuits2}.
