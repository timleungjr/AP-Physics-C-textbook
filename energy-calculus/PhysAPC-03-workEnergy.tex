\chapter{Work and Energy}
\label{chapter:energy}

We start with some definition at are (unfortunately) not very useful:
\begin{itemize}
\item \textbf{Energy} is the ability to do work.
\item \textbf{Work} is the mechanism in which energy is transformed.
\end{itemize}
At a minimum, these definitions at least tell two things:
\begin{itemize}
\item The concepts of work and energy cannot be separated: Defining work
  without leading to energy is rather pointless; defining energy without first
  referring to work makes no sense.
\item Word definitions are not enough. We also need to have \emph{mathematical}
  definitions as well.
\end{itemize}
It is customary to start with the definition of mechanical work, and use it
to define different forms of energies.

\section{Mechanical Work}
\label{sec:mechwork}
An infinitesimal amount of \textbf{mechanical work} $\dl W$ is done when a
force $\mathbf F$ displaces an object by an infinitesimal amount
$\dl\mathbf x$. If the force moves an object along the path $\mathcal C$, the
total work $W$ done by the force is defined by the integral:
\begin{equation}
  \boxed{
    W=\int_{\mathcal C}\dl W=\int_{\mathcal C}\mathbf F(\mathbf x)\cdot\dl\mathbf x
  }
  \label{work-definition}
\end{equation}
While both force and displacement are vectors, work is a scalar quantity. Work
by $\bm F$ can be positive or negative depending on the dot product. In
general, the amount of work done ($W$) depends on the path $\mathcal C$. %From
%Eq.~(\ref{work-definition}), we recognize that
No work is done if the force is perpendicular to displacement, (i.e.\
$\bm F\cdot\dl\bm x=0$) which means that the force did not \emph{cause} the
displacement, or if the object does not move, (i.e.\ $\dl\bm x=\bm 0$), or if
no force is applied during motion (i.e.\ $\bm F=\bm 0$).

For motion confined to one direction along a one-dimensional coordinate
system, Eq.~(\ref{work-definition}) reduces to\footnote{Note that direction
still matters for $F$ and $x$, even in 1D, in that there is still a positive
and negative direction (if $F$ and $\dl x$ are in the same direction, then
$W>0$; and if $F$ and $\dl x$ are in opposite direction, then $W<0$).}:
\begin{equation}
  \boxed{
    W=\int_{x_0}^{x_1} F(x)\dl x
  }
\end{equation}
For a constant force that moves an object along a straight path, the integral
simplifies to just the dot product of two vectors\endnote{We must remember how
  to express the dot products. If you know the angles between the vectors,
  then:
  \begin{equation*}
    \bm A\cdot\bm B= AB\cos\theta
  \end{equation*}
  And if you know the components of the vectors, then
  \begin{equation*}
    \bm A\cdot\bm B= A_xB_x+A_yB_y+A_zB_z
  \end{equation*}
}:
\begin{equation}
  \boxed{
    W = \bm F\cdot\Delta\bm x=F\Delta x\cos\theta
  }
  \label{eq:no-integration}
\end{equation}
where $\theta$ is the angle between the force and displacement vectors. We can
also use the above equation if the force $\bm F$ is \emph{averaged} over
the displacement, i.e.\
\begin{equation}
  \bm F=\bm F_\text{avg}=\frac{\int F(x)\dl x}{\Delta x}
\end{equation}
At this moment, it is unclear what the force $\bm F$ is. Whereas we use the
\emph{net force} when calculating acceleration in dynamics problems, when
calculating the work done by ``a force'', it could mean:
\begin{enumerate}[leftmargin=12pt]
\item\textbf{Work done by a \emph{specific} force.} There may be multiple
  forces acting on an object. We can calculate, based on the motion of the
  object, how much work is done by each force. Calclating the work done by a
  specific force allows us to study the exact mechanism in how energy is
  transformed.
  
\item\textbf{Work done by the \emph{net} force}, in other words, the \emph{sum}
  of all the work done by each force. This is also called the \textbf{net work}
  $W_\text{net}$:
  \begin{equation*}
    W_\text{net}
    = \sum_i W_i
    =\int_{\mathcal C}\bm F_\text{net}(x)\cdot\dl\bm x
  \end{equation*}
  The net work allows us to study what is the overall conservation of energy
  to the entire object.
\end{enumerate}

\fcolorbox{black}{yellow!10}{
  \small
  \begin{minipage}{.97\linewidth}
    \textbf{A simple example}: a worker pushes a heavy crate up a ramp with a
    varying applied force. The free-body diagram for this problem is shown
    below. Here, four forces act on the crate: gravity $\bm F_g$, normal force
    $\bm F_n$, kinetic friction $\bm f$, and applied force $\bm F_a(x)$, which
    is expressed as a function of the crate's displacement as it moves.
    \begin{center}
      \begin{tikzpicture}
        \draw[thick] (0,0)--(5,0);
        \draw[axes] (3,0) arc (0:25:3) node[midway,right]{$\theta$};
        \begin{scope}[rotate=25]
          \draw[thick] (0,0)--({5/cos(25)},0);
          \draw[thick,|->] (3.5,1.75)--+(2,0) node[right]{$\Delta x$};
          \draw[mass] (1.5,0) rectangle (3.5,1.5);
          \fill (2.5,.75) circle (.07);
          \draw[vector] (2.5,.75)--+(1.5,0) node[right]{$\bm F_a(x)$};
          \draw[vector] (2.5,.75)--+(0,1.5) node[above]{$\bm F_n$};
          \draw[vector,rotate around={-25:(2.5,.75)}]
          (2.5,.75)--+(0,-1.25) node[below]{$\bm F_g$};
          \draw[vector] (2.5,.75)--+(-1,0) node[left]{$\bm f$};
        \end{scope}
      \end{tikzpicture}
    \end{center}
    We can calculate the work done by each force:
    \begin{itemize}[leftmargin=12pt]
    \item Work done by the normal force is zero, because $\bm F_n$ is
      perpendicular to the direction of motion:
      \begin{equation*}
        W_n=\int\underbrace{\bm F_n\cdot\dl\bm x}_{=0}=0
      \end{equation*}
    \item Work done by kinetic friction is negative, because $\bm f$ is in the
      opposite direction to motion: %and therefore the dot product gives a $-1$.
      \begin{equation*}
        W_f=\int\bm f\cdot\dl\bm x=-\int f\dl x=-\mu_k F_n\Delta x<0
      \end{equation*}
    \item Work done by gravity is also negative, because the component of
      gravity along the direction of motion is in the opposite direction. (This
      is badly written!!)
    \item The work done by the applied force is positive, because applied force
      is in the same direction as motion.
      \begin{equation*}
        W_a(x)=\int_{x_0}^x F_a\dl x>0
      \end{equation*}
    \end{itemize}
    The total work (i.e.\ the net work) done by summing the work done by each
    force:
    \begin{align*}
      W_\text{net} &=W_a+W_n+W_g +W_f\\
      &=\int\left(\bm F_a+\bm F_n+\bm F_g+\bm F_f\right)
      \cdot\dl\bm x =\int\bm F_\text{net}\cdot\dl\bm x
    \end{align*}
  \end{minipage}
}


\section{Kinetic Energy}
\label{sec:transKE}
When a net force acts on an object (with constant mass) to accelerate it, work
is done, For motion in one dimension, the total/net work done is given by:
\footnote{When you do this problem ``properly'' in 2D or 3D, the only
difference is the dot product, which only \emph{slightly} increases the
complexity of the problem from the 1D case:
\begin{equation*}
  W_\text{net}
  = \int\bm F_\text{net}(\bm x)\cdot\dl\bm x
  = \cdots = m\int\bm v\cdot\dl\bm v
\end{equation*}
The difference in the dot product is a lot easier to evaluate than you might
think:
\begin{align*}
  m\int\bm v\cdot\dl\bm v
  &=m\left(\int v_x\dl v_x + \int v_y\dl v_y + \int v_z\dl v_z\right)\\
  &=\left(\frac12mv_x^2+\frac12mv_y^2 + \frac12mv_z^2\right)
  =\frac12mv^2
\end{align*}
where $v^2=v_x^2+v_y^2+v_z^2$. This is, of course, the same result that we
got from the one-dimensional case.}
\begin{equation}
  W_\text{net}
  =\int_{x_0}^{x_1}F_\text{net}(x)\dl x
  =\int ma\dl x
  =m\int\diff vt\dl x
\end{equation}
Since both $v(t)$ and $x(t)$ are continuously differentiable in time, we can
switch the order of the differentiation:
\begin{equation*}
  =m\int\diff xt\dl v=m\int_{v_0}^{v_1}v\dl v
\end{equation*}
The limits of integration switch from the initial and final position ($x_0$ and
$x_1$) to the initial and final velocities, where $v_0=v(x_0)$ and
$v_1=v(x_1)$. Evaluating this integral, we have:
\begin{equation*}
  =m\int_{v_0}^{v_1}v\dl v
  =\frac12mv^2\Big|^{v_1}_{v_0}
  =\frac12mv_1^2-\frac12mv_0^2
  =\Delta K
\end{equation*}
where $K$ is defined as the \textbf{kinetic energy}\footnote{This is more
specifically called the \emph{translational} kinetic energy, and it should be
distinguished from the \emph{rotational kinetic energy} which is used when an
object is rotating about a pivot.}:
\begin{equation}
  \boxed{
    K=\frac12mv^2
  }
\end{equation}
The definition of kinetic energy came from this integration: when the net force
on an object is doing work, that work is equal to the change in
\emph{something}, and we \emph{define} that quantity as the kinetic energy.
This is known as the \textbf{work-energy theorem}\footnote{Also known as the
\textbf{work-energy principle}, and \textbf{work-energy relationship}}:
\begin{equation}
  \boxed{
    W_\text{net}=\Delta K
  }
  \label{work-energy-theorem}
\end{equation}
When multiple forces act on an object, \emph{positive} net work will
\emph{increase} the kinetic energy of the object, while \emph{negative} net
work will decreases kinetic energy. Eq.~(\ref{work-energy-theorem}) applies
regardless of \emph{what} the net force is comprised of. Very importantly, the
work-energy theorem turns a potentially difficult integration problem
(integrating $W_\text{net}$) into a simple algebraic expression ($\Delta K$).


%  \textbf{Example 1:} A net force $F=4x$ (in newtons) acts on an object of mass
%  \SI2{\kilo\gram} as it moves along the $x$-axis from $x=1$ to $x=\SI5\metre$.
%  Given that the object is at rest at $x=1$,
%  \begin{enumerate}[(a)]
%  \item Calculate the net work
%  \item What is the final speed of the object?
%  \end{enumerate}


\section{Potential Energy}

Unlike kinetic energy, forms of energy that can be stored are called
\textbf{potential energy}.

\subsection{Gravitational Force \& Gravitational Potential Energy}
Consider an object that is dropped (free-falling) under the force of gravity
over a distance of $\Delta x$, shown in Fig.~\ref{fig:falling1}.
\begin{figure}[ht]
  \centering
  \begin{tikzpicture}[scale=.6]
    \draw[thick,fill=gray!10] (7.75,0) arc (75:105:30);
    \draw[dashed] (-5,1.05)--+(10,0);
    \draw[dashed] (-5,6)--+(10,0);
    \draw[dashed] (-5,3)--+(10,0);
    \draw[mass] (0,6) circle (.2) node[right=2]{$m$};
    \draw[vector,red] (0,6)--+(0,-2) node[right]{$\bm F_g$};
    \draw[vector] (-.5,6)--+(0,-3) node[midway,left]{$\Delta x$};
    \draw[vector,blue] (-1.7,1.05)--+(0,4.95) node[midway,left]{$h_0$};
    \draw[vector,blue] (-1,1)--+(0,2) node[midway,right]{$h_1$};
  \end{tikzpicture}
  \caption{Gravitational force doing positive work on a free-falling object.}
  \label{fig:falling1}
\end{figure}

When displacement $\Delta\bm x$ is small, acceleration due to gravity
$\bm g$ can be considered to be constant, therefore
$\bm F_g=m\bm g$ is a constant force. Work done by gravity can be
calculated using Eq.~\ref{eq:no-integration}, and no integration is needed.
Since both $\bm F_g$ and $\Delta\bm x$ are in the same direction, work
done by gravity ($W_g$) is positive. From the work-energy theorem
(Eq.~\ref{work-energy-theorem}), there is an increase in kinetic energy, and
the object speeds up:
\begin{equation*}
  W_g=mg\Delta x>0 \quad\longrightarrow\quad \Delta K > 0
\end{equation*}
This is consistent with our understanding of kinematics and dynamics. But the
work done by gravitational force can also be expressed in terms of the change
in height. Using ground as the reference level (i.e.\ $h=0$), the work done by
gravity can be written as:
\begin{equation*}
  W_g = mg(h_0-h_1)
\end{equation*}
We can further modify this equation:
\begin{align}
  W_g &= mg(h_0-h_1)\nonumber\\
  & = -mg(h_1-h_0)\nonumber\\
  & = -(mgh_1-mgh_0)\nonumber\\
  W_g &= -\Delta U_g
\end{align}
where $U_g$ is defined as the \textbf{gravitational potential energy}:
\begin{equation}
  \boxed{U_g=mgh}
\end{equation}
Since the choice of the reference level (where we define $h=0$) is arbitrary,
we are more interested in the \emph{change} in gravitational potential energy,
which is related to the work done by gravity:
\begin{equation}
  \boxed{
    W_g=-\Delta U_g
  }\quad\text{where}\quad
  \boxed{
    \Delta U_g=mg\Delta h
  }
  \label{work-potential-energy}
\end{equation}
In Eq.~\ref{work-potential-energy}, we note some special relationships between
the work done by gravity ($W_g$) and the change in the gravitational potential
energy ($U_g$):

\fcolorbox{black}{yellow!10}{
  \begin{minipage}{.95\textwidth}
    \begin{itemize}[nosep,leftmargin=10pt]
    \item \emph{Positive} work by gravity \emph{decreases} gravitational
      potential energy, while
    \item \emph{Negative} work by gravity \emph{increases} gravitational
      potential energy
    \item Work by gravity is \emph{path independent}: $W_g$ depends on the end
      points $h_0$ and $h_1$, but not \emph{how} it goes from
      $h_0\rightarrow h_1$
    \item Only work done by gravity can affect $U_g$
    \end{itemize}
  \end{minipage}
}

As you can see in the example in Fig.~\ref{path-independence}, the work done by
gravity is the same in all cases. That is not to say that there are no other
forces acting on the object; we have merely isolated the work done by the
gravitational force alone.
\begin{figure}[ht]
  \centering
  \begin{tikzpicture}
    \draw[thick,dashed] (0,0)--+(11,0) node[right]{$h_0$};
    \draw[thick,dashed] (0,-2)--+(11,0) node[right]{$h_1$};
    
    \fill[red] (.7,0) circle (.07);
    \draw[vector,red] (.7,0)--+(0,-2);
    \draw[thick,red] (.7,-2) circle (.07);
    \node[below,red] at (.7,-2.3){Dropped};
    
    \fill[violet] (3.5,0) circle (.07);
    \draw[vector,violet] (3.5,0)--+(0,1.2) arc(180:0:.05)--+(0,-3.2);
    \draw[thick,violet] (3.6,-2) circle (.07);
    \node[below,violet] at (3.55,-2.3){Thrown straight up};
      
    \fill[orange] (5.5,0) circle (.07);
    \draw[vector,orange] (5.5,0) to[out=50,in=120] +(2,-2);
    \draw[thick,orange] (7.5,-2) circle (.07);
    \node[below,orange] at (6.5,-2.3){Projectile};
      
    \fill[magenta] (8,0) circle (.07);
    \draw[vector,magenta] (8,0) to[out=-50,in=230] +(2.5,-2);
    \draw[thick,magenta] (10.5,-2) circle (.07);
    \node[below,magenta] at (9.25,-2.3){Arbitrary surface};
  \end{tikzpicture}  
  \caption{Work done by gravity ($W_g$) is the same in all above cases because
    they all have the same initial and final height.}
  \label{path-independence}
\end{figure}



\subsection{Law of Universal Gravitation}

\begin{figure}[ht]
  \centering
  \begin{tikzpicture}[scale=.65]
    \draw[vector,red] (0,0)--(2,0) node[right]{$\bm F_g$};
    \draw[vector,blue] (8,0)--(6,0) node[left]{$\bm F_g$};
    \shade[balloon1] circle (.7) node[white]{$m_1$};
    \shade[balloon2] (8,0) circle (1) node[white]{$m_2$};
    \draw[dashed] (0,0)--(0,-1.5);
    \draw[dashed] (8,0)--(8,-1.5);
    \draw[<->,thick] (0,-1.3)--(8,-1.3) node[midway,fill=white]{$r$};
  \end{tikzpicture}
  \caption{Gravity is a mutual attraction between massive objects}
\end{figure}

\textbf{Gravity} is the mutually attractive force between
massive objects. The magnitude of gravitational force between two
\textbf{point masses} is proportional to their masses ($m_1$, $m_2$), and
inversely proportional to the square of the distance ($r$) between them:
\begin{equation}
  \boxed{F_g=\frac{Gm_1m_2}{r^2}}
\end{equation}
where $G=\SI{6.67e-11}{N.m^2/kg^2}$ is the \textbf{universal gravitational
  constant}



%\begin{frame}{Universal Gravitation}
%  \begin{itemize}
%  \item If $m_1$ exerts a force $\bm F_g$ on $m_2$, then $m_2$ also exerts a
%    force $-\bm F_g$ on $m_1$. The attractive forces are equal in magnitude
%    and opposite in direction (third law of motion).
%  \item $m_1$ and $m_2$ are \emph{point masses} that do not occupy any space
%  \item For objects with \emph{spatial extend}\footnote{It means that the mass
%    actually takes up space}, the law assumes that one object is not inside the
%    other
%  \end{itemize}
%\end{frame}
%
%
%
%\begin{frame}{A Simple Example Problem}
%  \textbf{Example:} A \SI{65.0}{\kilo\gram} astronaut is walking on the surface
%  of the moon, which has a mean radius of \SI{1.74e3}{\kilo\metre} and a mass
%  of \SI{7.35e22}{\kilo\gram}. What is the weight of the astronaut?
%\end{frame}
%
%
%
%\begin{frame}{Example Problem}
%  \textbf{Example:} How far apart would you have to place two
%  \SI{7.0}{\kilo\gram} bowling balls so that the force of gravity between them
%  would be \SI{1.25e-8}\newton?
%
%  \vspace{.5in}Notice the magnitude of gravitational force between the two
%  objects. In fact, gravitational force is the weakest of all fundamental
%  forces.
%\end{frame}
%
%
%
%\begin{frame}{Relating Gravitational Field \& Gravitational Force}
%  When a mass $m$ is placed inside a gravitational field $\bm g$, it
%  experiences a gravitational force given by the familiar equation:
%  %$\bm g$ itself doesn't do anything unless another mass $m$ is inside this
%  %field. At which point, the other mass $m$ experiences a gravitational force
%  %related to $\bm g$ by:
%  
%  \eq{-.1in}{
%    \boxed{\bm F_g=m\bm g}
%  }
%
%  Regardless of what generated the gravitational field. When there are multiple
%  source masses, the total gravitational field is the vector sum of all the
%  gravitational fields from each source mass.
%
%  \eq{-.1in}{
%    \bm g =\bm g_1+\bm g_2+\bm g_3+\bm g_4+\cdots
%  }
%\end{frame}



The expression for \textbf{gravitational potential energy} can be obtained
from the law of universal gravitation using basic integral calculus:

\begin{equation}
  W_g = \int_{\bm r_1}^{\bm r_2}\bm F_g\cdot\dl\bm r
  = Gm_1m_2 \int_{r_1}^{r_2}\frac{\dl r}{r^2} = -\Delta U_g
\end{equation}
where we again define the gravitational potential energy stored between two
point masses $m_1$ and $m_2$:
\begin{equation}
  \boxed{U_g=-\frac{Gm_1m_2}r}
\end{equation}
$U_g$ is the work required to move two objects from $r$ to $\infty$. $U_g=0$ at
$r=\infty$ and \emph{decrease} as $r$ decreases




%\begin{frame}{Gravitational Potential Energy}
%  Since $g$ is not a constant, we use an equation consistent with the law of
%  universal gravity to obtain the general expression for
%  \textbf{gravitational potential energy} stored between a system of two
%  masses:
%  
%  \eq{-.05in}{
%    \boxed{U_g=-\frac{Gm_1m_2}r}
%  }
%  \begin{center}
%    \begin{tabular}{l|c|c}
%      \rowcolor{pink}
%      \textbf{Quantity} & \textbf{Symbol} & \textbf{SI Unit} \\ \hline
%      Gravitational potential energy & $U_g$ & \si\joule \\
%      Point masses & $m_1$, $m_2$ & \si{\kilo\gram} \\
%      Distance between centres of mass & $r$ & \si\metre \\
%      Universal gravitational constant & $G$ & \si{N.m^2/kg^2}
%    \end{tabular}
%  \end{center}
%  The ``reference level'' is chosen at infinity (i.e.\ $U_g=0$ at $r=\infty$)
%  and \emph{decrease} as $r$ decreases
%\end{frame}




As we look at work done by other forces, we will begin to see the same pattern
emerge for other forces.

\subsection{Spring Force \& Elastic Potential Energy}
The \textbf{spring force} $\bm F_s$ is the force that a
compressed/stretched spring exerts on the object connected to it, shown in
Fig.~\ref{hooke1}. An \emph{ideal} spring obeys \textbf{Hooke's law}, which
states that the spring force on a compressed/stretched spring is proportional
to the amount of spring displacement $\bm x$, and acts in the opposition
to the displacement:
\begin{equation}
  \boxed{
    \bm F_s=-k\bm x
  }
\end{equation}
The constant $k$ is called the \textbf{spring constant}\footnote{The spring
constant is also called the \textbf{force constant}, \textbf{Hooke's constant},
and in many engineering textbooks, \textbf{spring rate}.}, with a unit of
\emph{newton per meter} (\si{\newton\per\metre}). The spring constant depends
on the geometry of the spring, as well as the material that it is made of.
\begin{figure}[ht]
  \centering
  \begin{tikzpicture}
    \draw[mass] (5,.5) rectangle (6,1.5);
    \draw[thick,
      decoration={aspect=.6,segment length=5mm, amplitude=2.5mm, coil},
      decorate] (0,1)--(5,1);
    \fill[pattern=north east lines] (-.2,0) rectangle (0,2);
    \draw[thick] (0,.0)--(0,2);
    \fill[red] (5.5,1) circle (.06);
    \draw[vector,red] (5.5,1)--(4,1) node[above]{$\bm F_s$};
    \draw[dashed] (3,0)--(3,2) node[above]{\small unstretched/equilibrium};
    \draw[vector] (3,.3)--(5,.3) node[midway,below]{$\bm x$};
  \end{tikzpicture}
  \hspace{.2in}
  \begin{tikzpicture}
    \fill[pattern=north east lines] (-.2,0) rectangle (0,2);
    \draw[thick] (0,0)--(0,2);
    \draw[dashed] (3,0)--(3,2);
    \draw[mass] (1.5,.5) rectangle (2.5,1.5);
    \draw[thick,decorate,
      decoration={aspect=.3,segment length=1.5mm, amplitude=2.5mm, coil}]
    (0,1)--(1.5,1);
    \draw[vector] (3,.3)--(1.5,.3) node[midway,below]{$\bm x$};
    \fill[red] (2,1) circle (.06);
    \draw[vector,red] (2,1)--(3,1) node[above]{$\bm F_s$};
  \end{tikzpicture}
  \caption{Direction of spring force is always opposite to spring
    displacement.}
  \label{hooke1}
\end{figure}

As the spring force moves a mass connected to the spring, the work done by
the spring force is:
\begin{equation}
  W_s=\int_{x_0}^{x_1}F_s\dl x =-k\int_{x_0}^{x_1} x\dl x
  =-\frac12kx^2\Big|^{x_1}_{x_0}=-\Delta U_s
  \label{eq:spring-pot1}
\end{equation}
where $U_s$ is now defined as the \textbf{elastic potential energy}:
\begin{equation}
  \boxed{
    U_s=\frac12kx^2
  }
\end{equation}
Crucially, the work done by the spring force is related to the elastic
potential energy by:
\begin{equation}
  \boxed{
    W_s=-\Delta U_s
  }
\end{equation}
The integration in Eq.~\ref{eq:spring-pot1} shows the same properties as in
the work done by gravitational force.

\fcolorbox{black}{yellow!10}{
  \begin{minipage}{.95\textwidth}
    \begin{itemize}[nosep,leftmargin=10pt]
    \item \emph{Positive} work by the spring \emph{decreases} spring potential
      energy, while
    \item \emph{Negative} work by the spring \emph{increases} spring potential
      energy
    \item Work by the spring force is \emph{path independent:} $W_s$ depends on
      the end points $x_0$ and $x_1$, but not \emph{how} it moves from $x_0$ to
      $x_1$
    \item Only work done by $\bm F_s$ can affect $U_s$
    \end{itemize}
  \end{minipage}
}


\subsection{Electrostatic Force \& Electric Potential Energy}
Similar to the attractive force between masses, there is also a mutually
attractive or repulsive force between charged particles, given by
\textbf{Coulomb's law}:
\begin{equation}
  \boxed{
    \bm F_q=\frac{kq_1q_2}{r^2}\hat{\bm r}
  }
\end{equation}

The integral is nearly identical to that for the gravitational force:
\begin{equation}
  W_q=\int\bm F_q\cdot\dl\bm r
  =kq_1q_2\int_{r_0}^{r_1}\frac{\dl r}{r^2}
    =-\frac{kq_1q_2} r\Big|^{r_2}_{r_1}=-\Delta U_q
\end{equation}
where $U_q$ is the \textbf{electric potential energy} that is stored between
the two point charges, defined as:
\begin{equation}
  \boxed{
    U_q = \frac{kq_1q_2}r
  }
\end{equation}

Again, we see the same properties that we have observed 

\fcolorbox{black}{yellow!10}{
  \begin{minipage}{\textwidth}
    \begin{itemize}[nosep,leftmargin=10pt]
    \item\emph{Positive} work by the electric force \emph{decreases} electric
      potential energy, while
    \item\emph{Negative} work by the electric force \emph{increases} electric
      potential energy
    \item $W_q$ depends on the end points $r_0$ and $r_1$, but not \emph{how}
      it went from $r_0\rightarrow r_1$
    \item Only work done by $\bm F_q$ can affect $U_q$
    \end{itemize}
  \end{minipage}
}


\section{Conservative Forces}

These forces are called \textbf{conservative forces}
\begin{itemize}[nosep]
\item Gravitational force $\bm F_g$
\item Spring force $\bm F_s$
\item Electrostatic force $\bm F_q$
\item Magnetic force $\bm F_m$
\item Nuclear forces
\end{itemize}
Because they shared these common properties:
\begin{itemize}[nosep]
\item The work done by these forces relate to a change of a potential energy
  \begin{itemize}[nosep]
  \item Positive work decreases this related potential energy
  \item Negative work increases this related potential energy
  \end{itemize}
\item The work done by a conservative force is \emph{path independent}, in that
  it depends only on end points, but not \emph{how} it gets from one end point
  to the other
\end{itemize}



By the fundamental theorem of calculus, any conservative forces $\bm F$
must be the negative gradient of the potential energies:
\begin{equation}
  \boxed{
    \bm F=-\nabla U=-\diffp Ux\iii-\diffp Uy\jjj-\diffp Uz\kkk
  }
\end{equation}
In one-dimension, the gradient operator becomes just:
\begin{equation}
  \boxed{
    F=-\diff Ux
  }
\end{equation}
The direction of a conservative force \emph{always} decreases the potential
energy. (Pay attention to the negative sign. Students often forget it.)



\section{Energy Diagrams}
A plot of potential energy ($U$) vs.\ position ($x$) for a conservative force
\begin{figure}[ht]
  \centering
  \begin{tikzpicture}[scale=.8]
    \draw[axes] (0,0)--(10,0) node[right]{$x$};
    \draw[axes] (0,0)--(0,5) node[right]{$U(x)$};
    \draw[very thick] (.2,4.5) to[out=-70,in=180](1.5,2.5)
    to[out=0,in=180] (2.5,1)
    --(5.5,1) node[midway,below]{\scriptsize Neutral equilibrium}
    to[out=0,in=180] (7,3.5)
    to[out=0,in=180] (8,2.5)
    to[out=0,in=250] (9.5,4.5);
    \fill(1.5,2.5) circle (.07) node[below]{$A$};
    \fill(7,3.5) circle (.07) node[below]{$B$};
    \fill(8,2.5) circle (.07) node[above]{$C$};
    \draw[<-] (1.6,2.7)--(2,3.3) node[right]{\scriptsize Unstable equilibrium};
    \draw[<-] (6.9,3.6)--(6.4,4.2) node[left]{\scriptsize Unstable equilibrium};
    \draw[<-] (8,2.4)--(8,1.5) node[below]{\scriptsize Stable equilibrium};
  \end{tikzpicture}
\end{figure}

%  The expressions for potential energies also come from integrating the work
%  equation, in that work equals to the change in \emph{something}, and we
%  called that potential energy. Therefore:
%
%  \eq{-.2in}{
%    \boxed{
%      W_c=-\Delta U
%    }
%  }
%  \begin{itemize}
%  \item\vspace{-.15in}$\Delta U$ can be positive or negative depending on the
%    direction of the (conservative) force
%  \item Positive work \emph{decreases} the related potential energy
%  \item Negative work \emph{increases} the related potential energy
%  \end{itemize}

%  Positive work done by conservative forces on an object does two things:
%  \begin{enumerate}[1.]
%  \item Decrease its potential energy, while
%  \item Increase its kinetic energy by the same amount
%  \end{enumerate}
%  Mathematically, this shows that mechanical energy must \emph{always} be
%  conserved when there are only conservative forces:
%
%  \eq{-.1in}{
%    W_c=-\Delta U = \Delta K \quad\longrightarrow\quad
%    \Delta K + \Delta U =0
%  }
%
%  That's why those forces are called conservative forces, and they form the
%  basis for conservation of energy.


\section{Non-Conservative Forces}
%The majority of forces are \textbf{non-conservative}.
The majority of the common forces encountered in introductory physics courses
are generally \textbf{non-conservative}. They include, but not limited to:
\begin{itemize}[nosep]
\item Applied force
\item Tension force
\item Normal force
\item Static friction
\item Kinetic friction
\item Aerodynamic lift and drag 
\end{itemize}
The work-energy theorem (Eq.~\ref{work-energy-theorem}) still applies for
non-conservative forces. However, the work done by non-conservative forces
differs from conservative forces in that:
\begin{itemize}
\item There is \textbf{no related potential energies}: the work done by a
  non-conservative force transform energy from one form of kinetic energy to
  another
\item The work is \textbf{path dependent}
\end{itemize}


\subsection{Work by Static Friction}
We can illustrate the work done by a non-conservative force in general, by
examining how work is done by static friction, using a multi-body problem
commonly studied in dynamics.

Two blocks with masses $m_1$ and $m_2$, stacked vertically, move to the right
without slipping by an external applied force $\bm F$ applied to $m_1$, as
shown in Fig.~\ref{stacked1}. The coefficient of static friction between the
two blocks is $\mu_s$, but the contact between $m_1$ and the table is
frictionless.\footnote{The focus of the dynamics study is often to calculate the
maximum acceleration $\bm a_\text{max}$ of the blocks without them sliding
against each other, and the maximum applied force $\bm F_\textbf{max}$
associated with that maximum acceleration.} The focus of \emph{this} example is
to find the work done by the forces between the blocks as the blocks accelerate.
\begin{figure}[ht]
  \centering
  \begin{tikzpicture}
    \fill[pattern=north east lines] rectangle (6,-.2);
    \draw[thick] (0,0)--(6,0);
    \draw[thick] (1,0) rectangle (4,1.2) node[midway]{$m_1$};
    \draw[thick] (1.7,1.2) rectangle (3.3,2) node[midway]{$m_2$};
    \draw[vector] (4,.6)--+(1.5,0) node[right] {$\bm F$};
    \draw[<-] (.95,.05) to[out=150,in=0] (0,.5) node[left]{frictionless};
    \draw[<-] (3.35,1.25) to[out=20, in=180] (4.5,1.8) node[right]{$\mu_s$};
  \end{tikzpicture}
  \caption{An external force is applied to accelerate two stacked blocks to
    the right.}
  \label{stacked1}
\end{figure}

The free-body diagrams of the blocks are shown in Fig.~\ref{stacked-fbd}.
(The forces highlighted in the same colour are action-reaction pairs.) The
static friction between $m_1$ and $m_2$ is $\bm f$.
\begin{figure}[ht]
  \centering
  \begin{tikzpicture}
    \fill circle (.1);
    \draw[vector] (0,0)--(0,-1) node[below]{$\bm F_{g2}$};
    \draw[vector] (0,0)--(0, 1) node[above,fill=pink!30]{$\bm N_{12}$};
    \draw[vector] (0,0)--(1.5,0) node[right,fill=yellow!30]{$\bm f$};

    \fill (5,0) circle (.1);
    \draw[vector] (4.97,0)--+(0,-1) node[below left]{$\bm F_{g1}$};
    \draw[vector] (5.03,0)--+(0,-1)
    node[below right,fill=pink!30]{$\bm N_{12}$};
    \draw[vector] (5,0)--+(0,1) node[above]{$\bm N_1$};
    \draw[vector] (5,0)--+(-1.5,0) node[left,fill=yellow!30]{$\bm f$};
    \draw[vector] (5,0)--+(2,0) node[right]{$\bm F$};
  \end{tikzpicture}
  \caption{Free-body diagrams of the stacked masses.}
  \label{stacked-fbd}
\end{figure}

The top block $m_2$ accelerates to the right because there is static
friction $\bm f$ at the interface with $m_1$. Static friction is the
\emph{only} force doing work on $m_2$, and the work done by $\bm f$ is
\emph{positive}. $m_2$ gains kinetic energy, consistent with the work-energy
theorem (Eq.~\ref{work-energy-theorem}). The change in kinetic energy in $m_2$
as it moves from $x_0\longrightarrow x_1$ is
\begin{equation*}
  \Delta K_2=\int_{x_0}^{x_1} f\dl x
\end{equation*}
On the bottom block $m_1$, as it moves to the right, applied force $\bm F$
does positive work, while static friction $\bm f$ does \emph{negative}
work. At a minimum, we conclude that energy is transferred from $m_1$ to $m_2$,
but we can do a more thorough (although still very simple) analysis to find out
how much.

Without friction between $m_1$ and $m_2$, the net force on $m_1$ would have
just been the applied force $\bm F$, and the change in kinetic energy after
moving from $x_0\longrightarrow x_1$ to the right would have been simply
\begin{equation*}
  \Delta K_1 = \int_{x_0}^{x_1} F\dl x\quad\quad\text{(no friction)}
\end{equation*}
But with static friction present, the gain in kinetic energy is reduced:
\begin{equation*}
  \Delta K_1 = \int_{x_0}^{x_1} F_\text{net}\dl x =
  \int_{x_0}^{x_1} (F-f)\dl x = \int_{x_0}^{x_1} F\dl x -
  \underbrace{\int_{x_0}^{x_1} f\dl x}_{\Delta K_2}
\end{equation*}
Work done by static friction reduced the kinetic energy of $m_1$ by the same
amount that is gained by $m_2$. We therefore conclude that work done by
friction transfers kinetic energy from one block to another.


\section{Internal/Thermal Energy}
Consider a container of gas of mass $M$ moving at speed $v$ at a height $h$
above Earth (shown in Fig.~\ref{fig:gas}). It has a bulk kinetic energy of
\begin{equation*}
  K=\dfrac12 Mv^2
\end{equation*}
and a gravitational potential energy\footnote{Using the ground level as the
reference} of
\begin{equation*}
  U_g=Mgh
\end{equation*}
But the random motion of the air molecules also contribute to additional
energy, called the \textbf{internal energy} $E_\text{int}$, or
\textbf{thermal energy}.
\begin{figure}[ht]
  \centering
  \begin{tikzpicture}
    \draw[thick,fill=gray!10] (-.2,-.2) rectangle (2.2,2.2);
    \draw[vector] (0,2.5)--(2,2.5) node[right]{$v$};
    \draw[vector] (-.5,-2)--(-.5,1) node[midway,left]{$h$};
    \foreach \i in {1,...,40} \fill[red] (rand+1,rand+1) circle (.06);
  \end{tikzpicture}
  \caption{A container of gas moving above Earth has kinetic, gravitational
    potential, as well as internal energies}
  \label{fig:gas}
\end{figure}
Internal energy of a system of molecules is the sum of all their kinetic and
potential energies at the microscopic level:
\begin{equation*}
  E_\text{int}=K_\text{micro} + U_\text{micro}
\end{equation*}
As the name suggests, $E_\text{int}$ is proportional to molecules'
\textbf{absolute temperature}, measured in \emph{kelvin}.
%\endnote{For those who are keen to know, for a
%monatomic or ideal gas, the internal energy comes entirely from the kinetic
%energy from the 3 degrees to translational freedom, and is given by
%\begin{equation*}
%  E_\text{int}=\dfrac32Nk_bT
%\end{equation*}
%where $N$ is the number of molecules, $k_b$ is the Boltzmann's constant. For a
%diatomic gas, there are 3 degrees of translational freedom, and 2 degrees of
%rotational freedom, and the internal energy is given by:
%\begin{equation*}
%  E_\text{int}\approx\dfrac52NkT
%\end{equation*}
%For solids, there are 3 degrees of translational freedom, and 3 degrees of
%degrees of vibrational freedom, and therefore the internal energy is:
%\begin{equation*}
%  E_\text{int}\approx3Nk_bT
%\end{equation*}
%}
Discussions on thermal/internal energy and the behaviour of gases and solids
are part of a much larger discipline within physics called
\textbf{thermodynamics}, but it is outside the scope of this chapter.



\section{Law of Conservation of Energy}
Conservation of energy is most often stated using a statement that even a lay
person with no physics background watching a movie may be familiar with:
\begin{center}
  \fcolorbox{black}{cyan!10}{
    \begin{minipage}{.98\linewidth}
      \centering
      \textbf{Energy cannot be created or destroyed; it only changes in form.}
    \end{minipage}
  }
\end{center}
While this statement certainly is not incorrect (indeed, it captures the
essence of how energy is conserved), the actual \textbf{law of conservation of
  energy} is more nuanced, and can be boring exercise in bookkeeping:
\begin{center}
  \fcolorbox{black}{cyan!10}{
    \begin{minipage}{.98\linewidth}
      \textbf{The change in the total energy of a system is equal
        to the net external work done to the system.}
    \end{minipage}
  }
\end{center}
Mathematically, the equation for the law of energy conservation is very
simple:
\begin{equation}
  \boxed{
    \Delta E_\text{sys}=W_\text{ext}
  }
  \label{eq:energy-conservation-law}
\end{equation}


\subsection{System}
A \emph{system} of object is defined in the same way as in solving dynamics
problems (specifically, the multi-body problems studied in
Section~\ref{sec:multibody}): a predefined collection of objects that apply
forces on each other, and therefore may do work on each other. A system may be:
\begin{itemize}[leftmargin=12pt,topsep=0pt]
\item\textbf{Isolated:} All the forces that act on the objects in the system
  act on each other (third law of motion) and are therefore \emph{internal}.
  There are no external forces.
\item\textbf{Closed:} There are external forces, but they do not do any work.
\item\textbf{Open:} There are external forces, and they do mechanical work
  (``external work'') to the objects in the system
\end{itemize}

The system's energy include both all the kinetic and potential energies of the
objects (collectively known as the \textbf{mechanical energy}), as well as the
internal/thermal energies of the objects:
\begin{equation}
  \boxed{
    E_\text{sys}
    =\underbrace{\sum K+\sum U}_\text{mechanical energy}+\sum E_\text{int}
  }
  \label{eq:system-energy}
\end{equation}
Substituting the expression from Eq.~(\ref{eq:system-energy}) into
(\ref{eq:energy-conservation-law}), we get the ``change'' in system energy:
\begin{equation}
  \boxed{
    \Delta E_\text{sys}=\sum\Delta K + \sum\Delta U + \sum\Delta E_\text{int}
    =W_\text{ext}
  }
\end{equation}
The sign convention of the external work $W_\text{ext}$ is the same as how it
is defined in Section~\ref{sec:mechwork}. When external work is: 
\begin{itemize}[leftmargin=12pt,nosep]
\item\textbf{Positive}: work is done \text{to} the system; the system
  energy \emph{increases}
\item\textbf{Negative}: work is done \text{by} the system to the surrounding;
  the system energy \emph{decreases}
\end{itemize}
In isolated systems that does not interact with the outside (and therefore
no external work can be done to the system), the law of conservation of energy
further simplifies to
\begin{equation}
  \boxed{
    \Delta E_\text{sys}=\sum\Delta K+\sum\Delta U+\sum\Delta E_\text{int}=0
  }
\end{equation}

%  In almost all of the problem encountered in AP Physics C, there will be no
%  change in the internal energy of the system, and conservation of energy
%  reduces to:
%  
%  \eq{-.1in}{
%    \boxed{ \Delta K + \Delta U = W_\text{ext} }\quad\rightarrow\quad
%    \boxed{ U_1 + K_1 + W_\text{ext} = U_2 + K_2 }
%  }
%
%  An \textbf{isolated system} is a system of objects that does not interact with
%  the surrounding. Think of an isolated system as a bunch of objects inside an
%  insulated box.
%  \begin{center}
%    \begin{tikzpicture}[scale=.7]
%      \fill[pattern=north east lines] rectangle (5,4);
%      \draw[thick] rectangle (5,4);
%      \draw[thick,fill=blue!5](.2,.2) rectangle (4.8,3.8);
%      \draw[thick,
%        decoration={aspect=.3,segment length=2mm, amplitude=2.5mm, coil},
%        decorate] (2.5,3.8)--(2.5,2.2) node[midway,right]{$\;\;k$};
%      \draw[thick,fill=cyan](2,2.25) rectangle (3,1.25) node[midway]{$m$};
%    \end{tikzpicture}
%  \end{center}
%  Since the system is isolated from the surrounding environment, the
%  environment can't do any work on it. Likewise, the energy inside the system
%  cannot escape either.


\subsection{Gravity: Free Fall}
Assuming that there are no friction and drag (i.e.: air resistance, see
Section~\ref{sec:drag}), a free-falling object (shown in Fig.~\ref{fig:earth})
forms an isolated system with Earth. 
\begin{figure}[ht]
  \centering
  \begin{tikzpicture}[scale=.65]
    \draw[thick,fill=cyan!5] (7.75,0) arc (75:105:30);
    \draw[mass] (0,4) circle (.2) node[right=3]{$m$};
    \draw[vector,red] (0,4)--+(0,-2) node[below=-2]{$\bm F_g$};
  \end{tikzpicture}
  \caption{Isolated system with a mass and Earth}
  \label{fig:earth}
\end{figure}
This isolated system consists of only Earth and the mass $m$, and therefore the
energy of the system is the kinetic energy of the mass ($K$) and the
gravitational potential energy ($U_g$) stored between the mass and
Earth\footnote{$U_g$ is often described as ``the gravitational potential energy
of the object''. Strictly speaking, this is incorrect; this is a topic that
will be studied in more detail in Chapter~\ref{chapter:gravity}. However, this
will not affect how the calculations is done, and in the opinion of the author,
poses no major conceptual issue. You are, of course, welcome to disagree}.

As the object falls, the gravitational force (which is internal to the system)
is doing positive work on both the mass as well as Earth\footnote{Due to the
mass of Earth, its displacement due to this gravitational force is laughably
insignificant. We will therefore only focus on the motion of the mass}.
%Beginning
%with the work-energy theorem, we recognize that as the object falls, the only
%force that does work is gravity, and it is \emph{internal} to the system.

Since gravity is a conservative force, the work done transforms gravitational
potential energy into kinetic energy by the same amount, and since $F_g$ is
internal to the system, the change in the system's energy is zero. %constant
%relate $W_g=-\Delta U_g$.
%\begin{align*}
%  W_\text{net} &=\Delta K\\
%  -\Delta U_g &=\Delta K
%\end{align*}
%Giving us:
\begin{equation}
  \boxed{
    \Delta K + \Delta U_g=0
  }
\end{equation}
In practice, we can use the following equation for problem solving:
\begin{equation}
  \boxed{
    \underbrace{K+U_g}_\text{initial state}
    =\underbrace{K'+U_g'}_\text{final state}
  }
\end{equation}



\subsection{Gravity: Arbitrary Ramp}
Assuming that there is no friction or drag, energy is also conserved for an
object sliding down an arbitrarily-shaped ramp:
\begin{figure}[ht]
  \centering
  \begin{tikzpicture}
    \draw[thick] (0,4) to[out=-30,in=180] (3,1) to[out=0,in=180] (5,3)
    to[out=0,in=170] (8,0) to[out=-10,in=180] (10,0);
    \draw[mass,rotate around={-60.5:(1,2.93)}] (1,2.93) rectangle +(.6,.6);
    \fill[red] (1.4,2.8) circle (.08);
    \draw[vector,red] (1.4,2.8)--+(0,-1.5) node[left]{$\bm F_g$};
    \draw[vector,red,rotate around={30:(1.4,2.8)}] (1.4,2.8)--+(1.4,0)
    node[right]{$\bm F_n$};
  \end{tikzpicture}
  \caption{A mass sliding along an arbitrary ramp is an isolated system.}
  \label{fig:ramp-system}
\end{figure}
Normal force $\bm F_n$ and gravity $\bm F_g$ act on the object, but
only gravity does work ($\bm F_n$ is always perpendicular to motion). The sum
of the kinetic energy of the mass ($K$) and the gravitational potential energy
($U_g$) stored between the mass and Earth is constant
\begin{equation}
  \boxed{
    K+U_g=\text{constant}
  }
\end{equation}
The shape of the ramp does not matter, only the initial and final
height relative to the referene.


\subsection{Horizontal Spring-Mass System}
Assuming that there are no friction, drag or other damping forces present, a
horizontal spring-mass system shown in Fig.~\ref{fig:hspring-mass} is a closed
system that consists of the mass and the spring. (Earth is not part of the
system.)
\begin{figure}[ht]
  \centering
  \begin{tikzpicture}
    \draw[mass] (5,.5) rectangle (6,1.5);
    \draw[thick,decorate,
      decoration={coil,amplitude=6,aspect=.5,segment length=6}] (0,1)--(5,1);
    \fill[pattern=north east lines] (6.5,.5)--(6.5,.3)--(-.2,.3)
    --(-.2,2)--(0,2)--(0,.5)--cycle;
    \draw[very thick] (0,2)--(0,.5)--(6.5,.5);
    \draw[vector,red] (5.5,1)--(5.5,0) node[below]{$\bm F_g$};
    \draw[vector,red] (5.5,1)--(5.5,2) node[above]{$\bm F_n$};
    \draw[vector,red] (5.5,1)--(4.5,1) node[above]{$\bm F_s$};
    \fill[red] (5.5,1) circle (.06);
  \end{tikzpicture}
  \caption{A horizontal spring-mass system without friction or drag is a
  closed system}
  \label{fig:hspring-mass}
\end{figure}
The sum of the kinetic energy of the mass ($K$) and the elastic potential
energy stored in the spring ($U_s$) is constant
\begin{equation}
  K+U_s=\text{constant}
\end{equation}



\subsection{Vertical Spring-Mass System}
Assuming that there are no friction, drag or other damping forces in the
spring, the vertical spring-mass system (consists of the mass, the spring
and Earth) is an isolatedd system.
\begin{figure}[ht]
  \centering
  \begin{tikzpicture}
    \draw[mass] (.5,1.5) rectangle (1.5,2.5);% node[midway]{$m$};
    \draw[thick,decorate,decoration={
        aspect=.5,segment length=7, amplitude=5,coil}] (1,5)--(1,2.5); 
    \fill[pattern=north east lines] (0,5) rectangle (2,5.2);
    \draw[very thick] (0,5)--(2,5);
    \draw[vector,red] (1,2)--(1,1) node[right]{$\bm F_g$};
    \draw[vector,red] (1,2)--(1,3) node[right]{$\bm F_s$};
    \fill[red] (1,2) circle (.05);
  \end{tikzpicture}
  \caption{A vertical spring-mass system without friction or drag is an
  isolated sytem.}
  \label{fig:vspring-mass}
\end{figure}
The sum of the kinetic energy of the mass ($K$), the gravitational potential
energy stored between the mass and Earth ($U_g$), and the elastic potential
energy stored in the spring ($U_s$) is constant.
\begin{equation}
  K + U_g + U_s=\text{constant}
\end{equation}



\subsection{Simple Pendulum}
\label{sec:simple-pendulum-energy}
Assuming that there are no friction, drag or other damping forces in the
spring, the simple pendulum system (consists of the mass and Earth) is a
closed system.
%    \begin{itemize}
%    \item Gravity ($m\bm g$), which is conservative, is the only force that
%      does work
%    \item Tension ($\bm F_T$), which is non-conservative, does not do work on the
%      pendulum because it is always perpendicular to the motion of the pendulum
%      bob
%    \end{itemize}
The sum of the kinetic energy of the mass ($K$), the gravitational
potential energy stored between the mass and Earth ($U_g$) is constant:
\begin{equation}
  K + U_g =K'+U_g'
\end{equation}
    
\begin{figure}[ht]
  \centering
  \begin{tikzpicture}
    \fill[pattern=north east lines] (-1,0) rectangle (1,.2);
    \draw[thick] (-1,0)--(1,0);
    \begin{scope}[rotate=15]
      \draw[thick] (0,0)--(0,-5);
      \draw[mass] (0,-5) circle (.2) node[right=4]{$m$};
      \draw[red,vector] (0,-5)--(0,-3.5) node[left]{$\bm F_T$};
      \draw[red,vector,rotate around={-15:(0,-5)}] (0,-5)--(0,-6.3)
      node[below]{$\bm F_g$};
    \end{scope}
    \draw[dashed,thin] (0,0)--(0,-5);
    \draw[axes] (0,-2) arc (270:285:2) node[midway,below]{$\phi$};
  \end{tikzpicture}
  \caption{A simple pendulum system is a closed system}
  \label{fig:pendulum-system}
\end{figure}



\subsection{Isolated System with Changing Internal Energy}
Energy is always conserved as long as your system is defined properly. In
this case, the system consists of a mass, a spring, Earth and all the air
molecules inside the box:
\begin{figure}[ht]
  \centering
  \begin{tikzpicture}[scale=.8,thick]
    \fill[pattern=north east lines] rectangle (5,4);
    \draw rectangle (5,4);
    \draw[fill=blue!5] (.2,.2) rectangle (4.8,3.8);
    \draw[thick,decorate,
      decoration={aspect=.3,segment length=2mm, amplitude=2.5mm, coil}]
    (2.5,3.8)--(2.5,2.25) node[midway,right=4]{$k$};
    \draw[mass] (2,2.25) rectangle (3,1.25) node[midway]{$m$};
  \end{tikzpicture}
  \caption{In this isolated system, mechanical energy decreases in time while
    internal/thermal energy increases.}
  \label{fig:isolated-with-thermal}
\end{figure}
The energies of this system include the kinetic energy of the mass ($K$), 
the gravitational potential energy ($U_g$) between the mass and Earth, the
elastic potential energy ($U_s$) stored in the spring, as well as the
internal/thermal energy ($E_\text{int}$) of the air molecules and the mass.

As the mass vibrates, friction and drag with air slows it down, converting the
kinetic energy of the mass into the internal energy of the air. Total energy
is conserved even as the mass stops moving.
%\begin{figure}[ht]
%  \centering
%  \begin{tikzpicture}[scale=.6]
%    \fill[pattern=north east lines] rectangle (5,4);
%    \draw[thick] rectangle (5,4);
%    \draw[thick,fill=blue!5] (.2,.2) rectangle (4.8,3.8);
%    \draw[thick,
%      decoration={aspect=.3,segment length=2mm, amplitude=2.5mm, coil},
%      decorate] (2.5,3.8)--(2.5,2.25) node[midway,right]{$\;\;k$};
%    \draw[mass] (2,2.25) rectangle (3,1.25) node[midway]{$m$};
%  \end{tikzpicture}
%\end{figure}

\begin{equation}
  K + E_\text{int}+U_g+U_s=\text{constant}
\end{equation}

%  \vspace{.2in}Non-conservative forces doing work are \emph{internal} to the
%  system, and therefore energy is still conserved. (Work done by friction
%  transform from the kinetic energy of the mass to the kinetic energy of the
%  air molecules.)


\subsection{Isolated System vs.\ Open System}
Accounting for the change in the internal energy of the air molecules is not
always practical, especially when the air molecules are not confined to a box.
\begin{figure}[ht]
  \centering
  \begin{tikzpicture}[thick]
    \fill[blue!5] (-3,0) rectangle (8,4);
    \draw (-3,4)--(8,4);
    \draw[
      decoration={aspect=.3,segment length=2mm, amplitude=2.5mm, coil},
      decorate] (2.5,4)--(2.5,2.25) node[midway,right=4]{$k$};
    \draw[mass] (2,2.25) rectangle (3,1.25) node[midway]{$m$};
  \end{tikzpicture}
  \caption{A system with friction is drag is usually treated as an open
    system.}
  \label{fig:open-system}
\end{figure}

The solution:
\begin{itemize}
\item Take the air molecule out of the \emph{system}
\item No longer an isolated system
\end{itemize}
The negative work done by kinetic friction and drag (collectively known as
$W_f$) are treated as external work between initial and final states
\begin{equation}
  \underbrace{K + U_g + U_e}_\text{initial state} + W_f=
  \underbrace{K' + U_g' + U_e'}_\text{final state}
\end{equation}

%  If \emph{only} conservative forces are doing work, mechanical energy (i.e.\
%  $K+U$) is always conserved:
%
%  \eq{-.2in}{
%    \boxed{K+U =K'+U'}
%  }
%  
%  When external non-conservative forces are also doing work, instead of
%  \emph{trying} to isolate the system, we can instead calculate the work done
%  by them $W_{nc}$ and add it to the total energy of the system
%    
%  \eq{-.2in}{
%    \boxed{K+U+W_{nc}=K'+U'}
%  }
%
%
%
%\textbf{Example:} A mass $m$ is dropped from a height of $h$ above the
%equilibrium position of a spring. Set up the equation that determines the
%spring's compression $d$ when the object is instantaneously at rest.
%\begin{center}
%  \pic{.35}{spring-example1}
%\end{center}
%
%
%  \textbf{Example 3:} A mass $m$ is pulled a distance $d$ up an incline (angle
%  of elevation $\theta$) at constant speed using a rope that is parallel to
%  the incline. The coefficient of friction is $\mu_k$.
%  \begin{enumerate}[(a)]
%  \item What is the magnitude of the tension force in the rope?
%  \item What is the magnitude of the normal force?
%  \item What is the work done by the normal force?
%  \item What is the work done by friction?
%  \item What is the work done by the tension force?
%  \item What is the net work?
%  \item What is the change in total mechanical energy?
%  \item Show that $\Delta E_{mech}=W_{nc}$.
%  \end{enumerate}


\section{Power \& Efficiency}
\textbf{Power} is the \emph{rate} at which work is done, i.e.\ the rate at
which energy is being transformed:
\begin{equation}
  \boxed{P(t) = \diff Wt}\quad\quad
  \boxed{\overline P = \frac W{\Delta t}}
\end{equation}
%\begin{center}
%  \begin{tabular}{l|c|c}
%    \rowcolor{pink}
%    \textbf{Quantity}  & \textbf{Symbol} & \textbf{SI Unit} \\ \hline
%    Instantaneous and average power & $P$, $\overline P$ & \si\watt \\
%    Work done          & $W$ & \si\joule \\
%    Time interval      & $\Delta t$ & \si\second
%  \end{tabular}
%\end{center}
In engineering, power is often more critical than the actual amount of work
done.

If a force is used to push an object at a constant velocity, the
power produced by the force is:
\begin{equation}
  P=\diff Wt=\frac{\bm F\cdot\dl\bm x}{\dl t}
  =\bm F\cdot\diff{\bm x}t
  \quad\rightarrow\quad
  \boxed{P=\bm F\cdot\bm v}
\end{equation}
Application: aerodynamics
\begin{itemize}
\item When an object moves through air, the applied force must overcome air
  resistance (drag force), which is proportional with $v^2$
\item Therefore ``aerodynamic power'' must scale with $v^3$ (i.e.\ doubling
  your speed requires $2^3=8$ times more power)
\item Important when aerodynamic forces dominate
\end{itemize}

\textbf{Efficiency} is the ratio of useful energy or work output to the total
energy or work input
\begin{equation}
  \boxed{ \eta = \frac{E_o}{E_i}\times\SI{100}\percent }\quad
  \boxed{ \eta = \frac{W_o}{W_i}\times\SI{100}\percent }
\end{equation}
\begin{center}
  \begin{tabular}{l|c|c}
    \rowcolor{pink}
    \textbf{Quantity} & \textbf{Symbol} & \textbf{SI Unit} \\ \hline
    Useful output energy & $E_o$  & \si\joule \\
    Input energy         & $E_i$  & \si\joule \\
    Useful output work   & $W_o$  & \si\joule \\
    Input work           & $W_i$  & \si\joule \\
    Efficiency           & $\eta$ & no units
  \end{tabular}
\end{center}
Efficiency is always $0\leq\eta<\SI{100}\percent$
%\theendnotes
