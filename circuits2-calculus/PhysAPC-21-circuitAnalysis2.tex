%\chapter{Circuit Analysis, Part 2}
%\label{chapter:circuits2}

%\section{Capacitors in Circuit}

\section{Capacitors in Parallel}

\begin{figure}[ht]
  \centering
  \begin{tikzpicture}[scale=1.2,thick]
    \draw (0,1) to[short,o-] (1,1) to[C=$C_1$] (1,0) to[short,-o] (0,0);
    \draw (1,1)--(3,1) to[C=$C_2$] (3,0)--(1,0);
    \draw (3,1)--(5,1) to[C=$C_3\ldots$] (5,0)--(3,0);
  \end{tikzpicture}
  \caption{Capacitors connected in parallel}
  \label{fig:parallel-C}
\end{figure}
From the voltage law, we know that the voltage across all the capacitors are
the same, i.e.\ $V_1=V_2=V_3=\cdots=V$. We can express the total charge
$Q_\text{tot}$ stored across all the capacitors in terms of capacitance and
this common voltage $V$: 
\begin{equation}
  Q_\text{tot}=Q_1+Q_2+Q_3+\cdots=C_1V+C_2V+C_3V+\cdots
\end{equation}  
Factoring out $V$ from each term gives us the equivalent capacitance:
\begin{equation}
  \boxed{C_p=\sum_i C_i}
\end{equation}




\section{Capacitors in Series}
Likewise, we can do a similar analysis to capacitors connected in series.
\begin{figure}[ht]
  \centering
  \begin{tikzpicture}[scale=1.2,thick]
    \draw (0,0) to[C=$C_1$,o-] (1.25,0) to[C=$C_2$] (2.5,0)
    to[C=$C_3$,-o] (3.75,0);
  \end{tikzpicture}
  \caption{Capacitors connected in series}
  \label{fig:series-C}
\end{figure}
The total voltage across these capacitors are the sum of the voltages across
each of them, i.e.\ $V_\text{tot}=V_1+V_2+V_3+\cdots$
  
The charge stored on all the capacitors must be the same! The
total voltage in terms of capacitance and charge is:
\begin{equation}
  V_\text{tot}=\frac Q{C_1}+\frac Q{C_2}+\frac Q{C_3}+\cdots
\end{equation}





%\section{Equivalent Capacitance in Series}
The inverse of the equivalent capacitance for $N$ capacitors connected in
series is the sum of the inverses of the individual capacitance.

\begin{equation}
  \boxed{ \frac1{C_s}=\sum_i\frac1{C_i} }
\end{equation}
Make sure we don't confuse ourselves with resistors.


%\section{How Do We Know That Charges Are The Same?}
It's simple to show that the charges across all the capacitors are the same
\begin{center}
  \begin{tikzpicture}[scale=1.5]
    \draw[thick] (0,0) to[C=$C_1$,o-] (1.25,0) to[C=$C_2$,-o] (2.5,0);
    \draw[dashed] (0.625,-.75) rectangle (1.875,1);
  \end{tikzpicture}
\end{center}
The capacitor plates and the wire connecting them are really one piece of
conductor. There is nowhere for the charges to leave the conductor, therefore
when charges are accumulating on $C_1$, $C_2$ must also have the same charge
because of conservation of charges.




\section{RC Circuits}

An \textbf{RC circuit} is one that has both resistors and capacitors. The
simplest form is a resistor and capacitor connected in series, and then
connect to a voltage source.
%\begin{center}
%  \begin{tikzpicture}[american voltages,thick]
%    \draw (0,0) to[battery,l=$V$] (0,2) to[R=$R$] (2,2)
%    to[C=$C$] (2,0)--(0,0);
%  \end{tikzpicture}
%\end{center}
%Because of the nature of capacitors, the current through the circuit will not
%be steady as were the case with only resistors.




\subsection{Discharging Capacitor}

The simplest form of an \emph{RC} circuit is one where the resistor ($R$) is
connected to a capacitor ($C$) that is initially charged to a voltage of
$V_c=Q_0/C$ where $Q_0$ was the initial charge across the capacitor
plates\footnote{As a reminder, the positive ``terminal'' carries a charge of
$+Q$, and the negative ``terminal'' has a charge of $-Q$}. The circuit is shown
in Fig.~\ref{fig:RC1}.
\begin{figure}[ht]
  \centering
  \begin{tikzpicture}[scale=1.5,thick]
    \draw (0,0) to[R=$R$] (0,2)--(2,2) to[C=$C$] (2,0)--(0,0);
  \end{tikzpicture}
  \caption{A resistor and capacitor connected in series}
  \label{fig:RC1}
\end{figure}
%The analysis starts with something simpler. There is no voltage source,
%and the capacitor is already charged to $V_c=Q_\text{tot}/C$. What happens
%when the current begin to flow?

At $t=0$, the switch is closed, completing the circuit. As current flows, the
charge on the capacitor decreases. Over time the current decreases, until the
capacitor is fully discharged, and current stops flowing. The energy that is
stored in the capacitor is dissipated by the resistor as heat. To analyze the
circuit, we apply Kirchhoff's voltage law: the voltage gain in the capacitor
$V_c$ is equal to the voltage drop by the resistor $V_R$:
\begin{equation*}
  V_c-\underbrace{IR}_{V_R}=0
\end{equation*}
The voltage across a capacitor is $V_c=Q(t)/C$. The current flow in
the circuit \emph{decreases} the total charge of the capacitor, therefore
$I=-\diff Qt$. Substituting these expressions in the above equation,
Kirchhoff's voltage law becomes a first-order ordinary differential equation
of $Q(t)$:
\begin{equation}
  \frac QC+R\diff Qt=0
  \label{eq:RC1-diff}
\end{equation}

\fcolorbox{black}{pink!10}{
  \small
  \begin{minipage}{.97\linewidth}
    \textbf{If you don't know how to solve Eq.~\ref{eq:RC1-diff}:}
    This is a standard first-order ordinary differential equation with constant
    coefficients. The standard solution is to first separate the variables
    and then integrating both sides:
    \begin{equation*}
      \frac QC=-R\diff Qt
      \quad\longrightarrow\quad
      %\frac{\dl Q}Q = \frac{-\dl t}{RC}
      %\quad\longrightarrow\quad
      %which we can now integrate and ``exponentiate'':
      %\begin{equation}
      \int\frac{\dl Q}Q = \int\frac{-\dl t}{RC}
      \quad\longrightarrow\quad
      \ln Q=\frac{-t}{RC} + K
      \quad\longrightarrow\quad
      Q=e^Ke^{-t/RC}
    \end{equation*}
    We know that the initial charge across the capacitor is $Q_0$. Setting
    $t=0$ and $Q=Q_0$ in the above equation, we find that the initial charge is
    also the constant of integration $e^K$:
    \begin{equation*}
      e^K=Q_0
    \end{equation*}
    This gives the time-dependent solution:
    \begin{equation*}
      Q(t)=Q_0e^{-t/RC}
    \end{equation*}
  \end{minipage}
}
    
The charge across the capacitor is an exponential decay function:
\begin{equation*}
  \boxed{
    Q(t)=Q_0e^{-t/\tau}
  }
\end{equation*}
%where $Q_0=Q_\text{tot}$ is the initial charge on the capacitor, and
where $\tau=RC$ is called the \textbf{time constant}, with an SI unit of
\emph{seconds}. It is the time it takes for the charge on the capacitor to
decrease to $1/e$ of the original value. Taking the time
derivative of $Q(t)$ gives us the current through the circuit:
\begin{equation}
  \boxed{
    I(t)=-\diff Qt=I_0e^{-t/\tau}
  }
\end{equation}
where the initial current at $t=0$ is given by $I_0=Q_\text{tot}/\tau=V_c/R$,
in agreement with Ohm's law.




\subsection{Charging a Capacitor}
The reverse of discharging a capacitor is to charge up the capacitor, by
connecting the capacitor $C$, a resistor $R$ and a constant voltage source
$\mathcal E$ in series, as shown in Fig.~\ref{fig:RC2}. The capacitor is
initiall uncharged (i.e.\ $Q(0)=0$ and therefore $V_c(0)=0$).

\begin{figure}[ht]
  \centering
  \begin{tikzpicture}[american voltages,scale=1.5,thick]
    \draw (0,0) to[battery,l=$\mathcal E$] (0,2) to[R=$R$] (2,2)
    to[C=$C$] (2,0)--(0,0);
  \end{tikzpicture}
  \caption{A capacitor, resistor and a voltage source connected in series}
  \label{fig:RC2}
\end{figure}

At $t=0$, the switch is closed, completing the circuit, and the capacitor
begins to charge. We can analyze the circuit by applying Kirchhoff's voltage
law:
\begin{equation*}
  %\mathcal E-R\diff Qt-\frac QC=0
  \mathcal E-\underbrace{IR}_{V_R}-V_c=0
\end{equation*}
where $V_R=IR$ is the voltage drop across the resistor, and
$V_c=Q/C$ is the voltage drop across the capacitor\footnote{Unlike in the
previous case, as we charge the capacitor, the voltage across the capacitor
will always be lower than $\mathcal E$, therefore it would be a voltage
\emph{drop} instead of a voltage \emph{gain}}. This time, the current
%This time we recognize again that $V_c=Q/C$ by definition, and that the current
flow in the circuit would \emph{increase} the charge across capacitor,
therefore $I=\diff Qt$ (not negative). Kirchhoff's voltage law now becomes
another first-order ordinary differential equation:
\begin{equation}
  \mathcal E-R\diff Qt-\frac QC=0
  \label{eq:RC2-diff}
\end{equation}

\fcolorbox{black}{pink!50}{
  \small
  \begin{minipage}{.97\linewidth}
    \textbf{If you don't know how to solve Eq.~\ref{eq:RC2-diff}:}
    Again, separating variables, and integrating, we get:
    \begin{equation}
      \int\frac{\dl Q}{C\mathcal E-Q}=\int\frac{\dl t}{RC}
      \quad\rightarrow\quad-\ln(C\mathcal E-Q)=\frac t{RC}+K
    \end{equation}
    ``Exponentiating'' both sides, we have
    \begin{equation}
      C\mathcal E-Q=e^Ke^{-t/RC}
    \end{equation}
    To find the constant of integration $K$, we note that at $t=0$, the charge
    across the capacitor is $0$, and we get
    \begin{equation}
      e^K=C\mathcal E=Q_\text{tot}
    \end{equation}
    which is the charge stored in the capacitor at the end. Substitute this back
    into the equation:
    \begin{equation}
      \boxed{
        Q(t)=Q_\text{tot}(1-e^{-t/RC})
      }
    \end{equation}
  \end{minipage}
}

Solving the differential equation, we find that the charge stored in the
capacitor is given by the exponential decay equation:
\begin{equation}
  \boxed{Q(t)=Q_\text{tot}(1-e^{-t/\tau})}
\end{equation}
where $Q_\text{tot}$ is the final total charge across the capacitor
plates\footnote{At the expense of sounding like a broken record, let's again
remember that one plate will have a charge of $+Q_\text{tot}$ while the other
plate will have a charge of $-Q_\text{tot}$.}. The time constant $\tau=RC$ for
charging a capacitor is the same as when the capacitor is discharging.
Differentiating with time, we find the current through the circuit; it is
identical to the equation for discharge:
\begin{equation}
  \boxed{I(t)=\diff Qt=I_0e^{-t/\tau}}
\end{equation}
where the initial current is given by $I_0=Q_\text{tot}/\tau=\mathcal E/R$.
This makes sense because $V_C(0)=0$, and all of the energy must be
dissipated through the resistor. Similarly, $I_\infty=0$.

The example for charging the capacitor is notable for two details that will
allow us to solve problems involving more complex configurations of resistors
and capacitors:
\begin{enumerate}[leftmargin=15pt]
\item When a capacitor is uncharged (i.e.\ when $Q=0$, there is no voltage
  across the plate ($V_c=0$), and the capacitor acts like a short circuit.
\item When a capacitor is fully charged, there is a voltage across it, but no
  current flows \emph{through} it ($I_c=0$). Effectively the capacitor acts
  like an open circuit.
\end{enumerate}

\subsection{More Difficult \emph{RC} Circuits}
Most \emph{RC} circuits have multiple capacitors and resistors instead of
the series resistor plus capacitor that we studied in the previous two
sections. In these cases, it would be very difficult to calculate the time
constant $\tau$. However, we note that the behaviour is always an exponential
decay function, and we are interested in:
\begin{itemize}
\item The initial flow of current (at $t=0$), ``when the switch is closed'',
  and
\item The steady-state solution (as $t\rightarrow\infty$), ``after the switch
  has been closed for a long time''.
\end{itemize}

\begin{center}
  \begin{tikzpicture}[scale=1.2,american voltages,thick]
    \draw (0,0) to[battery1,l=12<\volt>] (0,2) to[R=4<\ohm>] (2,2)
    to[short,-*] (2.46,2.3);
    \draw (2.5,2) to[short,*-] (3,2) to[short] (4,2)
    to[C=6<\micro\farad>] (4,0)--(0,0);
    \draw (3,0) to[R=8<\ohm>] (3,2);
  \end{tikzpicture}
\end{center}
\textbf{Example:} The capacitor in the circuit is initially uncharged. Find
the current through the battery
\begin{enumerate}
\item Immediately after the switch is closed
\item A long time after the switch is closed
\end{enumerate}




\section{LR Circuits}
Coils and solenoids in circuits are known as ``inductors'' and have large self
inductance $L$. When connected to a circuit, this self inductance prevents
currents rising and falling instantaneously. A basic \textbf{\emph{LR} circuit}
consists of a resistor ($R$) and an inductor ($L$) connected in series with a
constant voltage source ($\mathcal E$), as shown in Fig.~\ref{fig:LR}. The
inductor has a large self inductance, but the resistance is negligible.
\begin{figure}[ht]
  \centering
  \begin{tikzpicture}[american voltages,scale=1.2,thick]
    \draw (0,0) to[battery,l=$\mathcal E$] (0,2) to[short,-*] (0.5,2);
    \draw (0.51,2.3) to[short,*-*] (1,2)--(1.5,2) to [R=$R$] (3,2)
    to [L=$L$] (3,0)--(0,0);
  \end{tikzpicture}
  \caption{A resistor, inductor and voltage source connected in series}
  \label{fig:LR}
\end{figure}

At $t=0$, the swich is closed, completing the circuit, and current begins to
flow in the circuit. The rapid change in current causes the magnetic flux
through the inductor to rapidly increase, creating a back emf $V_L$ at the
inductor. The prevents the current from reaching steady-state value immediately.
As current slowly increases, the magnetic flux through the inductor and the
back emf decreases. Eventually, as $t\rightarrow\infty$, magnetic flux is
reduced to zero, and the circuit reaches steady state.

%\begin{tikzpicture}[american voltages,scale=1.2,thick]
%  \draw (0,0) to[battery,l=$\mathcal E$] (0,2) to[short,-*] (0.5,2);
%  \draw (.51,2.3) to[short,*-*] (1,2)--(1.5,2) to [R=$R$] (3,2)
%  to [L=$L$] (3,0)--(0,0);
%\end{tikzpicture}

To analyze the circiut, we apply Kirchhoff's voltage law: the voltage gain by
the battery must be the same as voltage drop across the resistor $V_R$ and
the inductor $V_L$:
\begin{equation}
  \mathcal E-V_R-V_L=0
\end{equation}
Applying Ohm's law ($V_R=IR$) and the equation for \emph{back emf} ($V_L=L\diff It$), the equation becomes a first-order ordinary differential equation for
the current $I(t)$ through the circuit. This equation is the same form as we
solved for the \emph{RC} circuits.
\begin{equation}
  \mathcal E-IR-L\diff It=0
\end{equation}
Following the same procedure as charging a capacitor, the time-dependent
current is found to be:
\begin{equation}
  I(t)=\frac{\mathcal E}R\left(1-e^{-t/\tau}\right)
  \label{eq:LR-current}
\end{equation}
Where $\tau=L/R$ defined as the time constant. We notice the following
behaviour in the circuit:
\begin{itemize}
\item At $t=0$, immediately after the switch is closed, $I(0)=0$. The back emf
  prevents any current through the circuit. The inductor effectively acts like
  an open circuit.
\item As $t\rightarrow\infty$, the back emf drops to zero, and we can show
  from Eq.~\ref{eq:LR-current} that $I(\infty)=\dfrac{\mathcal E}R$, which
  means that energy is only dissipated by the resistor, and the inductor
  effectively acts like a short circuit.
\end{itemize}
These two points will allow us to solve more complicated problems.



\section{\emph{LC} Circuit}
The final type of circuit is the \emph{LC} circuit. In its simplest form, the
circuit has an inductor ($L$) and capacitor ($C$) connected in series. The
capacitor is initially charged to a voltage of $V_c$, as shown in
Fig.~\ref{fig:LC}.
\begin{figure}[ht]
  \centering
  \begin{tikzpicture}[american voltages,scale=1.2,thick]
    \draw (0,0) to[L=$L$] (0,2)--(2,2) to[C=$C$] (2,0)--(0,0);
  \end{tikzpicture}
  \caption{A capacitor and an inductor are connected in series to form an
    \emph{LC} circuit.}
  \label{fig:LC}
\end{figure}
Like the other circuits analyzed in previous sections, the switch is closed at
$t=0$ to complete the circuit. Applying Kirchhoff's voltage law:  
\begin{equation*}
  V_C-V_L=0
  \quad\rightarrow\quad
  \frac QC - \left(-L\diff It\right)=0
  \quad\rightarrow\quad
  \frac QC + L\diff It=0
\end{equation*}
Since both terms are continuously differentiable, we can differentiate both
sides of the equation, which gives us a second-order ordinary differential
equation with constant coefficients.
\begin{equation*}
  L\diff[2]It+\frac1C\diff Qt=0
  \quad\rightarrow\quad
  \diff[2]It+\frac1{LC}I=0
\end{equation*}
This is the same ordinary differential equation that we studied in
Chapter 9, and the solution is the simple harmonic motion\footnote{We have
chosen the sine function here because the current at $t=0$ is $I(0)=0$.
Choosing the sine function means that we don't need a phase constant in
the solution}.
\begin{equation*}
  I(t)=I_0\sin(\omega t)\quad\text{where}\quad
  \omega=\frac1{\sqrt{LC}}
\end{equation*}
