\chapter{Momentum, Impulse and Collisions}
\label{chapter:momentum}

\textbf{Momentum} (or \textbf{translational momentum}, or \textbf{linear
  momentum}) is a quantity of motion defined as:
\begin{equation}
  \boxed{\bm p=m\bm v}
\end{equation}
\begin{center}
  \begin{tabular}{l|c|c}
    \rowcolor{pink}
    \textbf{Quantity}   & \textbf{Symbol} & \textbf{SI Unit} \\ \hline
    Momentum  & $\bm p$ & \si{\kilo\gram\metre\per\second} \\
    Mass      & $m$     & \si{\kilo\gram} \\
    Velocity  & $\bm v$ & \si{\metre\per\second}
  \end{tabular}
\end{center}
For rotational motion of a rigid body, there is also \textbf{angular momentum}
which will be studied in a later topic.

Taking the time derivative of the momentum vector from an inertial frame of
reference using the chain rule:
\begin{equation}
  \diff{\bm p}t=\diff{(m\bm v)}t
  =m\diff{\bm v}t+\diff mt\bm v
  =m\bm a+\dot m\bm v
\end{equation}
For constant mass $m$ (i.e.\ $\dot m=0$), this right-hand-side reduces to the
familiar form of the second law of motion, $m\bm a$.




%\begin{frame}{General Form of First \& Second Laws of Motion}
Accounting for changing mass, the general form of the first and second law:
\begin{itemize}
\item\textbf{The momentum of an object or a system of objects remains
  constant until a net external force acts on it.}
\item\textbf{The net external force acting on an object
  is the rate of change of its momentum.}
\end{itemize}
Summarizing this into a single equation:
\begin{equation}
  \boxed{
    \bm F_\text{net}(t)=\diff{\bm p}t
  }
  \label{eq:2nd-law-general-form}
\end{equation}

%  %Like the ``special case'' from the dynamics section,
This equation is only applicable from an inertial frame of reference.


\section{Impulse}

\textbf{Impulse} $\bm J$ is defined as the time integral of force $\bm F$:
\begin{equation}
  \bm J=\int_{t_1}^{t_2}\bm F(t)\dl t
\end{equation}
We can calculate the force generated by
\begin{itemize}
\item any of the forces acting on the object, or
\item the net force (called the \textbf{net impulse})
\end{itemize}
over the time interval beteen $t_1$ and $t_2$. Since $\bm F$, $\bm p$ and
$\bm J$ are all vectors, so the integral can be evaluated in each of the
$\iii$, $\jjj$ and $\kkk$ directions:
\begin{equation}
  J_x=\int_{t_1}^{t_2}F_x(t)\dl t\quad\quad
  J_y=\int_{t_1}^{t_2}F_y(t)\dl t\quad\quad
  J_z=\int_{t_1}^{t_2}F_z(t)\dl t
\end{equation}


\section{Impulse-Momentum Theorem}
Rearranging the variables in the general form of the second law of motion:
\begin{equation}
  \bm F_\text{net}(t)=\diff{\bm p}t\;\rightarrow\;
  \bm F_\text{net}(t)\dl t=\dl\bm p
\end{equation}
Integrating both sides, we get the \textbf{impulse-momentum theorem}:
\begin{equation}
  \int_{t_1}^{t_2}\bm F_\text{net}(t)\dl t
  =\int_{p_1}^{p_2}\dl\bm p
  \quad\rightarrow\quad
  \boxed{ \bm J_\text{net} =\Delta\bm p}
\end{equation}
The \textbf{net impulse} is equalled to the change in momentum of the object.


%\section{Average Force}
\textbf{Average force} $\bm F_\text{avg}$ is the time-averaged force vector
that gets the same impulse. It is used extensively in introductory physics
courses to avoid integration
\begin{equation}
  \bm J=\int_{t_1}^{t_2}\bm F(t)\dl t=\bm F_\text{avg} \Delta t
\end{equation}


%  \textbf{Example:} Jim pushes a box with mass \SI{1.0}{\kilo\gram} with a
%  \SI{5.0}{\newton} force for \SI{10}{\second} while the box stays on the same
%  place. Find the impulse of the pushing force, friction force, the
%  gravitational force, and the net force.


%  \textbf{Example:} A rocket generates a thrust force by ejecting hot gases
%  from an engine. If it takes \SI1{\milli\second} to combust
%  \SI{1.0}{\kilo\gram} of fuel, ejecting it at a speed of
%  \SI{1000}{\metre\per\second}, what thrust is generated?  
%  \vspace{.15in}\begin{enumerate}[A.]
%  \item \SI{1000}\newton
%  \item \SI{10000}\newton
%  \item \SI{100000}\newton
%  \item \SI{1000000}\newton
%  \end{enumerate}


%  \textbf{Example:} A rocket for mining the asteroid belt is designed like a
%  large scoop. It is approaching asteroids at a velocity of
%  \SI{e4}{\metre\per\second}. The asteroids are much smaller than the rocket.
%  If the rocket scoops asteroids at a rate of \SI{100}{\kilo\gram\per\second},
%  what thrust (force) must the rocket's engine provide in order for the rocket
%  to maintain constant velocity? Ignore any variation in the rocket's mass due
%  to the burning fuel.
%  \begin{enumerate}[A.]
%  \item\SI{e3}\newton
%  \item\SI{e6}\newton
%  \item\SI{e9}\newton
%  \item\SI{e12}\newton
%  \end{enumerate}


%  \textbf{Example:} Two balls of the same mass are dropped from the same
%  height onto the floor. The first ball bounces upwards from the floor
%  elastically. The second ball sticks to the floor. The first applies an
%  impulse to the floor of $I_1$ and the second applies an impulse $I_2$. The
%  two impulses obey:
%  \begin{enumerate}[(a)]
%  \item $I_2=2I_1$
%  \item $I_2=I_1/2$
%  \item $I_2=4I_1$
%  \item $I_2=I_1/4$
%  \end{enumerate}
%
%
%
%
\section{Collisions}
%
%\begin{frame}{Conservation of Momentum}
\begin{itemize}
\item From the third law of motion, we know that the action and reaction
  forces between two objects are always equal in magnitude and in opposite
  direction. Thus, their total impulse would be zero.
  
\item When there is no external force, the momentum of the total system will
  always be constant:
  \begin{equation}
    \boxed{
      \sum_i\bm p_i=\sum_i\bm p_i'
    }
  \end{equation}
\end{itemize}


\subsection{Classifications of Collisions}
\begin{itemize}
\item Elastic Collision:
  \begin{itemize}
  \item Total kinetic energy is conserved
  \item Momentum is conserved
  \end{itemize}
\item Inelastic collision:
  \begin{itemize}
  \item Kinetic energy is \textbf{not} conserved
  \item Momentum is conserved
  \end{itemize}
\item Completely inelastic collision:
  \begin{itemize}
  \item ``Perfectly inelastic collision''
  \item A special case of inelastic collision
  \item The objects move together after the collision
  \item Kinetic energy is \textbf{not} conserved
  \item  Momentum is conserved
  \end{itemize}
\end{itemize}




%\begin{frame}{How to Solve Conservation of Momentum Problem}
%  \begin{enumerate}
%  \item Check whether the condition for the conservation of momentum is
%    satisfied (i.e.\ are there any external forces?)
%  \item If so, write out expressions for initial momentum and final momentum,
%    and equate the two. You will get 1 to 3 equations (one for each direction).
%  \item Solve these equations, find the quantity you need to find.
%  \end{enumerate}
%  Remember that momentum is a vector. If there is no external force component
%  in some direction, then the momentum component in this direction is still
%  conserved.




\subsection{Before We Dive Into Some Exercises}
The most typical applications of momentum conservation are collision and
explosions
\begin{itemize}
\item\textbf{Collision: A hits B}
  \begin{itemize}
  \item Regardless of whether they move together or not afterwards, momentum
    is conserved
  \item Head-on collisions are usually 1D
  \item Glancing collisions are usually 2D or 3D
  \end{itemize}
\item\textbf{Explosion: A explodes and becomes B and C (and D and E\ldots)}
  \begin{itemize}
  \item A perfectly inelastic collision in reverse
  \item Total momentum of B and C (and D and E\ldots) is the same as A in the
    beginning
  \item Usually a 2D or 3D problem
  \end{itemize}
\end{itemize}

%
%
%
%  \textbf{Example:} Two blocks A and B, both have mass \SI{1.}{\kilo\gram}.
%  Block A has velocity \SI{3.}{\metre\per\second} and block B is at rest. Their
%  distance is \SI{1.}{\metre}. The surface is has dynamic friction coefficient
%  $\mu_k=0.02$. After they collide, they move together, what would be the final
%  velocity of these two blocks? How far can they go after the collision?


%  Max throws a ball into the air with an initial speed $\SI{10}{\metre\per\second} at an
%  angle of $60$ degree with the horizontal direction. By accident, the ball
%  splits into two parts (horizontally) in the air. Suppose both parts land at
%  the same time, neglecting the air resistance,
%  \begin{enumerate}
%  \item If one part is \SI{5}{\m} away from its original position (same
%    direction as the initial speed), where is the second part?
%  \item How about one parties \SI{5}{\metre} away from the original position in the
%    direction that has an angle of $30$ degree with its initial speed?
%  \end{enumerate}



%  \textbf{Example:} Two objects with equal mass are heading toward each
%  other with equal speeds, undergo a head-on collision. Which one of the
%  following statement is correct?
%  \begin{enumerate}[A.]
%  \item Their final velocities are zero
%  \item Their final velocities may be zero
%  \item Each must have a final velocity equal to the other's initial velocity
%  \item Their velocities must be reduced in magnitude
%  \end{enumerate}


%  \textbf{Example:} Two astronauts, each of mass \SI{75}{\kilo\gram}, are
%  floating next to each other in space, outside the space shuttle. One of them
%  pushes the other through a distance of \SI{1.}{\metre} (about an arm's
%  length) with a force of \SI{300}{\newton}. What is the final relative
%  velocity of the two?
%  \begin{enumerate}[A.]
%  \item \SI{2.}{\metre\per\second}
%  \item \SI{2.83}{\metre\per\second}
%  \item \SI{4.}{\metre\per\second}
%  \item \SI{16.}{\metre\per\second}
%  \end{enumerate}

%  \textbf{Example:} A billiard ball of mass \SI{.155}{\kilo\gram}
%  (``cue ball'') moves with a velocity of \SI{1.25}{\metre\per\second} toward
%  a stationary billiard ball (``eight ball'') of identical mass and strikes it
%  with a glancing blow. The cue ball moves off at an angle of \ang{29.7}
%  clockwise from its original direction, with a speed of
%  \SI{.956}{\metre\per\second}. What is the final velocity of the eight ball?


%  \textbf{Example:} A ball is dropped from a height $h$. It hits the ground
%  and bounces up with a momentum loss of \SI{10}{\percent} due to the impact.
%  The maximum height it will reach is:
%  \begin{enumerate}[(a)]
%  \item\num{.90}$h$
%  \item\num{.81}$h$
%  \item\num{.949}$h$
%  \item\num{.3}$h$
%  \end{enumerate}



%  \textbf{Example:} A simple pendulum has a bob of mass \SI{2}{\kilo\gram}
%  hanging on a cord of length \SI{1}{\metre}. Suppose the pendulum is raised
%  until it is horizontal (and angular displacement of \ang{90}) and then
%  released. What is the speed of the bob at the bottom of its swing?
%  \begin{enumerate}[(a)]
%  \item\SI{9.91}{\metre\per\second}
%  \item\SI{19.6}{\metre\per\second}
%  \item\SI{3.13}{\metre\per\second}
%  \item\SI{4.43}{\metre\per\second}
%  \end{enumerate}



%  \textbf{Example:} A toy firing a ball vertically consists of a vertical
%  spring which is compressed by \SI{.10}{\metre}. A force of \SI{10.}{\newton}
%  is needed to hold the spring at that compression. If a ball of mass
%  \SI{.050}{\kilo\gram} is placed on the compressed spring and the spring is
%  released, the ball will reach a height (above its initial position) of:
%  \begin{enumerate}[(a)]
%  \item \SI{1.}{\metre}
%  \item \SI{1.2}{\metre}
%  \item \SI{1.4}{\metre}
%  \item \SI{1.6}{\metre}
%  \end{enumerate}




\section{Elastic Collisions}

In an elastic collisions, \emph{both} momentum and kinetic energy is conserved.
In a 1D collision, both equations below have to be satisfied:
\begin{align*}
  \sum m_iv_i&=\sum m_iv_i'\\
  \sum\frac12 m_iv_i^2&=\sum\frac12 m_iv_i'^2
\end{align*}
\textbf{How kinetic energy is conserved:} In an elastic collision, energy is
first converted into a potential energy (e.g.\ elastic potential energy in a
spring), and then all the energy is released back as kinetic energy.


For collision of two objects, the conservation of momentum equation can be
expressed as:
\begin{equation}
  \boxed{m_1(v_1-v_1')=m_2(v_2'-v_2)}
\end{equation}
By moving $m_1$ terms to the left, and $m_2$ terms to the right. Likewise,
the conservation of energy can also be arranged as:
\begin{equation}
  \boxed{m_1(v_1^2-v_1'^2)=m_2(v_2'^2-v_2^2)}
\end{equation}
By multiplying every term by 2, and again, moving $m_1$ terms to the left,
and $v_2$ terms to the right.

Dividing the equations (2) by (1) from the last slide, we get:
\begin{equation}
  \frac{(2)}{(1)}\quad\rightarrow\quad
  \frac{m_1(v_1^2-v_1'^2)}{m_1(v_1-v_1')}=
  \frac{m_2(v_2'^2-v_2^2)}{m_2(v_2'-v_2)}
\end{equation}

$m_1$ and $m_2$ terms cancel out, while the terms in the numerator can be
expanded as the difference of two squares which is then simplified:
\begin{equation}
  \frac{(v_1+v_1')(v_1-v_1')}{(v_1-v_1')}=
  \frac{(v_2'+v_2)(v_2'-v_2)}{(v_2'-v_2)}
\end{equation}
Leading to the final expression, which is substituted back into (1)
\begin{equation}
  v_1 +v_1'= v_2+v_2'
\end{equation}

%    \begin{displaymath}
%      \boxed{v_A'=\frac{m_A-m_B}{m_A+m_B}v_A}\quad
%      \boxed{v_B'=\frac{2m_A}{m_A+m_B}v_A}
%    \end{displaymath}
%  }
%
%  These equations work for \emph{all} elastic impact where object B (in this
%  example, the truck) is stationary when impact occurs. Substituting values for
%  $m_A$, $m_B$ and $v_A$, we get:
%  \begin{displaymath}
%    v_A'=\frac{m_A-m_B}{m_A+m_B}v_A=\frac{(1000-3000)}{(1000+3000)}\times 20
%    = \boxed{\SI{-10}{\metre\per\second}}
%  \end{displaymath}
%  \begin{displaymath}
%    v_B'=\frac{2m_A}{m_A+m_B}v_A=\frac{(2\times 1000)}{(1000+3000)}\times 20
%    = \boxed{\SI{10}{\metre\per\second}}
%  \end{displaymath}





When two objects 1 and 2 of mass $m_1$ and $m_2$ and  collide elastically,
their final velocities will be determined by the initial velocities $v_1$ and
$v_2$:

\begin{align*}
  v_1'&=\frac{m_2-m_1}{m_2+m_1}v_1+\frac{2m_2}{m_2+m_1}v_2\\
  v_2'&=\frac{2m_1}{m_2+m_1}v_1 + \frac{m_2-m_1}{m_2+m_1}v_2
\end{align*}

These equations are \emph{not} provided in the AP exam equation sheet, which
means that we are more interested in the behavior qualitatively rather than
quantitatively.


If both objects have equal mass ($m_1=m_2=m$) and the second object is
initially at rest ($v_2=0$), then the equations simplifies to
  
\begin{align*}
  v_1'&=\frac{v_1(m-m)+2mv_2}{m+m}=0\\
  v_2'&=\frac{v_2(m-m)+2mv_1}{m+m}=v_1
\end{align*}
All the momentum and energy from $m_1$ is transferred to $m_2$. Object 1
stops all together, while object 2 continues with the initial momentum and
velocity of Object 1.

Another special case is when $m_1\gg m_2$ and $v_2=0$ (i.e.\ a large object
colliding with a small stationary object) then we can effectively ``ignore''
$m_2$:
  
\begin{align*}
  v_1'&=\frac{v_1(m_1-m_2)+2m_2v_2}{m_1+m_2}\approx
  \frac{m_1v_1}{m_1}=v_1\\
  v_2'&=\frac{v_2(m_2-m_1)+2m_1v_1}{m_1+m_2}\approx
  \frac{2m_1v_1}{m_1}=2v_1
\end{align*}
Object 1 continues to move like nothing happened, but object 2 is pushed to
move at \emph{twice} the initial speed of object 1.

In the reverse case, if $m_1\ll m_2$, and $v_2=0$ (a small object colliding
with a large stationary object), then we can ``ignore'' the $m_1$ term:
\begin{align*}
  v_1'&=\frac{v_1(m_1-m_2)+2m_2v_2}{m_1+m_2}\approx
  \frac{-m_2v_1}{m_2}=-v_1\\
  v_2'&=\frac{v_2(m_2-m_1)+2m_1v_1}{m_1+m_2}\approx 0
\end{align*}
Object 1 bounces off object 2, and travels in the opposite direction with the
same velocity magnitude, while object 2 does not move.


%  \textbf{Example:} Blocks A and B have the same mass; A hits B with a speed
%  of \SI{5.0}{\metre\per\second} while B is initially at rest. If the collision
%  is elastic, what would be the final speed of these two objects?

%  \textbf{Example:} Blocks A and B with the same mass; A has a velocity
%  \SI{3.0}{\metre\per\second} to the east while B has
%  \SI{2.0}{\metre\per\second} to the west. If the collision is elastic, after
%  the collision, what would the velocity of the two blocks be?


%  \textbf{Example:} Throw a ball to a really big wall, when the ball reaches
%  the wall, it has a velocity \SI{10}{\metre\per\second} toward the wall. If
%  the collision is elastic, what would the final velocity of the ball be?


%\begin{frame}{Elastic Collision Example}
%  \textbf{Example:} Throw a ball with a velocity \SI{4.0}{\metre\per\second}
%  toward a train with a velocity \SI{40}{\metre\per\second} toward the ball.
%  If the collision is elastic, what would the final velocity of the ball be?


%\begin{frame}{Inelastic Collision: Calculating Energy Loss}
%  \textbf{Example:} Two blocks A and B with mass \SI{2.0}{\kilo\gram}, block
%  A hits B with velocity \SI{4.0}{\metre\per\second} while B is at rest.
%  \begin{enumerate}[(a)]
%  \item Suppose the collision is completely inelastic, what would the final
%    velocity of A and B be?
%  \item What is the loss of energy?
%  \end{enumerate}

