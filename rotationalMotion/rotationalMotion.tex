\chapter{Rotational Motion of Rigid Bodies}
\label{chapter:rotMotion}


Consider the uniform circular motion of an object with a (constant) angular
velocity $\bm\omega$\footnote{Recall that if rotation is counterclockwise,
$\bm\omega$ is \emph{out of the page}; if rotation is clockwise, $\bm\omega$ is
\emph{into the page}.}, as shown in Fig~\ref{fig:uniform}.
\begin{figure}[ht]
  \centering
  \begin{tikzpicture}[scale=.8]
    \draw[axes] (-3,0)--(3,0) node[right]{$x$};
    \draw[axes] (0,-3)--(0,3) node[above]{$y$};
    \draw[dashed] circle(2.5);
    \begin{scope}[rotate=38]
      \draw[vector] (2.5,0)--+(-1.5,0) node[below]{$\bm F_c$};
      \draw[vector] (2.5,0)--+(0,1.5) node[above]{$\bm v$};
      \draw[mass] (2.5,0) circle (.1);
    \end{scope}
  \end{tikzpicture}
  \caption{Uniform circular motion of an object about a fixed centre.}
  \label{fig:uniform}
\end{figure}
The centripetal force $\bm F_c$ that generated his motion points towards
the centre of the circular path and is therefore always orthogonal to the
velocity vector $\bm v$, i.e.\ $\bm F_c\perp\bm v$. Since
$\bm F_c$ is orthogonal to motion, therefore does not do any mechanical
work. Rotational kinetic energy is constant, and angular velocity
$\bm\omega$ also remains constant. The object, under this centripetal force,
will remain in the same uniform circular motion motion forever, or until the
net force changes. In other words, the \emph{rotational state} of the object
does not change. We conclude that the rotational of an object is not determined
by merely what forces are acting it.

From our everyday experience, we know that when tightening or loosening a
nut/bolt by turning a wrench (see Fig.\ref{fig:wrench}), the applied force has
to be directed at some distance away from the nut/bolt, and that how easily we
can turn the nut/bolt depends on the distance, the applied force as well as the
orientation of the force.
\begin{figure}[ht]
  \centering
  \includegraphics[width=.4\textwidth]{rotationalMotion/wrench}
  \caption{A force must be applied at some distance from the pivot of the
    rotational motion to affect rotational motion.}
  \label{fig:wrench}
\end{figure}


\section{Torque}
\textbf{Torque}\footnote{Also known as the \textbf{moment of force}, or just
\textbf{moment}} ($\bm\tau$) is the tendency for a force to change the
rotational motion of a body. Torque is generated when a force $\bm F$ acts
at a point some position $\bm r$ from a \textbf{pivot}\footnote{Also know
as the \textbf{fulcrum}}.
%\item There is an angle $\phi$ between $\bm F$ and $\bm r$
%\item In the example below, a force $\bm F$ is applied $\bm r$ away from
%  the pivot at an angle $\phi$. This generates a torque around the pivot.
%\end{itemize}
%\begin{figure}[ht]
%  \centering
%  \begin{tikzpicture}[scale=1.2]
%    \draw[thick,fill=lightgray] (-.13,-.13) rectangle (5.13,.13);
%    \draw[vector,red] (0,0)--(5,0) node[midway,below]{$\bm r$};
%    \draw[vector,blue] (5,0)--(5.7,-1.5) node[right]{$\bm F$};
%    \draw[dashed] (5,0)--(6,0);
%    \draw[axes] (5.7,0) arc (0:-atan(1.5/.7):.7)
%    node[midway,right]{$\phi$};
%    \fill circle (.06) node[above=1]{\textbf{pivot}};
%  \end{tikzpicture}
%  \caption{Generating a torque}
%  \label{fig:torque1}
%\end{figure}
The amount of torque generated depends on the force $\bm F$, where the force is
applied relative to the pitvot (position vector $\bm r$), and the orientation
$\phi$ of the force relative to the position vector. Mathematically, it is
defined as the cross product of $\bm F$ and $\bm r$. The unit for torque is a
\emph{newton metre} (\si{\newton\metre}).
\begin{equation}
  \boxed{
    \bm\tau=\bm r\times\bm F
  }
\end{equation}
If the position and force vectors are confined to a plane, then we can treat
torque as a one-dimensional vector:
\begin{equation}
  \boxed{\tau=rF_\perp=rF\sin\phi}
  \label{torque2}
\end{equation}
where $\phi$ is the angle between the force and position vectors, and
$F_\perp=F\sin\phi$ is the component of force orthogonal to the position
vector. $\tau$ is positive if it is counterclockwise, and negative if it is
clockwise.

The simplest case is when force is orthogonal to the position vector
(Fig.~\ref{fig:force-orthogonal}). In this case, $\tau=rF$, and the position
vector is also know as the \textbf{moment arm}.
\begin{figure}[ht]
  \centering
  \begin{tikzpicture}
    \draw[thick,fill=lightgray] (-.12,-.12) rectangle (5.12,.12);
    \draw[vector,violet] (0,0)--(5,0) node[midway,below]{$\bm r$};
    \draw[vector,orange] (5,0)--(5,2) node[above]{$\bm F$};
    \fill circle (.08) node[above]{pivot};
  \end{tikzpicture}
  \caption{Force is applied orthogonal to the position vector}
  \label{fig:force-orthogonal}
\end{figure}

Generally, $\bm F$ is not orthogonal to $\bm r$ (Fig.~\ref{fig:angled}). In
this case, there are two ways to evaluate torque. In the first method, we
decompose $\bm F$ into to its \emph{orthogonal} and \emph{parallel} components
($\bm F_\perp$ and $\bm F_\parallel$). By definition, only $\bm F_\perp$ generates
a torque; $\bm F_\parallel$ does not. We can then calculate the magnitude of
torque using Eq.~\ref{torque2}.
\begin{figure}[ht]
  \centering
  \begin{tikzpicture}
    \draw[thick,fill=lightgray] (-.12,-.12) rectangle(5.12,.12);
    \begin{scope}[vector]
      \draw[red] (0,0)--(5,0) node[midway,below]{$\bm r$};
      \draw[blue] (5,0)--(6.2,1.6) node[above]{$\bm F$};
      \draw[blue,dashed] (5,0)--(5,1.6) node[left]{$\bm F_\perp$};
      \draw[blue,dashed] (5,0)--(6.2,0) node[right]{$\bm F_\parallel$};
    \end{scope}
    \begin{scope}[rotate=-atan(3/4)]
      \draw (4,-.5)--(4,5.5);% node[pos=0,right]{line of action of $\bm F$};
      \draw (0,-1)--(0,2);
      \begin{scope}[vector]
        \draw[red] (0,0)--(4,0) node[midway,below]{$\bm r_\perp$};
        \draw[blue,dash dot] (4,0)--(4,2) node[midway,right]{$\bm F$};
      \end{scope}
      \fill[blue] (4,0) circle (.04);
    \end{scope}
    \fill circle (.08) node[above]{pivot};
    \draw (5,0)--(6.3,0);
    \draw (5.6,0) arc (0:atan(4/3):.6) node[pos=2/3,right]{$\phi$};
    \draw (.6,0) arc (0:atan(4/3):.6) node[pos=2/3,right]{$\phi$};
  \end{tikzpicture}
  \caption{Force is applied at an angle $\phi$ to the position vector.}
  \label{fig:angled}
\end{figure}

Alternatively, we can also move $\bm F$ along its line of action until it 
it intersects the orthogonal component of the position vector $\bm r_\perp$, as
shown in Fig.~\ref{fig:angled}. In this case, the torque generated is given by
$\tau = r_\perp F=(r\sin\phi)F$, wihch simplies to Eq.~\ref{torque2}. The moment
arm is $\bm r_\perp$ in this case.

No torque is generated if the line of action of the force goes through the
pivot. This can be when the force acts through the pivot, or when
\begin{figure}[ht]
  \centering
  \begin{subfigure}{.7\textwidth}
    \centering
    \begin{tikzpicture}[scale=.7]
      \draw[thick,fill=lightgray] (-.2,-.2) rectangle(5.2,.2);
      \draw[vector,red] (0,0)--(5,0) node[midway,below]{$\bm r$};
      \draw[vector,blue] (5,0)--(7,0) node[above]{$\bm F$};
      \draw (-1,0)--(9,0) node[above]{line of action};
      \fill circle (.08) node[above]{pivot};
      \fill[blue] (5,0) circle (.04);
    \end{tikzpicture}
    \caption{No torque is generated if $\bm F$ is applied parallel to $\bm r$.}
  \end{subfigure}  
  \begin{subfigure}{.7\textwidth}
    \centering
    \begin{tikzpicture}[scale=.7]
      \draw[thick,fill=lightgray] (-.2,-.2) rectangle (5.2,.2);
      \begin{scope}[rotate=-atan(3/4)]
        \draw (0,-1)--(0,3) node[right]{line of action};
        \draw[vector,blue] (0,0)--(0,2) node[right]{$\bm F$};
      \end{scope}
      \fill circle (.08) node[above]{pivot};
    \end{tikzpicture}
    \caption{No torque is generated if $\bm F$ is applied at the pivot}
  \end{subfigure}
\end{figure}

When there are multiple forces generating torque about a point, the net torque
is the vector sum of all torques acting at that point:
\begin{equation}
  \boxed{
    \bm\tau_\text{net}=\sum_i\bm\tau_i
  }
\end{equation}
Torque can come from individual forces acting at a point, e.g.\ when the
gravitational force, acting at the centre of mass, rotates a beam, or when an
applied force rotates a door handle. We can express those simply as:
\begin{equation*}
  \bm\tau_i = \bm r_i\times\bm F_i
\end{equation*}
Often, however, the forces that generate torques are spread over an area,
e.g.\ force on the head of a screwdriver. For rotational motion, we only care
about the torque. In most cases, we are not concenred with exactly how the
force generates the torque, only that a torque is generated.

\fcolorbox{black}{yellow!5}{
  \small
  \begin{minipage}{.97\linewidth}
    \textbf{Example:} Find the net torque on point C.
    \begin{center}
      \begin{tikzpicture}[scale=1.75]
        \begin{scope}[thick]
          \fill[blue!40,draw=black] (-1.65,-.12) rectangle (1.65,.12);
          \draw[<->] (-1.5,-.27)--(1.5,-.27) node[midway,fill=yellow!5]{3.0 m};
          \draw[<->] (-1.5,.27)--(0,.27) node[midway,fill=yellow!5]{1.5 m};
          \draw[dashed] (-2.2,0)--(2.2,0);
          \draw[dashed] (0,0)--(0,.5);
        \end{scope}
        \begin{scope}[orange,vector]
          \draw[rotate=-45] (0,0)--(0,1.5) node[right]{\SI{30}\newton};
          \draw[rotate around={30:(-1.5,0)}] (-1.5,0)--(-2.5,0)
          node[left]{\SI{20}\newton};
          \draw[rotate around={-30:(1.5,0)}] (1.5,0)--(2.2,0)
          node[right]{\SI{10}\newton};
        \end{scope}
        \draw[axes] (0,.4) arc(90:45:.4) node[pos=.7,above]{\ang{45}};
        \draw[axes] (-1.9,0) arc(180:210:.4) node[midway,left]{\ang{30}};
        \draw[axes] (1.9,0) arc(0:-30:.4) node[midway,right]{\ang{30}};
        \fill (-1.5,0) circle(.1) node[white]{\textbf{A}};
        \fill circle(.1) node[white]{\textbf{B}};
        \fill (1.5,0) circle (.1) node[white]{\textbf{C}};
      \end{tikzpicture}
    \end{center}
  \end{minipage}
}



\section{Angular Momentum}
Consider a mass $m$ connected to a massless beam rotates with velocity
$\bm v$ at a position $\bm r$ from the centre, as shown in
Fig.~\ref{fig:mass-on-stick}.
\begin{figure}[ht]
  \centering
  \begin{tikzpicture}[scale=2]
    \begin{scope}[rotate=70]
      \draw[line width=3.5] (0,0)--(2,0);
      \fill (2,0) circle (.04) node[right]{$m$};
      \fill circle (.05) node[right]{pivot};
      \draw[vector,red] (0,0)--(2,0) node[midway,right]{$\bm r$};
      \draw[vector,blue] (2,0)--(2,.5) node[left]{$\bm v$};
    \end{scope}
  \end{tikzpicture}
  \caption{A mass rotating about a fixed centre has both linear momentum and
    angular monentum}
  \label{fig:mass-on-stick}
\end{figure}
At any moment in time, it has a linear momentum of $\bm p=m\bm v$, but it
also has an \textbf{angular momentum} $\bm L$, defined in a way analagous to
the relationship between force and torque:
\begin{equation}
  \boxed{
    \bm L=\bm r\times\bm p
  }
  \label{eq:ang-momentum-def}
\end{equation}
The unit for angular momentum is a \emph{kilogram meter squared per second}
(\si{\newton\metre\squared\per\second}). Expanding the term in
Eq.~\ref{eq:ang-momentum-def} with $\bm p=m\bm v$, and
$\bm v=\bm\omega\times\bm r$, we can express angular momentum only in
quantities related to rotations:
\begin{equation}
  \bm L
  =m(\bm r\times\bm v)
  =m(\bm r\times(\bm\omega\times\bm r)) = mr^2\bm\omega
  \label{eq:l-vector}
\end{equation}
The direction of the angular momentum is the same as the angular velocity
vector. When the motion is confined to one plane (e.g.\ the $xy$-plane),
angular momentum can be expressed as a one-dimensional vector, where:
\begin{equation}
  L=rmv=mr^2\omega
  \label{eq:l-scalar}
\end{equation}
$L$ is positive if motion is counterclockwise, and negative if it is clocwise.

The scalar quantity $mr^2$ defined in Eqs.~\ref{eq:l-vector} and
\ref{eq:l-scalar} is called the \textbf{moment of inertia} $I$ (or
\textbf{rotational inertia}), with a unit of \textbf{kilogram meter squared}
(\si{\kilo\gram\metre\squared}). Moment of inertia can be considered to be an
object's ``rotational mass''. Angular momentum now be expressed as:
\begin{equation}
  \boxed{
    \bm L=I\bm\omega
  }
  \label{eq:angular-momentum}
\end{equation}
For a \emph{single particle} of mass $m$ rotating at a fixed distance $r$ from
the pivot, moment of inertia is defined as:
\begin{equation}
  \boxed{I=mr^2}
\end{equation}
For a collection of $N$ discrete particles rotating with the same $\bm\omega$,
each of mass $m_i$ at distance $r_i$ from a common pivot, the total moment of
inertia is the sum of the inidividual moments of inertia:
\begin{equation}
  \boxed{I=\sum_{i=1}^N m_ir_i^2}
\end{equation}
As the number of masses approaches infinity ($N\rightarrow\infty$), i.e.\ a
continuous distribution of mass rotating about a pivot, the summation
becomes an integral:
\begin{equation}
  \boxed{I=\int r^2\dl m}
\end{equation}
We have included the moment of inertia of several three-dimensional objects in
Table~\ref{tabl:moment-3d}.
\begin{table}[ht]
  \centering
  \pic{.75}{rotationalMotion/mic}
  \caption{Moment of inertia of simple 3D geometric shapes. The mass densities
  are assumed to be uniform}
  \label{tabl:moment-3d}
\end{table}
Note that while mass $m$ is an intrinsic property that remains constant
regardless of how you measure it, moment of inertia $I$ depend on the
orientation of the axis of rotation, and how the mass is distributed. Both of
these can change with time.

%Comparing the exressions of linear momentum to that of angular momentum in
%Eq.~\ref{eq:angular-momentum}, we can see that they are very similar in form,
%in that they are the multiple of a ``mass'' term and a velocity term:
%\begin{align*}
%  \bm p &= m\bm v\\
%  \bm L &= I\bm\omega
%\end{align*}
%Just as $\bm p$ describes the \emph{translational} state of motion of an
%object, $\bm L$ describes its \emph{rotational} state of a rotating system.



\section{Laws of Motion for Rotational Motion}
In the first two sections of this chapter, we have defined the concept of
torque and linear momentum. We also know that torque is responsible for
changing the rotational state of an object. Now we can our knowledge of the
laws of motion for linear/translational motion to formulate the the laws of
motion for rorational motion.

\subsection{First Law}% of Motion}
An object is in \textbf{translational equilibrium} when the net force acting on
it is zero:
\begin{equation*}
  \bm F_\text{net}=\bm 0
\end{equation*}
This does \emph{not} mean that the object has no translational motion; it just
means that the object's translational state is not changing, i.e.\ momentum is
constant in time, or $\dot{\bm p}=0$. For an object with constant mass $m$,
this means that the acceleration at the centre of mass is zero, i.e.\
$\bm a_\text{cm}=\bm 0$, and therefore $\bm v=\text{constant}$.

Likewise, an object is in \textbf{rotational equilibrium} when the net
\emph{torque} acting on it is zero:
\begin{equation}
  \bm\tau_\text{net}=\bm0
\end{equation}
This does \emph{not} mean that the object has no rotational motion; it just
means that the object's rotational state is not changing, i.e.\ angular
momentum does not change with time, or $\dot{\bm L}=0$. For constant moment
of inertia $I$, this means that $\bm\alpha=\bm 0$, and
$\bm\omega=\text{constant}$.



\subsection{Second Law}
Applying the general form of the second law of motion
(Eq.~\ref{eq:2nd-law-general-form} in Chapter~\ref{chapter:momentum}) that
relates the net force to the change in linear momentum, we find that net torque
is the time rate of change of angular momentum:
\begin{equation*}
  \bm\tau_\text{net}=\bm r\times\bm F_\text{net}
  =\bm r\times\diff{\bm p}t
  =\diff{(\bm r\times\bm p)}t
  =\diff{\bm L}t
\end{equation*}
%  \begin{itemize}
%  \item If the net torque on a system is zero, then the rate of change
%    of angular momentum is zero, and we say that the angular momentum is
%    conserved. 
%  \item e.g.\ When an ice skater starts to spin and draws his arms inward.
%    Since angular momentum is conserved, a decrease in $r$ means an
%    increase in $\omega$.
%  \end{itemize}
%\end{frame}


For translational motion, the general form of the second law of motion relate
the net force to the rate of change of the object's momentum:
\begin{equation}
  \bm F_\text{net}(t)=\diff{\bm p}t
  \label{eq:2nd-law-translational}
\end{equation}
For objects with constant mass $m$, Eq.~\ref{eq:2nd-law-translational} reduces
to the more familiar form, with the acceleration evaluated at the object's
centre of mass:
\begin{equation*}
  \bm F(t)=m\bm a_\text{cm}(t)
\end{equation*}

Likewise, the second law of motion for rotational motion is similar,
but with torque $\bm\tau$ replacing force $\bm F$, and angular momentum
$\bm L$ replacing linear momentum $\bm p$:
\begin{equation}
  \boxed{
    \bm\tau_\text{net}(t)=\diff{\bm L}t
  }
  \label{eq:2nd-law-rotational}
\end{equation}
For objects with constant momentum of inertia $I$,
Eq.~\ref{eq:2nd-law-rotational} reduces to:
\begin{equation}
  \boxed{
    \bm\tau_\text{net}(t)=I\bm\alpha(t)
  }
\end{equation}

%\subsection{Third Law of Motion}

\subsection{But there is no rotational motion, is there?}
Even when there is no apparent rotational motion, it does not necessarily mean
that angular momentum is zero. In the example shown in
Figure~\ref{fig:straight-line-angular-momentum}, a mass $m$ travels along a
straight path at constant velocity (uniform motion), but the angular momentum
around point $P$ is not zero.
\begin{figure}[ht]
  \centering
  \begin{tikzpicture}
    \draw[dashed] (-5,0)--(5,0);
    \draw[vector,red] (-5,0)--(-3,0) node[below]{$\bm v$};
    \shade[ball color=red] (-5,0) circle (.2) node[above=3]{$m$};
    \draw[vector,blue] (0,-2)--(-5,0)
    node[midway,below,rotate=-atan{2/5}]{$\bm r_1$};
    
    \shade[ball color=red] circle (.2) node[above=3]{$m$};
    \draw[vector,red] (.2,0)--(2,0) node[right]{$\bm v$};
    \draw[vector,blue] (0,-2)--(0,0) node[midway,left]{$\bm r_2$};
    
    \shade[ball color=red] (4,0) circle (.2) node[above=3]{$m$};
    \draw[vector,red] (4.2,0)--(6,0) node[right]{$\bm v$};
    \draw[vector,blue] (0,-2)--(4,0)
    node[midway,below,rotate=atan{2/4}]{$\bm r_3$};
    
    \fill (0,-2) circle (.05) node[below]{$P$};
  \end{tikzpicture}
  \caption{Angular momentum of an object travelling in uniform motion}
  \label{fig:straight-line-angular-momentum}
\end{figure}
Since there is no force and no torque acting on the object, both translational
momentum ($\bm p=m\bm v$) and angular momentum ($\bm L=\bm r\times\bm p$) are
constant.


\begin{example}
    \textbf{Example:} A skater extends her arms (both arms!), holding a
    \SI{2.0}{\kilo\gram} mass in each hand. She is rotating about a vertical
    axis at a given rate. She brings her arms inward toward her body in such a
    way that the distance of each mass from the axis changes from
    \SI{1.0}{\metre} to \SI{.50}\metre. Her rate of rotation (neglecting her
    own mass) will?
\end{example}


\fcolorbox{black}{yellow!5}{
  \small
  \begin{minipage}{.96\linewidth}
    \textbf{Example:} A \SI{1.0}{\kilo\gram} mass swings in a vertical circle
    after having been released from a horizontal position with zero initial
    velocity. The mass is attached to a massless rigid rod of length
    \SI{1.5}\metre. What is the angular momentum of the mass, when it is in its
    lowest position?
  \end{minipage}
}


\section{Work \& Energy in Rotational Motion}
We now turn our attention to the work and energy of objects that are rotating.

\subsection{Rotational Work}
For translational motion, mechanical work in a one-dimensional coordinate
system is defined as:
\begin{equation*}
  W_t=\int_{x_0}^{x_1}F(x)\dl x
\end{equation*}
Using the above definition, we can also calculate mechanical work for
rotational motion: %, mechanical work is defined similarly as:
\begin{equation}
  W_r=\int_{x_0}^{x_1} F\dl x
  =\int_{\theta_0}^{\theta_1} F (r\dl\theta)
\end{equation}
Noting that $\tau=rF$, rotational work becomes:
\begin{important-equation}
  W_r=\int_{\theta_0}^{\theta_1}\tau(\theta)\dl\theta
\end{important-equation}

\subsection{Rotational Kinetic Energy}
\label{sec:rotKE}
We can apply the same process as in Sec.~\ref{sec:transKE} to find the
\textbf{rotational kinetic energy} $K_r$:
\begin{align}
  W_{r,\text{net}}&=\int_{\theta_0}^{\theta_1}\tau_\text{net}\dl\theta
  =\int I\alpha\dl\theta
  =I\int \diff{\omega}t\dl\theta
  =I\int \diff{\theta}t\dl\omega\nonumber\\
  & =I\int_{\omega_0}^{\omega_1} \omega\dl\omega
  =\frac12I\omega_1^2-\frac12I\omega_0^2=\Delta K_r
\end{align}
where $K_r$ is defined as:
\begin{important-equation}
  \text{Rotational kinetic energy:}\quad K_r=\frac12I\omega^2
\end{important-equation}
Not surprisingly, the work-energy principle still applies to rotational motion:
Net rotational work on an object is equal to the change in kinetic energy.
\begin{important-equation}
  \text{Work-energy principle:}\quad W_r=\Delta K_r
\end{important-equation}
Just as the net/total translational work on an object changes its translational
kinetic energy (Eq.~\ref{work-energy-theorem} in Chapter~\ref{chapter:energy}),
the net rotational work on an object also changes its rotational kinetic energy.
%To find the kinetic energy of a rotating system of particles (discrete number
%of particles, or continuous mass distribution), we sum the kinetic energies of
%the individual particles:
%\begin{equation}
%  K_r=\sum_i\frac12m_iv_i^2=\sum\frac12m_i(r_i\omega)^2
%  =\frac12\left(\sum_i m_ir_i^2\right)\omega^2
%\end{equation}
%It's no surprise that rotational kinetic energy is given by:

\subsection{Total Kinetic Energy of a Rotating System}
The total kinetic energy of a rotating system is the sum of its translational
and rotational kinetic energies at its center of mass:
\begin{equation}
  \boxed{K=K_t+K_r=\frac12mv_\text{cm}^2+\frac12I_\text{cm}\omega^2}
\end{equation}
In this case, $I_\text{cm}$ is calculated at the centre of mass. For simple
problems, we only need to compute rotational kinetic energy at the pivot:
\begin{equation}
  \boxed{K=\frac12I_\text{p}\omega^2}
\end{equation}
In this case, the $I_\text{p}$ is calculated at the pivot.
%  \textbf{IMPORTANT:} $I_\text{cm}\neq I_\text{p}$



\section{Problem Solving with Both Rotational and Translational Motion}

Generally, when solving for rotational problems like the ones described in the
previous
sections, it is imperative to carefully draw a free-body diagram to account for
all the forces and torques acting on an object, as we have done in the previous
examples. A few things to keep in mind:
\begin{itemize}[nosep]
\item The direction of friction force is not always obvious.
\item The magnitude of any static friction force cannot be assumed to be at
  maximum.
\item If the object is to change its rotational state, there must be a net
  torque causing it.
\end{itemize}
Once the free-body diagram is
complete, we can breaks down the \emph{forces} into $\iii$, $\jjj$ and $\kkk$
components. We have now a set of three equations from the second law of motion:
\begin{equation*}
  \sum F_x=ma_x\quad\quad \sum F_y=ma_y\quad\quad \sum F_z=ma_z
\end{equation*}
It is likely that only \emph{one} direction will have acceleration.\footnote{In
  fact, whenever possible, it is a good practice to orient the Cartesian
  coordinate system such that acceleration only occurs in one direction.} In
the problems that are presented in this handout, there are no forces in the
$\kkk$ direction. We have only needed to use the $\jjj$ direction in the third
(with slippage) problem to calculate the normal force, so that the kinetic
friction $f_s$ can be calculated.

Because the motion is rotational in nature, we will also have to sum the net
torque along those same three axes as well:
\begin{equation*}
  \sum\tau_x=I_x\alpha_x\quad\quad \sum\tau_y=I_y\alpha_y\quad\quad 
  \sum\tau_z=I_z\alpha_z
\end{equation*}
In simpler cases like the ones presented here, net torque will only be along
the $\kkk$ direction, and there were no torque by any of the forces along the
$\iii$ and $\jjj$ directions (although for more complicated problems, there can
be net torque in all three directions). Note that the moments of inertia are
not equal ($I_x\neq I_y\neq I_z$) if the rolling object is not a sphere.

Depending on whether an object rolls with or without slipping, there may be
no relationship between angular velocity $\omega$ and translational velocity
$\bm v$, or between angular acceleration $\alpha$ and translational
acceleration $\bm a$. But in any case, you will be left with a system of
equations with equal number of unknowns to be solved. Use whatever method that
you are comfortable with to solve for the answers.





\section{Examples}
\subsection{A Spinning Disk}
Consider a uniform disk of radius $R$ and mass $M$ that is free to rotate about
its centre of mass (Fig.~\ref{disk1}). To analyse its rotational
motion, we first find its moment of inertia about the pivot from
Fig.~\ref{moment-3d}. For a uniform disk, $I=\frac12MR^2$. Since the centre of
mass is also the pivot, we only have to consider its rotational motion.
\begin{figure}[ht]
  \centering
  \begin{tikzpicture}
    \draw[mass] circle (2);
    \draw[thick] circle (.13);
    \fill (0,0)--(.13,0) arc (0:90:.13) node[above=-2]{pivot}--(0,0);
    \fill (0,0)--(-.13,0) arc (180:270:.13)--(0,0);
%    \draw[axes,rotate=-30] (1,0) arc (0:60:1) node[above]{$\omega$, $\alpha$};
    \draw[axes,rotate=40] (0,0)--(-2,0) node[midway,right=2]{$R$};
%    \node at (0,-2.5) {Mass $M$};
%    \node at (0,-3.3) {
%    \only<2>{
%      \begin{scope}[violet]
%        \draw[thick,dash dot] (0,-.13)--(0,-2);
%        \foreach \y in {-.4,-.8,...,-2.1} \draw[vector] (0,\y)--(-.8*\y,\y);
%        \node[right] at (1.6,-2) {$\bm v=\omega R$};
%      \end{scope}
    %    }
  \end{tikzpicture}
  \caption{A disk spinning in the counter-clockwise direction}
  \label{disk1}
\end{figure}

%    \column{.7\textwidth}
%    \only<2>{Speed at any point on this disk is proportional to its distance
%      to the pivot:
%
%      \eq{-.1in}{
%        v=\omega r \quad\text{for}\quad 0\leq r\leq R
%      }
%    }
%    \only<3>{When it is spinning, it has an angular momentum of
%
%      \eq{-.1in}{
%        L = I\omega
%      }
%
%      and a rotational kinetic energy:
%      
%      \eq{-.1in}{
%        K_r = \frac12I\omega^2
%      }
%    }
%    
%    \item Any angular acceleration that it may have will be due to the net
%      torque at the pivot/CM:
%      \begin{displaymath}
%        \tau_\text{net}=\frac{\Delta L}{\Delta t}=I\alpha
%      \end{displaymath}
%    \end{itemize}



%\begin{frame}{A Rotating Rod}
\begin{figure}[ht]
  \centering
  \begin{tikzpicture}
    \draw[mass] (-.15,0) rectangle (.15,-5);
    \draw[thick] (0,-2.5) circle(.13);
    \fill (0,-2.5)--(.13,-2.5) node[right=-1]{CM} arc(0:90:.13)
    --(0,-2.5);
    \fill (0,-2.5)--(-.13,-2.5) arc(180:270:.13)--(0,-2.5);
    \fill circle(.07) node[above]{pivot};
    \draw[axes,rotate=-90] (.7,0) arc (0:180:.7)
    node[left]{$\omega$, $\alpha$};
    \draw[thick,|<->|] (-.5,0)--(-.5,-5) node[midway,fill=black!2]{$L$};
%      \node at (0,-2.5) {Mass $M$};
%      \node at (0,-3.3) {$I=\dfrac12MR^2$};
%      \only<2>{
%        \begin{scope}[violet]
%          \draw[thick,dash dot] (0,-.13)--(0,-2);
%          \foreach \y in {-.4,-.8,...,-2.1}\draw[vector] (0,\y)--(-.8*\y,\y);
%          \node[right] at (1.6,-2) {$\bm v=\omega R$};
%        \end{scope}
    %      }
  \end{tikzpicture}
  \caption{A rod rotating about a pivot that is not at the centre of mass}
  \label{rod1}
\end{figure}



\section{Pure Rolling Problems}

The case of a rolling sphere is a standard example of combining the dynamics of
translational and rotational motions of a rigid body. Here, we present two
examples of a non-slip (i.e.\ pure rolling) motion, and one example with
slippage.

A pure rolling problem is one that contains both translational and rotational
motion, as shown in Fig.~\ref{fig:trans-and-rot}.
\begin{figure}[ht]
  \centering
  \begin{tikzpicture}
    \foreach \x in {0,4.5,9}{
      \fill (\x,0) circle (1);
      \fill[white] (\x,0) circle (.8);
      \fill (\x,0) circle (.1);
    }
    \foreach \y in {-1,0,1}{
      \fill[red] (0,\y) circle (.05);
      \draw[vector,magenta] (0,\y)--(1.5,\y) node[right]{$v$};
    }
    \node[above] at (0,1.5) {Translation};
    
    \node at (2.5,0) {\Huge$+$};
    
    \draw[vector,cyan,rotate around={-40:(4.5,0)}] (4.5,1.1) arc (90:0:1.1)
    node[right]{$\omega$};
    \draw[vector,cyan] (4.5,1)--+(1.5,0) node[above]{$v=\omega R$};
    \draw[vector,cyan] (4.5,-1)--+(-1.5,0) node[below]{$v=\omega R$};
    \foreach \y in {-1,1} \fill[blue] (4.5,\y) circle (.05);
    \node[above] at (4.5,1.5) {Rotation};
    
    \node at (7,0) {\Huge$=$};
    
    \foreach \y in {0,1} \fill[violet] (9,\y) circle (.05);
    \draw[vector,violet] (9,1)--(12,1) node[above]{$v=2\omega R$};
    \draw[vector,violet] (9,0)--(10.5,0) node[right]{$v=\omega R$};
    \fill[violet] (9,-1) circle (.05) node[below]{$v=0$};
  \end{tikzpicture}
  \caption{The rolling problem is one that combines translational and
    rotational motion. At the contact point, if there is no slipping, the
    velocity is zero.}
  \label{fig:trans-and-rot}
\end{figure}
 If there is no slippage at
the contact point, the velocity of the object must be zero. We can see that
in this case, the angular speed $\omega$ and the translational speed  of the
centre of mass $v_\text{cm}$ must be related by
\begin{equation}
  \boxed{
    v_\text{cm}=\omega R
  }
  \label{eq:relate1}
\end{equation}
where $R$ is the radius of the object. Since $R$ is constant, the linear
acceleration of the CM must also be related to the angular acceleration by
taking the time derivative of Eq.~\ref{eq:relate1}:
\begin{equation}
  \diff{v_\text{cm}}t= \diff\omega{t}R\quad\rightarrow\quad
    \boxed{
      a_\text{cm}=\alpha R
    }
\end{equation}


\subsection{Pure Rolling on Flat Surface}
\label{no-slip-ball}
In the first and simplest case, shown in Fig.~\ref{roll-flat}, a uniform sphere
of radius $R$ rolls along a smooth, flat surface with a translational velocity
$\bm v$ and an angular velocity $\bm\omega$. Since there is no slipping,
$v=\omega R$. We assume that the sphere and the surface are both perfectly
rigid (i.e.\ they do not deform).
%We also assume that the sphere and the surface are both perfectly smooth.
% without defects even at the microscopic level. 
The forces are also shown in Fig.~\ref{roll-flat}. Notice that
\emph{there is no friction between the sphere and the surface}.
\begin{figure}[!ht]
  \centering
  \begin{tikzpicture}[scale=1.5]
    \shade[ball color=gray!20] circle (1);
    \draw[thick] (-2,-1)--(2,-1);
    \draw[vector] (1.5,0)--(2.5,0) node[right]{$\bm v$};
    \draw[vector] (0,1.2) arc (90:60:1.2) node[midway,above]{$\bm\omega$};
    \draw[axes] (2.5,-1.5)--+(.7,0) node[right]{$x$};
    \draw[axes] (2.5,-1.5)--+(0,.7) node[above]{$y$};
    \fill circle (.05) node[right]{cm};
    \draw[vector,blue] (0,0)--+(0,-1.5) node[below]{$\bm F_g$};
    \draw[vector,red] (-.03,-1)--+(0,1.5) node[above]{$\bm F_n$};
  \end{tikzpicture}
  \caption{Free-body diagram on a uniform density solid sphere rolling on a
    smooth flat surface without slipping.}
  \label{roll-flat}
\end{figure}

It is clear that there is zero net force acting on the sphere, therefore the
the translational state (i.e.\ momentum $\bm p=m\bm v$) stays constant in time.
Also, both gravity and normal forces go through the centre of mass, therefore
there is zero net torque acting on the sphere, and rotational state
(i.e.\ angular momentum $\bm L=I\omega$) is constant in time.In theory (if our assumptions are correct), the sphere will roll along forever.


%But of course, we are very observant.
But even a casual observer will notice that, in reality, a ball slows down and
eventually come to a stop. A steel ball bearing on a track can roll over a
much longer distance and time than a soccer ball, but neither staying rolling
forever. So what causes this? Specifically, what is missing in the free-body
diagram in Fig.~\ref{roll-flat}?

%It seems that our assumptions aren't quite correct.
%There are two major oversights in our initial assumptions.
%\begin{enumerate}[leftmargin=12pt]
%\item\textbf{Nothing is perfectly smooth.} Firstly, our original assumptions
%  mean that the contact area is infinitesimal small, and the normal force, by
%  basic geometry, points straight toward the CM. However, we should recognize
%  that neither the ball bearing nor the rail are perfectly smooth. When a
%  non-smooth ball rolls over a non-smooth surface, their surface roughness
%  means that the contact point is finite in size, and that the normal force
%  does not necessarily point toward the CM of the ball. This means that unlike
%  in Fig.~\ref{roll-flat}, there is a net force and net torque that will slow
%  down the motion of the ball.
We must recognize that \textbf{nothing is perfectly rigid.}\footnote{This
should be obvious, but in the pursuit of learning physics, this is a detail
that may be lost. When two objects collide in any collision, it takes a finite
amount of time for either objects to accelerate to the new velocities. If both
objects are perfectly rigid, then the collision will occur over an infinitely
small time interval, with infinitely large forces acting on them.} Both the
ball and the surface deform as they make contact. A perfect illustration is how
a tire flattens when it makes contact with the ground, shown in
Fig.~\ref{tire1}.
\begin{figure}[!ht]
  \centering
  \pic{.3}{rotationalMotion/OAGZy}
  \caption{Deformation of a tire under load as it rolls over a surface
    without slipping.}
  \label{tire1}
\end{figure}

The normal force is large in magnitude on
the front side is large in magnitude than on the other, and therefore exerts
both a resistive force to slow down the wheel, as well as negative torque to
slow down the tire. Also, the normal force does not point toward the CM. This
is called \textbf{rolling resistance}.
%\end{enumerate}  
%Now that we have understood the basic problem, we can tackle the next problem
%that involves friction.


\subsection{Pure Rolling on an Inclined Surface}
But what if the uniform sphere rolls down a ramp of angle $\theta$ instead? The
free-body diagram for that sphere is shown in Fig.~\ref{roll-ramp}. The radius
of the sphere is $R$ and its mass is $M$.
\begin{figure}[!ht]
  \centering
  \begin{tikzpicture}[scale=1.3]
    \begin{scope}[rotate=-30]
      \shade[ball color=gray!20] circle(1);
      \draw[thick] (-2,-1)--(5,-1);
      \draw[->,rotate=30](0,0)--(-1,0) node[left]{$R$};
      \draw[vector] (0,1.2) arc(90:60:1.2) node[pos=.3,right]{$\bm\omega$};
      \draw[axes] (2,0)--(3,0) node[right]{$x$};
      \draw[axes] (2,0)--(2,1) node[above]{$y$};
      \fill circle (.05) node[right]{cm};
      \draw[vector,blue,rotate=30] (0,0)--(0,-1.5) node[below]{$\bm F_g$};
      \draw[vector,red] (.0,-1)--(.0,.3) node[above]{$\bm F_n$};
      \draw[vector,orange] (.0,-1)--(-.5,-1) node[left]{$\bm f_s$};
      \begin{scope}[rotate around={30:(5,-1)}]
        \draw[thick] (5,-1)--(3,-1) node[pos=.4,above]{$\theta$};
      \end{scope}
    \end{scope}
  \end{tikzpicture}
  \caption{Free-body diagram on a smooth solid sphere of radius $R$ rolling
    down a smooth ramp without slipping.}
    %The ball travels distance $d$ to the
    %bottom of the ramp.}
  \label{roll-ramp}
\end{figure}

%\textbf{Be careful what forces are acting on it.}
Gravity $\bm F_g$ acts at the CM, while normal force $\bm F_n$ acts at the
point of contact. This time, there is also static friction $\bm f_s$ acting up
the ramp ($-\iii$-dir.) at the point of contact between the sphere and the
surface. (Without friction, the sphere will just \emph{slide} down the ramp
without rotation.) To solve this problem, we have three dynamic equations. For
translational motion, the acceleration of the CM is entirely along the
$+\jjj$-direction:
%along the three
%axes\footnote{The $z$-axis points \emph{out} of the page. Counter-clockwise
%rotational motion is positive, while clockwise rotational motion is negative}:
\begin{align}
  \sum F_x&=mg\sin\theta-f_s=ma_\text{cm}\\ \label{f_x}
  \sum F_y&=N-mg\cos\theta=0
\end{align}
and for rotational motion, the (clockwise) torque is provided by the static
friction force alone, since neither gravity nor normal force generate a torque
about the CM:
\begin{equation}
  \sum\tau=rf_s=I\alpha \label{tau}
\end{equation}
For a uniform sphere, $I=\frac25MR^2$. We also recognize that for pure rolling,
$a_\text{cm}=\alpha R$. 
%\textbf{Don't be so sure about what $\mu_s$ tells us.}
At this stage, the static friction force is \emph{not} known and is a quantity
that needs to be solved.\footnote{The coeefficient of static friction $\mu_s$
can only tell you the \emph{maximum} static friction, not the \emph{actual}
friction that exists. However, we can use it to double check to see if the
answer makes sense.}

Solving this problem (Eqs.~\ref{f_x} and \ref{tau}) means finding two unknowns:
$a_\text{cm}$ and $f_s$. We first solve for the magnitude of static friction:

%\textbf{Relating rotational and translational motions.} Inserting the
%expression for the moment of inertia of the solid sphere $I_z=\dfrac25 mR^2$
%
%we can use
%Eq.~\ref{tau} to express static friction in terms of linear acceleration $a$:
\begin{equation}
  f_s=\frac{I\alpha}R=
  \left[\frac25 mR^2\right]
  \left[\frac{a_\text{cm}}R\right]
  \left[\frac1R\right]=\frac25ma_\text{cm}
  \label{f_s}
\end{equation}
Substituting the expression in Eq.~\ref{f_s} into Eq.~\ref{f_x}, the force
equation in the $\iii$-direction becomes:
\begin{equation}
  mg\sin\theta-\frac25 ma_\text{cm}=ma_\text{cm}
\end{equation}
Cancelling the mass terms and solving for acceleration, we find a constant
acceleration of :
\begin{equation}
  a_\text{cm}=\frac57 g\sin\theta
  \label{pure-roll-accel}
\end{equation}

Compare the results in Eq.~\ref{pure-roll-accel} to that of an object
\emph{sliding} without friction down the same ramp, the acceleration for the
sliding block is $a=g\sin\theta$ which is higher than the pure rolling case.
%The simplest explanation is that some of the gravitational potential energy is
%converted to both translational and rotational kinetic energies, while for the
%sliding case, all of the potential energy is converted into translational
%kinetic energy.
%If this is indeed the case, it means that the sphere will actually slip while
%rolling down the ramp, and the friction at the contact point is in fact kinetic
%friction.

Since acceleration is constant, kinematic equation can be used to compute the
speed of the sphere when it reaches the bottom of the ramp, a distance $d$ away.
If the sphere starts from rest:
\begin{equation}
  v=\sqrt{2ad}=\sqrt{\frac{10}{7}gd\sin\theta}
\end{equation}
\textbf{Sanity Check:} we must make sure that the friction calculated in
Eq.~\ref{f_s} has not exceeded the maximum static friction, given by
\begin{equation}
  f_s\leq\mu_sF_N
  \label{maxf}
\end{equation}
Combining the expression for $f_s$ in Eq.~\ref{f_s}, the acceleration in
Eq.~\ref{pure-roll-accel}, and the normal force on an incline, Eq.~\ref{maxf}
becomes:
\begin{align}
  f_s&\leq\mu_sF_N\nonumber\\
  \frac27mg\sin\theta&\leq\mu_smg\cos\theta\nonumber\\
  \frac27\tan\theta&\leq\mu_s
\end{align}
If the ramp angle $\theta$ is too steep, then the friction will transition from
static to kinetic, which is a much more difficult problem.

\textbf{Energy Conservation Consideration:} A much simpler way to find $v$ at
the bottom of the ram is by using the conservation of energy. In this case,
kinetic energy is split between translational kinetic energy $K_t$ and
rotational kinetic energy $K_r$:
\begin{align*}
  \Delta U_g&=K_t+K_r\\
  mg{\color{red}\Delta h}&=\frac12 mv^2+
  \frac12
  {\color{blue}I}{\color{orange}\omega}^2\\
  mg{\color{red}d\sin\theta}&=\frac12 mv^2+\frac12
  \left({\color{blue}\frac25 mR^2}\right)
  \left({\color{orange}\frac vR}\right)^2
  =\frac12 mv^2+\frac15mv^2=\frac7{10}mv^2
\end{align*}
Cancelling mass terms on both sides, and solving for $v$, we arrive at
the same expression as using dynamics and kinematics equations, in a fraction
of the time and effort:
\begin{equation*}
  v=\sqrt{\frac{10}7gd\sin\theta}
\end{equation*}

\textbf{But why is energy conserved?} That the total system energy is conserved
even when there is friction should be a significant insight for the novice
physics student. Clearly, static friction is \emph{non-conservative}; surely it
would have done some non-conservative external work. However, a careful look at
the work done by the static friction reveals how energy is transformed:
\begin{itemize}[leftmargin=12pt]
\item Static friction does \emph{positive rotational work}. It is the only
  force that generates a torque, therefore the positive work done by $\bm f_s$
  increases the rotational kinetic energy, i.e.\ $W_r=\Delta K_r>0$ by the
  work-energy theorem. This is supported by the fact that as the ball rolls, the
  angular velocity increases.
\item At the same time, static friction also does \emph{negative translational
  work}, as the direction of the frictional force is in the opposite direction
  to the translational motion of the CM. Again, by the
  work-energy theorem, i.e.\ $W_t=\Delta K_t<0$. This is supported by the fact
  that the (translational) velocity for the rolling case is \emph{lower} than
  the sliding case with no friction.
\end{itemize}
The work done by static friction essentially converts some translational kinetic
energy into rotational kinetic energy\footnote{This should be obvious, but it
may not be: only conservative forces convert kinetic energy into the related
potential energy, so static friction cannot directly convert from gravitational
potential energy to rotational kinetic energy.}. In this case, the system
remains isolated from the surroundings, and therefore, although static friction
is non-conservative, the work done is not external to the system, and therefore
the total energy of the system is conserved.

\subsection{Rolling on Flat Surface with Slippage}
We return to the flat-surface problem, but this time, we allow slippage at the
point of contact between the sphere and the surface. In this case, because there
is relative motion between them, there is \emph{kinetic} friction $f_k=\mu_kF_N$
at the point of contact, as shown in Fig.~\ref{slip1}. At that point, the
sphere slides to the left relative to the surface, and therefore the force of
friction is toward the right.
\begin{figure}[!ht]
  \centering
  \begin{tikzpicture}[scale=1.3]
    \shade[ball color=gray!20] circle (1);
    \draw[thick](-2,-1)--(2,-1);
    \draw[vector] (0,1.2) arc (90:60:1.2) node[below right]{$\bm\omega$};
    \draw[axes] (2,0)--(3,0) node[right]{$x$};
    \draw[axes] (2,0)--(2,1) node[above]{$y$};
    \fill circle (.05) node[right]{\small cm};
    \draw[vector,blue] (0,0)--(0,-1.5) node[below]{$\bm F_g$};
    \draw[vector,red] (-.05,-1)--(-.05,.5) node[above]{$\bm F_N$};
    \draw[vector,orange] (0,-1)--(.75,-1) node[below]{$\bm f_k$};
  \end{tikzpicture}
  \caption{Force diagram on a smooth solid sphere rolling on a flat surface with
    slippage.}
  \label{slip1}
\end{figure}

There is now a net force in the $+x$ direction, and a positive net torque in
the $+z$ direction (i.e.\ net torque is counter-clockwise). The consequences
are that:
\begin{enumerate}[topsep=0pt]
\item The net force caused by kinetic friction $\bm f_k$ causes the sphere to
  accelerate toward the $+x$ direction. Since $\bm f_k$ is constant, the
  acceleration is also constant as well (for as long as the sphere slips). At
  first glance, this may seem counter intuitive, but, we know that a car with
  its tires spinning on ice will still have a small acceleration.
\item The net torque in the $+z$ (counter clockwise) direction
  causes the angular velocity $\bm\omega$ to decrease over time.
\end{enumerate}
It is important to note that, unlike the previous no-slip cases where we can 
relate angular acceleration $\alpha$ with linear acceleration $a$ by the radius
of the sphere, for the slippage case, there is \emph{no relationship between
  $\alpha$ and $a$.} The velocity of the sphere $\bm v$ toward the right can
be expressed with a simple kinetic equation:
\begin{equation}
  v_x=v_0+at
\end{equation}
while the angular velocity of the sphere is given by:
\begin{equation}
  \omega=\omega_0+\alpha t
\end{equation}
Note that $\omega_0$ is negative, since the rotation is clockwise. At some time
$t$ there will be a point in time where $v=\omega r$. When this happens, the
sphere stops slipping, and the problem returns to the no-slip case that was
discussed in Section~\ref{no-slip-ball}.
