\section{Rigid Body Dynamics of Multiple Bodies}
\label{sec:multibody}


\begin{center}
  \pic{.7}{dynamics-calculus/graphics/worldslongestroadtrainwithpowertrailer8}
\end{center}
\begin{itemize}
\item The objects are connected by a cable or a solid linkage with negligible
  mass
\item All objects (usually) have the same acceleration
\item Require multiple free-body diagrams
\end{itemize}


\section{Introduction}
To solve a multi-body problem, you can follow these procedures:
\begin{enumerate}
\item Draw a FBD on each of the objects
\item Sum all the forces on all the objects along the direction of motion
  \begin{itemize}
  \item Direction of motion is usually very obvious
    \item All internal forces should cancel and do not figure into the
      acceleration of the system
  \end{itemize}
\item Compute the acceleration of the entire system using second law of motion
  \begin{itemize}
  \item Remember that (usually) every object has the same acceleration!
  \end{itemize}
\item Go back to the FBD of each of the objects and compute the unknown
  forces
\end{enumerate}

\section{Motion Along Level Surfaces}

\subsection{Objects Connected by Massless Cables}
Three masses ($m_1$, $m_2$ and $m_3$) are connected by massless
cables\footnote{Obviously cables are not \emph{literally} massless, but here,
we assume that the masses of the cables are insignificant compared to the
masses, and therefore we can ignore them without making our answers
inaccurate}, and pulled to the right by an external force $\vec F$ across a
level surface, as shown in Figure~\ref{fig:tension}. The coefficient of kinetic
friction between the masses and the surface is $\mu$.
\begin{figure}[ht]
  \centering
  \begin{tikzpicture}
    \draw[thick] (0,0)--(10,0);
    \draw[mass] (1,0) rectangle (2.5,1) node[midway]{$m_3$};
    \draw[brown,line width=2] (2.5,.5)--(4,.5)
    node[midway,above]{$T_2$};
    \draw[mass] (4,0) rectangle (5.5,1) node[midway]{$m_2$};
    \draw[brown,line width=2] (5.5,.5)--(7,.5)
    node[midway,above]{$T_1$};
    \draw[mass] (7,0) rectangle (8.5,1) node[midway]{$m_1$};
    \draw[vector] (8.5,.5)--(9.3,.5) node[right]{$\vec F$};
  \end{tikzpicture}
  \caption{Three masses are connected by massless cables and accelerate
    together.}
  \label{fig:tension}
\end{figure}

For this example, we want to consider the following questions:
\begin{enumerate}[nosep,leftmargin=15pt]
\item What are the forces acting on each of the masses?
\item What is the acceleration of the system, assuming that the cables do not
  break?
\item What are the magnitudes of the tension forces ($T_1$ and $T_2$) in the
  two cables?
\end{enumerate}

\textbf{Step 1---Free-Body Diagrams:} Our first step to finding the solution to
the questions is to draw free-body diagrams for each of the masses, as shown in
Figure~\ref{fig:fbd1}.
\begin{figure}[ht]
  \centering
  \begin{tikzpicture}[lightgray]
    \draw (0,0)--(9.5,0);
    \draw (1,0) rectangle (2.5,1) node[midway]{$m_3$};
    \draw[line width=2] (2.5,.5)--(4,.5);
    \draw (4,0) rectangle (5.5,1) node[midway]{$m_2$};
    \draw[line width=2] (5.5,.5)--(7,.5);
    \draw (7,0) rectangle (8.5,1) node[midway]{$m_1$};
    \begin{scope}[vector]
      \begin{scope}[orange]
        \fill (1.75,.5) circle (.1);
        \draw (1.75,.5)--(1.75,-1) node[below]{$m_3\vec g$};
          \draw (1.75,.5)--(1.75,2) node[above]{$\vec N_3$};
          \draw (1.75,.5)--(3,.5) node[above]{$\vec T_2$};
          \draw (1.75,.5)--(1,.5) node[below]{$\vec f_3$};
      \end{scope}
      \begin{scope}[violet]
        \fill (4.75,.5) circle (.1);
        \draw (4.75,.5)--(4.75,-1) node[below]{$m_2\vec g$};
        \draw (4.75,.5)--(4.75,2) node[above]{$\vec N_2$};
        \draw (4.75,.5)--(6,.5) node[above]{$\vec T_1$};
        \draw (4.75,.54)--(3.5,.54) node[above]{$\vec T_2$};
        \draw (4.75,.46)--(4,.46) node[below]{$\vec f_2$};
      \end{scope}
      \begin{scope}[blue]
        \fill (7.75,.5) circle (.1);
        \draw (7.75,.5)--(7.75,-1) node[below]{$m_1\vec g$};
        \draw (7.75,.5)--(7.75,2) node[above]{$\vec N_1$};
        \draw (7.75,.5)--(9.75,.5) node[right]{$\vec F$};
        \draw (7.75,.54)--(6.5,.54) node[above]{$\vec T_1$};
        \draw (7.75,.46)--(7,.46) node[below]{$\vec f_1$};
      \end{scope}
    \end{scope}
  \end{tikzpicture}
  \caption{Free-body diagrams}
  \label{fig:fbd1}
\end{figure}

In doing so, we note that
\begin{itemize}[nosep,leftmargin=15pt]
\item External force $\vec F$ is only applied to $m_1$, therefore it
  should not appear on the other free-body diagrams
\item Kinetic friction of the masses are: $f_1=\mu N_1=\mu m_1g$,
  $f_2=\mu N_2=\mu m_2g$, and $f_3=\mu N_3=\mu m_3g$. (In a general problem, the
  coefficients of friction do not need to be the same for all the masses, but
  we are simplifying the problem here.
\item $T_1$ and $T_2$ are action-reaction pair of forces
\end{itemize}


\textbf{Step 2---Second law of motion:} Once the free-body diagrams are
completed, we can sum the forces along the direction of motion for each object,
i.e.\ applying the second law of motion individually on each object.
\begin{align}
  {\color{blue} F-\mu m_1g-T_1} &= {\color{blue}m_1a}\label{eq:m1}\\
  {\color{orange} T_1-\mu m_2g-T_2} &= {\color{orange}m_2a}\label{eq:m2}\\
  {\color{violet} T_2-\mu m_3g} &= {\color{violet}m_3a}\label{eq:m3}
\end{align}
Since the cables do not break, all three masses have the same acceleration $a$
towards the right. Therefore we can \emph{sum} the three equations above
(Eqs.~\ref{eq:m1}--\ref{eq:m3}) to get the expression:
\begin{align}
  \nonumber
  F-\mu(m_1+m_2+m_3)g &= (m_1+m_2+m_3)a\\
  F-\mu Mg &= Ma \label{eq:law2-system}
\end{align}
which is just the second law of motion applied to the whole system as a single
object. Here, we introduce the variable $M=m_1+m_2+m_3$ to represent the total
mass of the whole system.

We can now use Eq.~\ref{eq:law2-system} to calculate the acceleration of the
entire system:
\begin{equation}
  a=\frac FM-\mu g
  \label{eq:accel}
\end{equation}
Depending on the problem, you may be solving for the algebraic expression, or
an actual numerical value.

\textbf{Step 3---Unknown Forces:} Once the acceleration of the system is known,
we can find all the other unknown forces---in this case, tension forces $T_1$
and $T_2$.

The easiest way is to solve for $T_2$ is to look at the force balance of
$\color{orange}m_3$ (Eq.~\ref{eq:m3}). Solving for $T_2$ and substituting the
acceleration that we got from Eq.~\ref{eq:accel}, we get:
\begin{align*}
  T_2&=m_3 a+\mu m_3 g\\
  &=m_3({\color{red}a}+\mu g)\\
  &=m_3\left({\color{red}\frac FM-\mu g}+\mu g\right)\\
  &=\frac{m_3}M F\\
  T_2 &=\frac{m_3}{m_1+m_2+m_3} F
\end{align*}
Then, to solve for $T_1$, we can use the force balance of $\color{blue}m_1$ in
Eq.~\ref{eq:m1}. Solving for $T_1$, and substituting the acceleration from
Eq.~\ref{eq:accel}, we find the expression for $T_1$:
\begin{align*}
  T_1 &=F-m_1({\color{red}a}+\mu g)\\
  &=F-m_1\left({\color{red}\frac FM-\mu g}+\mu g\right)\\
  &=F-m_1\frac FM\\
  &=\frac{M-m_1}MF\\
  T_1&=\left(\frac{m_2+m_3}{m_1+m_2+m_3}\right)F
\end{align*}
Of course, we can also use the force balance of $\color{violet}m_2$ to find
$T_1$. Solving for $T_1$ in Eq.~\ref{eq:m2}:
\begin{align*}
  T_1 &=T_2 + \mu m_2g + m_2a\\
  &=T_2 + m_2(a + \mu g)\\
  &=\frac{m_3}MF + m_2\left(\frac FM -\mu g + \mu g\right)\\
  &=\frac{m_2+m_3}MF\\
  &=\left(\frac{m_2+m_3}{m_1+m_2+m_3}\right)F\\
\end{align*}
which is the same as before.

But there is yet another way to find $T_1$! This time we treat
$\color{violet}m_2$ and $\color{orange}m_3$ as a single object. In this case, we add Eqs.~\ref{eq:m2} and \ref{eq:m3} together, and then solving for $T_1$
and substituting acceleration from Eq.~\ref{eq:accel}:
\begin{align*}
  T_1& =(m_2+m_3)(a+\mu g)\\
  & =(m_2+m_3)(\frac FM-\mu g+\mu g)\\
  & =\frac{m_2+m_3}MF\\
  & =\frac{m_2+m_3}{m_1+m_2+m_3}F\\
\end{align*}
Which, not surprisingly, is the same answer as using the other two methods.
Ultimately, the multi-problem is an exercise that you have done many times
in your math classes: solving multiple equations with multiple unknowns. In
this case, there are 3 unknowns ($a$, $T_1$ and $T_2$) and 3 equations.

\newpage
\subsection{Another Connected Bodies Example}
Two masses ($m_1$ and $m_2$) are being pushed along a level surface by an
external applied force $\vec F$, as shown in Figure~\ref{fig:pushed}. The
coefficient of kinetic friction between the masses and the surface is $\mu$.
Like the previous example, we want to consider the following questions:
\begin{enumerate}[nosep,leftmargin=15pt]
\item What is the acceleration of the masses?
\item What is the normal force that $m_1$ exerts on $m_2$?
\end{enumerate}
This problem is very similar to the previous example.

\begin{figure}[ht]
  \centering
  \begin{tikzpicture}[scale=4/3]
    \draw[thick] (-1.5,0)--(3.5,0);
    \draw[mass] rectangle (1,1) node[midway]{$m_1$};
    \draw[mass] (1,0) rectangle (2.5,1.25) node[midway]{$m_2$};
    \draw[vector] (-1,.5)--(0,.5) node[pos=0,left]{$\vec F$};
  \end{tikzpicture}
  \caption{Two masses pushed together by an external force}
  \label{fig:pushed}
\end{figure}

\textbf{Step 1---Free-body diagrams:} Like the previous problem, we start with
drawing free-body diagrams for each of the masses, shown in
Figure~\ref{fig:fbd-pushed}.
\begin{figure}[ht]
  \centering
  \begin{subfigure}{.4\textwidth}
    \centering
    \begin{tikzpicture}[scale=4/3,orange,vector]
      \fill circle (.07);
      \draw (0,0)--+(1.4,0) node[right]{$\vec F$};
      \draw (0,0)--+(0,-1) node[right]{$m_1\vec g$};
      \draw (0,0)--+(0,1) node[left]{$\vec N_1$};
      \draw (0,.05)--+(-.8,0) node[above]{$\vec N_{12}$};
      \draw (0,0)--+(-.5,0) node[below]{$\vec f_1$};
    \end{tikzpicture}
    \subcaption{Mass $m_1$}
  \end{subfigure}
  \begin{subfigure}{.4\textwidth}
    \centering
    \begin{tikzpicture}[scale=4/3,vector,violet]
      \fill circle (.07);
      \draw (0,0)--+(0,1) node[right]{$\vec N_2$};
      \draw (0,0)--+(-.45,0) node[below]{$\vec f_2$};
      \draw (0,0)--+(0,-1) node[right]{$m_2\vec g$};
      \draw (0,0)--+(.8,0) node[right]{$\vec N_{12}$};
    \end{tikzpicture}
    \subcaption{Mass $m_2$}
  \end{subfigure}
  \caption{Free-body diagrams for the pushed-blocks example}
  \label{fig:fbd-pushed}
\end{figure}

There are a few details about the normal forces.
\begin{itemize}[nosep,leftmargin=15pt]
\item All normal forces ($N_1$, $N_2$ and $N_{12}$) appear in the free-body
  diagrams as if they only act on the centre of mass. In fact, the normal force
  is spread over the entire contact surface.
\item The normal force $\vec N_{12}$ is at the contact between the two masses.
  It is an action-reaction pair, similar to the tension force in the previous
  example. When $m_1$ pushes against $m_2$ (action), $m_2$ pushes back against
  $m_1$ (reaction).
\end{itemize}  


\textbf{Step 2--Second law of motion:} Using the free-body digrams, we then sum
forces along the direction of motion for each mass:
\begin{align}
  {\color{orange}F-N_{21}-\underbrace{\mu m_1g}_{f_1}} &={\color{orange}m_1a} \\
  {\color{violet}N_{12}-\underbrace{\mu m_2g}_{f_2}} &= {\color{violet}m_2a}
\end{align}
Again, we recognize that both masses must have the acceleration. Adding them
together, we have
\begin{align}
  \nonumber
  F - \mu(m_1+m_2)g  &= (m_1+m_2)a\\
  F - \mu Mg  &= Ma \label{eq:system}
\end{align}
where $M=m_1+m_2$ is the total mass of the system of objects.
Eq.~\ref{eq:system} is essentially the second law of motion applied to the
entire system as if it is a single object. Solving for acceleration, we have:
\begin{equation}
  a= \frac FM- \mu g
  \label{eq:accel2}
\end{equation}
Eq.~\ref{eq:accel2} is identical to Eq.~\ref{eq:accel} from our first example.
Again, like the last problem, acceleration can be expressed as an algebraic
expression or with numerical values depending on the problem.

\textbf{Step 3--Find remaining forces:} Once acceleration is known, we can find
the remaining unknown force: the normal force $N_{12}$ between the blocks.
The magnitude of the normal force $N_{12}$ can be calculated using the
force-balance equations of either mass. Using the equation for $M_1$ and solving
for $N_{12}$:
\begin{align*}
  N_{12} &= F- m_1({\color{red}a}+\mu g)\\
  &= {\color{blue}F} - m_1\left({\color{red}\frac FM-\mu g} +\mu g\right) \\
  &= {\color{blue}M\frac FM} - m_1\left(\frac FM\right) \\
  &= \frac{M-m_1}M F\\
  N_{12}&= \frac{m_2}{m_1+m_2} F
\end{align*}
Or we can apply the same procedure to the force balance for $m_2$:
\begin{align*}
  N_{12} &= m_2 ({\color{red}a} + \mu g) \\
  &= m_2 \left({\color{red}\frac FM-\mu g} + \mu g\right) \\
  &= m_2 \left(\frac FM\right) \\
  &= \frac{m_2}{m_1+m_2} F
\end{align*}
which is the same as before.

\newpage
\subsection{Stacked Objects}
Two blocks ($m_1$ and $m_2$) are stacked on top of each other above a
frictionless table, as shown in Figure~\ref{fig:stacked}. An external force
$\vec F$ is applied to $m_2$, causing both blocks to accelerate to the right
together without slipping. The coefficients of static and kinetic friction
between the masses are $\mu_s$ and $\mu_k$ respectively.
\begin{figure}[ht]
  \centering
  \begin{tikzpicture}
    \draw[thick] (-1,0)--(5.5,0);
    \draw[mass] rectangle (3,1) node[midway]{$m_2$};
    \draw[mass] (.75,1) rectangle (2.25,1.75) node[midway]{$m_1$};
    \draw[vector] (3,.5)--(4.5,.5) node[right]{$\vec F$};
  \end{tikzpicture}
  \caption{Two masses stacked on top of each other on a frictionless table.}
  \label{fig:stacked}
\end{figure}

We want to consider the following questions:
\begin{enumerate}[nosep,leftmargin=12pt]
\item What is the maximum acceleration of the masses without slipping?
\item What is the magnitude of the external force $F$ at maximum
  acceleration?
\item What is the acceleration of $m_1$ if $F$ exceeds this maximum value?
\end{enumerate}


\textbf{Step 1--Free-body diagrams:} As in the previous example, we draw
free-body diagrams of both masses.
\begin{figure}[ht]
  \centering
  \begin{subfigure}{.4\textwidth}
    \centering
    \begin{tikzpicture}[orange,vector]
      \fill (1.5,1.375) circle (.1);
      \draw (1.5,1.375)--+(1.5,0) node[right,fill=yellow!30]{$\vec f$};
      \draw (1.5,1.375)--+(0,-1.2) node[right]{$m_1\vec g$};
      \draw (1.5,1.375)--+(0,1.2)  node[left=3,fill=pink!20]{$\vec N_{12}$};
    \end{tikzpicture}
  \end{subfigure}
  \begin{subfigure}{.4\textwidth}
    \centering
    \begin{tikzpicture}[vector,violet]
      \fill (1.5,.5) circle (.1);
      \draw (1.5,.5)--+(2,0) node[right]{$\vec F$};
      \draw (1.5,.5)--+(-1.5,0) node[left,fill=yellow!30]{$\vec f$};
      \draw (1.55,.5)--+(0,-1.2) node[right]{$m_2\vec g$};
      \draw (1.45,.5)--+(0,-1.2) node[left=3,fill=pink!20]{$\vec N_{12}$};
      \draw (1.5,.5)--+(0,1.2) node[left]{$\vec N_2$};
    \end{tikzpicture}
  \end{subfigure}
\end{figure}
There are two action-reaction pairs of forces:
\begin{itemize}
\item Normal force $N_{12}$ at the interface between $m_1$ and $m_2$ acts upwards
  on $m_1$ and downwards on $m_2$
\item Static friction\footnote{That is, if the blocks don't slide against each
other. If they do, then the friction would be kinetic friction instead} $f$,
  also at the interface between $m_1$ and $m_2$, acts forward on $m_1$, and
  backwards on $m_2$.
\end{itemize}
$f$ is the force that actually accelerates $m_1$ forward! (Think about
what happens when there is no friction between the masses.)


\textbf{Step 2:} For problems like this, we focus on the top blocks first.
The force that pulls the block forward is the static friction force $f_s$.
The second law of motion applied to $m_1$ is:
%  \vspace{.1in}
%  \begin{columns}
%    \column{.25\textwidth}
%    \centering
%    \begin{tikzpicture}[scale=.95,vector,orange]
%      \fill circle (.08);
%      \draw (0,0)--(1.5,0) node[right]{$\vec f_s$};
%      \draw (0,0)--(0,-1.2) node[right]{$m_1\vec g$};
%      \draw (0,0)--(0,1.2) node[left]{$\vec N_{12}$};
%    \end{tikzpicture}
%
Summing the forces along the direction of motion:
\begin{equation}
  f = m_1a
\end{equation}
Maximum acceleration $a_\text{max}$ occurs when static friction is also at
maximum, i.e.\ $f=\max f_s=\mu_s N_{12}=\mu_s m_1g$:
\begin{align}
  \nonumber
  \mu_s m_1g &= m_1a_\text{max}\\
  a_\text{max} &=\mu_s g
\end{align}
We see that the maximum acceleration is, in fact, independent of mass.

\textbf{Step 3:} Once the $a_\text{max}$ is known, we can use it to calculate
the maximum applied force. We can do this by substituting the expression for
acceleation into the force balance for $m_2$:
\begin{align*}
  F_\text{net}=F_\text{max}-f &= m_2a_\text{max}\\
  F_\text{max} &= m_2a_\text{max}+f_s\\
  &= m_2(\mu_sg)+\mu_s m_1g\\
  F_\text{max} &=\mu_s(m_1+m_2)g
\end{align*}
Or treat the whole system as a single object with total mass of $m_1+m_2$:
\begin{align}
  \nonumber
  F_\text{max} &= (m_1+m_2)a_\text{max}\\
  &= \mu_s(m_1+m_2)g
\end{align}
Of course both ways gives you the same answer.

\section{Atwood Machine--Pulley Problems}

\subsection{Example Problem: Atwood Machine}
An \textbf{Atwood machine} is made of two objects connected by a rope that
runs over a pulley. The pulley allows the direction of force and direction
of motion to change between two objects.
%  \begin{columns}
%    \column{.35\textwidth}
%    \begin{center}
%      \pic1{graphics/pulley_prob_2}
%    \end{center}
%    \column{.65\textwidth}
%    \textbf{Example:} The object on the left has a mass of $M$ and the object
%    on the right has a mass of $m$.
%    \begin{itemize}
%    \item What is the acceleration of the masses?
%    \item What is the tension in the rope?
%    \end{itemize}
%  \end{columns}


%  More typically, an Atwood machine problem is one where two objects are
%  sliding on a surface. These surfaces may have (or may not) have friction. In
%  this example, two blocks are connected by a massless string over a
%  frictionless pulley as shown in the diagram.
\begin{figure}[ht]
  \centering
  \begin{tikzpicture}[scale=1.5,thick]
    \draw[ultra thick,brown] (-2,.4)--(.1,.4);
    \draw (0,0)--(-4,0) node[pos=.35,below]{$\mu$};
    \draw[fill=magenta!50] (-2,0) rectangle +(-1,.75) node[midway]{$m$};
    \begin{scope}[rotate=-30]
      \draw[ultra thick,brown] (1,.4)--(-.05,.4);
      \draw (0,0)--(3,0) node[midway,below left]{$\mu$};
      \draw[mass] (1,0) rectangle (2.5,1) node[midway,rotate=-30]{$M$};
    \end{scope}
    \begin{scope}[rotate=-15]
      \draw[fill=gray] (0,.3) circle (.15);
      \draw[fill=lightgray] (0,.3) circle (.1);
      \draw[ultra thick] (0,0)--(0,.3);
      \fill (0,.3) circle (.04);
      \end{scope}
    \draw[gray] (0,0)--(0,-1.5);
    \draw[->] (0,-.5) arc(270:330:.5) node[midway,below]{$\phi$};
  \end{tikzpicture}
\end{figure}
%  \begin{enumerate}[(a)]
%  \item Determine the acceleration of the blocks.
%  \item Calculate the tension in the string.
%  \end{enumerate}
%\end{document}
%
