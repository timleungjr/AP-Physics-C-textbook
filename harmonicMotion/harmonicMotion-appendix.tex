\chapter{Calculations for Harmonic Motion Problems}
\label{appendix:harmonic-motion}

In this appendix, we will show some of the calculus that we used to generate
the graphs in Chapter~\ref{chapter:harmonic-motion}.



\section{Under-Damped Oscillator}
For the under-damped case with small values of $b$ ($b^2<4mk$), the solution 
has both an exponential decay term and a sinusoidal (oscillatory) term:
\begin{equation}
  x(t)= a e^{-\lambda t}\cos(\omega' t+\theta_0)
  \label{eq:appendix-damped-position}
\end{equation}
where, from Eq.~\ref{damping1}:
\begin{equation*}
  \lambda=\frac b{2m}\quad\text{and}\quad
  \omega'=\sqrt{\omega^2-\lambda^2}
\end{equation*}
In Fig.~\ref{fig:damping-effects}, we plotted the solutions $x(t)$ with various
damping factors $b$. In every case, motion starts at amplitude
(i.e.\ $x(0)=A_0$) from rest (i.e.\ $v(0)=\dot x(0)=0$). In the undamped case
($b=0$), we would have just used a cosine function $a=A_0$ and $\theta_0=0$.
However, for the under-damped case, neither is true. Instead, we have to use
the initial conditions to find both $a$ and $\theta_0$. Since we know the
initial position, we can substitute $t=0$ into
Eq.~\ref{eq:appendix-damped-position} and get:
\begin{equation}
  x(0)= a e^0\cos(0+\theta_0)=A_0\quad\quad\longrightarrow\quad\quad
  a=\frac{A_0}{\cos\theta_0}
\end{equation}
Taking the derivative of Eq.~\ref{eq:appendix-damped-position} with respect to
time gives us the velocity profile $v(t)$ of the motion---and don't forget to
apply the product rule and chain rule properly:
\begin{equation}
  v(t)=\diff xt
  = -ae^{\lambda t}\left[
    \lambda\cos(\omega' t+\theta_0)
    +\omega'\sin(\omega' t+\theta_0)
    \right]
  \label{eq:appendix-damped-velocity}
\end{equation}
Setting the initial condition $v(0)=0$, we can find the phase constant
$\theta_0$:
\begin{align*}
  v(0) &=-ae^0\left[\lambda\cos(\theta_0)+\omega\sin(\theta_0)\right]=0\\
  0 &=\lambda\cos(\theta_0)+\omega'\sin(\theta_0) \\
  -\lambda\cos(\theta_0) &= \omega\sin(\theta_0) \\
  -\frac{\lambda}{\omega'} &=\tan(\theta_0)\\
  \theta_0 &=\tan^{-1}\left[-\frac{\lambda}{\omega'}\right]
\end{align*}
To satisfy the initial conditions, the phase constant is not zero. Our final
equation for the position, which is plotted in Fig.~\ref{fig:damping-effects},
is now:
\begin{equation}
  x(t)=\left[\frac{A_0}{\cos\theta_0}\right] e^{-\lambda t}\cos(\omega't+\theta_0)
\end{equation}

%where $A_0$ is the initial amplitude. Like the simple harmonic oscillator,
%$A_0$, $\theta_0$, and whether to use a sine or cosine function, are based on
%initial conditions. The motion of the mass is called a
%\textbf{damped harmonic motion}, and such a system is called a
%\textbf{damped harmonic oscillator}.
%
%Unlike the simple harmonic oscillator, the motion of the damped oscillator is
%\emph{quasi}-periodic because the oscillation pattern is not perfectly
%repeated. The natural frequency\footnote{Because the motion is quasi-periodic,
%this frequency should be properly called a ``quasi-frequency''.} $\omega'$ for
%the damped oscillator is shifted from the undamped case $\omega$ based on the
%damping factor $b$:
%\begin{important-equation}
%  \quad\text{where}\quad
%  \omega=\sqrt{\frac km}
%  \label
%\end{important-equation}
%Note that $\omega'<\omega$ (natural frequency decreases) because of the
%damping factor $b$.



\section{Critical Damping}
For the critical damping case, where $b=b_c=\sqrt{4mk}$, our solution is in the
form:
\begin{equation}
  x(t)=(c_1+c_2t)e^{-\lambda t}
\end{equation}
Again, with $\lambda$ defined the same way. Again, we want to find the solution
$x(t)$ with motion starting at amplitude ($x(0)=A_0$) from rest (i.e.\
$v(0)=\dot x(0)=0$). Substituting $x(0)=0$ into the above euqation, we can
easily find that 
\begin{equation}
  x(0)=(c_1+\cancel{c_2t})\cancel{e^{-\lambda t}}=A_0
  \quad\to\quad
  c_1=A_0
\end{equation}
Again, we take the derivative of $x$ with respect to time, and using product
rule and chain rule, we find the velocity profile ($v(t)$) to be:
\begin{equation}
  v(t)=\diff xt=-(c_2\lambda t +c_1\lambda -c_2)e^{-\lambda t}
\end{equation}
putting in initial conditions
\begin{equation}
  v(0)=-(\cancel{c_2\lambda t} +c_1\lambda -c_2)\cancel{e^{-\lambda t}}=0
\end{equation}
which leads us to the equation:
\begin{equation}
  c_1\lambda -c_2=0\quad\to\quad c_2=c_1\lambda=A_0\lambda
\end{equation}

The final solution is
\begin{equation}
  x(t)=A_0(1+\lambda t)e^{-\lambda t}
\end{equation}

%\begin{important-equation}
%\end{important-equation}
%At critical damping, the solution to Eq.~\ref{eq:ode2} for the position of the
%mass $x(t)$ is
%where $c_1$ and $c_2$ are determined by initial conditions. If a mass is
%released at amplitude $A_0$ from rest (i.e.\ $x(0)=A_0$, $\dot x(0)=0$), then
%Eq.~\ref{eq:critically-damped-solution} becomes:
%\begin{equation}
%  x(t)=\left[A_0+\frac{\lambda}A_0\right]e^{-\lambda t}
%  \label{eq:critically-damped-solution-2}
%\end{equation}
%\begin{remark}
%  The solution in Eq.~\ref{eq:critically-damped-solution} are really two
%  linearly independent solutions:
%  \begin{align*}
%    x_1(t) &= c_1e^{\frac{b_ct}{2m}} \\
%    x_2(t) &= c_2te^{\frac{b_ct}{2m}}
%  \end{align*}
%  We should not be surprised that both solutions are also orthogonal functions.
%\end{remark}
%A critically damped system returns to its equilibrium position in the shortest
%time with \emph{no} oscillation.
%\textbf{EXPAND ON THIS SECTION: Critical or near-critical damping is desired
%in many engineering designs (e.g.\ shock absorbers on car suspensions).}

\section{Over-Damped}

For the over-damped system, where $b>b_c$
%, the system becomes \textbf{over-damped}. In such a case, the
%solution Eq.~\ref{eq:ode2} is the linear combination of two exponential decay
%functions:
\begin{equation}
  x(t)=c_1e^{r_1t}+c_2e^{r_2t}
  \quad\text{where}\quad
  r_1=\frac{-b+\sqrt{b^2-4mk}}{2m}\quad
  r_2=\frac{-b-\sqrt{b^2-4mk}}{2m}
\end{equation}
With initial condition $x(0)=A_0$, we have
\begin{equation}
  c_1+c_2=A_0
\end{equation}
Taking the derivative with respect to time, we have
\begin{equation}
  v(t)=\diff xt=r_1c_1e^{r_1t}+r_2c_2e^{r_2t}
\end{equation}
With initial condition $v(0)=0$, we have a second equation:
\begin{equation}
  v(0)=r_1c_1+r_2c_2=0
\end{equation}
To find the full solution, we solve the set of linear equations:
\begin{align*}
  c_1 + c_2 &= A_0\\
  c_1r_1 + c_2r_2 &=0
\end{align*}
