\chapter{Circular Motion}
\label{chapter:circ-motion}

The \textbf{circular motion} (Fig.~\ref{fig:circ-motion1}) is the simplest form
of curvilinear motions, an object of mass $m$ moves in a circular path about a
fixed center.

Like we did in Chapters~\ref{chapter:kinematics} and \ref{chapter:dynamics},
we will begin studying the circular motion, first be defining the coordinate
system, and then to the kinematic quantities, and then onto dynamics.

\section{Polar Coordinate System}
In the majority of types of two-dimensional motion discussed in earlier
chapters, it is usually best to describe an object's position using the
Cartesian coordinate system, i.e.\ using the $x$ and $y$ coordinates as
functions of time:
\begin{equation}
  \bm r(t)=x(t)\iii + y(t)\jjj
\end{equation}
For circular motion or general \textbf{curvilinear motions}, the
\textbf{polar coordinate system} is preferred. The position of an object is
described by:
\begin{equation}
  \bm r(t)=r(t)\hat{\bm r} + \theta(t)\hat{\bm\theta}
\end{equation}
where $r(t)$ is distance from the origin, and $\theta(t)$ is the standard angle,
measured counterclockwise from the $x$ axis. Some examples of curvilinear
motion are shown in Fig.~\ref{fig:curvilinear-motions}.
\begin{figure}[ht]
  \centering
  \begin{subfigure}{.24\linewidth}
    \centering
    \begin{tikzpicture}[scale=.9]
      \draw[axes] (-2,0)--(2,0) node[right]{$x$};
      \draw[axes] (0,-2)--(0,2) node[above]{$y$};
      \draw[function] circle (1.5);
    \end{tikzpicture}
    \caption{Circular motion}
    \label{fig:circ-motion1}
  \end{subfigure}
  \begin{subfigure}{.24\linewidth}
    \centering
    \begin{tikzpicture}[scale=.9]
      \draw[axes] (-2,0)--(2,0) node[right]{$x$};
      \draw[axes] (0,-2)--(0,2) node[above]{$y$};
      \draw[function,rotate=30] ellipse (1.8 and 1);
    \end{tikzpicture}
    \caption{Elliptical motion}
  \end{subfigure}  
  \begin{subfigure}{.24\linewidth}
    \centering
    \begin{tikzpicture}[scale=.9]
      \draw[axes] (-2,0)--(2,0) node[right]{$x$};
      \draw[axes] (0,-2)--(0,2) node[above]{$y$};
      \draw[function,domain={-1.7:1.7}] plot(\x,{.5*(\x*\x)-.3});
    \end{tikzpicture}
    \caption{Parabolic motion}
  \end{subfigure}
  \begin{subfigure}{.24\linewidth}
    \centering
    \begin{tikzpicture}[scale=.9]
      \draw[axes] (-2,0)--(2,0) node[right]{$x$};
      \draw[axes] (0,-2)--(0,2) node[above]{$y$};
%      \draw[domain=0:25,variable=\t,samples=200,function]
%      plot({\t r:1.75*exp(-.1*\t)});
    \end{tikzpicture}
    \caption{Inward/Outward spiral}
  \end{subfigure}
  \caption{Examples of curvilinear motion in two dimensions}
  \label{fig:curvilinear-motions}
\end{figure}

Like the Cartesian system, the polar coordinate system is also right-handed;
basics vectors $\hat{\bm r}$ (``radial direction'') and $\hat{\bm\theta}$
(``angular direction'') point in the directions shown in
Fig.~\ref{fig:basis-vecs},
\begin{figure}[ht]
  \centering
  \begin{tikzpicture}[scale=.75]
    \draw[axes] (-3,0)--(3,0) node[right]{$x$};
    \draw[axes] (0,-3)--(0,3) node[above]{$y$};
    \draw[vector] (0,0)--(1,0) node[below]{$\iii$};
    \draw[vector] (0,0)--(0,1) node[left] {$\jjj$};
    \draw circle (2.5);
    \begin{scope}[rotate=38]
      \draw[vector] (0,0)--(2.45,0) node[midway,above]{$r$};
      \draw[vector] (2.5,0)--(3.5,0) node[right]{$\hat r$};
      \draw[vector] (2.5,0)--(2.5,1) node[above]{$\hat\theta$};
      \draw[mass] (2.5,0) circle (.1);
    \end{scope}
    \draw[axes] (1.5,0) arc (0:38:1.5) node[pos=.55,right]{$\theta$};
  \end{tikzpicture}
  \caption{Basis vectors for rectilinear and curvilinear motions}
  \label{fig:basis-vecs}
\end{figure}
and rotate as the object moves. It is clear from basic geometry that Cartesian
and polar coordinates are related by:
\begin{align*}
  x(t)&=r(t)\cos\left(\theta(t)\right)\\
  y(t)&=r(t)\sin\left(\theta(t)\right)
\end{align*}


%\section{Cylindrical Coordinates in 3D}
%
%One way to extend the coordinates coordinate system into 3D is the
%\textbf{cylindrical coordinate system}. Note that the discussions for this
%topic focuses on $xy$ plane. Since the $z$-axis is linearly independent of
%the $xy$ plane, motion along that direction is independent.
%
%\begin{figure}[ht]
%  \centering
%  \begin{tikzpicture}[scale=.75]
%    \draw[axes] (0,0)--(-2.5,-2.5) node[below]{$x$};
%    \draw[axes] (0,0)--(5,0) node[right]{$y$};
%    \draw[axes] (0,0)--(0,5) node[above]{$z$};
%    \draw[axes] (-1,-1) arc (-110:-45:2) node[midway,below]{$\theta$};
%    \draw[dashed,fill=green!40,opacity=.4](0,0)--(3,-1.5)
%    node[pos=.6,below left,opacity=1]{$r$}--(3,2.5)
%    node[midway,right,black,opacity=1]{$z$}--(0,4);
%    \fill (3,2.5) circle(.1) node[right]{$\bm r(r,\theta,z)$};
%  \end{tikzpicture}
%\end{figure}


\section{Kinematics of Circular Motion}

\subsection{Angular Position}
For constant distance $r$ to the origin, the \textbf{angular position}
$\theta$ determines an object's position as a continuous function of
time\footnote{The more mathematically rigorous notation is to express the
angular position as a vector along the angular direction:
\begin{displaymath}
  \bm\theta=\theta(t)\hat{\bm\theta}
\end{displaymath}
The magnitude is $\theta(t)$ and the direction is $\hat{\bm\theta}$.}:
\begin{equation}
  \boxed{
    \theta=\theta(t)
  }
\end{equation}
The unit for angular position is a \emph{radian} (rad): If motion is confined
to the $xy$-plane, then $\theta$ is the standard angle: $\theta$ is positive
when measured counterclockwise from the $x$-axis, and negative when it is
measured clockwise.
%\begin{figure}[ht]
%  \centering
%  \begin{tikzpicture}[scale=.75]
%    \draw[axes] (-3,0)--(3,0) node[right]{$x$};
%    \draw[axes] (0,-3)--(0,3) node[above]{$y$};
%    \draw[axes] (1,0) arc (0:38:1) node[midway,right]{$\theta$};
%    \draw circle(2.5);
%    \begin{scope}[rotate=38]
%      \draw[vector] (0,0)--(2.44,0) node[midway,above]{$\bm r$};
%      \draw[mass] (2.5,0) circle (.1);
%    \end{scope}
%  \end{tikzpicture}
%\end{figure}

\subsection{Angular Velocity and Velocity}
Analogous to the relationship between position and velocity, for circular
motion, the instantaneous \textbf{angular velocity} (or \textbf{angular
  frequency}) $\omega(t)$, is the time derivative of the angular position,
measured in \emph{radian per second} (\si{\radian\per\second}). It is also a
a function of time:
\begin{equation}
  \boxed{
    \omega(t)=\diff{\theta}t
  }
\end{equation}
Using the fundamental theorem of calculus, angular position $\theta(t)$ and
angular displacement $\Delta\theta(t)$ can be obtained by integrating
$\omega(t)$. The difference, of course, is that one integral is indefinite, and
the other is not.
\begin{equation}
  \boxed{
    \theta(t)=\int\omega(t)\dl t +\theta_0
  }\quad\quad
  \boxed{
    \Delta\theta(t)=\int_{t_0}^t\omega(t)\dl t
  }
\end{equation}

Again, if motion is confined to the $xy$-plane, then $\omega$ is positive when
the object moves in the counterclockwise direction, and negative when it moves
clockwise (Fig.~\ref{fig:omega-plus-minus}).
\begin{figure}[ht]
  \centering
  \begin{tikzpicture}[scale=.75]
    \draw[axes] (-3,0)--(3,0) node[right]{$x$};
    \draw[axes] (0,-3)--(0,3) node[above]{$y$};
    \draw[axes] (1,0) arc (0:38:1) node[midway,right]{$\theta$};
    \draw circle (2.5);
    \begin{scope}[rotate=38]
      \draw[thick] (0,0)--(2.44,0) node[midway,above]{$r$};
      \draw[vector] (2.5,.08)--(2.5,1.5) node[above]{$\bm v$};
      \draw[mass] (2.5,0) circle (.1);
    \end{scope}
  \end{tikzpicture}
  \caption{Sign convention for $\omega$ and direction of velocity vector
    $\bm v$ when circular motion is confined to the $xy$-plane}
  \label{fig:omega-plus-minus}
\end{figure}

As the object moves with angular speed $\omega$, the actual velocity vector
$\bm v$ is tangent to the circle\footnote{For your information, the velocity
vector along \emph{any} path is \emph{always} tangent to the path.}. Obviously
$\bm v$ is not constant in time, as its direction is always changing. However,
there is a simple mathematical relationship between the speed of object
$v(t)=|\bm v(t)|$ and the angular speed $|\omega|$:
\begin{equation}
  v=r|\omega|
\end{equation}
%\begin{itemize}
%\item The direction of $\bm v$ is tangent to circle, along
%  $\hat\theta$, and therefore $\perp$ to $\hat r$
%\item If $\omega>0$, the motion is counter-clockwise
%\item If $\omega<0$, the motion is clockwise
%\end{itemize}
  
%    \begin{tikzpicture}[scale=.75]
%      \draw[axes] (-3,0)--(3,0) node[right]{$x$};
%      \draw[axes] (0,-3)--(0,3) node[above]{$y$};
%      \draw[axes] (1,0) arc (0:38:1) node[midway,right]{$\theta$};
%      \draw circle (2.5);
%      \begin{scope}[rotate=38]
%        \draw[vector] (0,0)--(2.44,0) node[midway,above]{$\bm r$};
%        \draw[vector] (2.5,.08)--(2.5,1.5) node[above]{$\bm v$};
%        \draw[mass] (2.5,0) circle (.1);
%      \end{scope}
%    \end{tikzpicture}

The velocity of the object in circular motion is more properly related to
the angular velocity using this vector cross product:
\begin{equation}
  \bm v=\bm\omega\times\bm r
\end{equation}
$\bm\omega$ is out of the page if motion is counterclockwise, and into the page
if motion is clockwise. Visualizing $\bm\omega$ takes practice, but this vector
notation is mathematically rigorous and consistent
  




%\begin{frame}{Relativity Velocity}
%  If two points $A$ and $B$ are rotating with the same angular velocity with the
%  same cent, their relative position is given by:
%
%  \begin{equation}
%    \boxed{
%      \bm V_B=\bm V_A+ \bm\omega\times\bm r_{BA}
%    }
%  }
%
%  Where $\bm r_{BA}$ is the position of $B$ relative to $A$.



\subsection{Angular Acceleration and Tangential Acceleration}
Analogous to the relation between velocity $\bm v$ and acceleration $\bm a$,
the time derivative of angular velocity $\omega(t)$ is
\textbf{angular acceleration}:
\begin{equation}
  \alpha=\diff{\omega}t=\diff[2]{\theta}t
\end{equation}
The unit for angular acceleration is \emph{radian per second squared}
\si{\radian\per\second\squared}. The sign convention for $\alpha$ is the same
as for $\theta$ and $\omega$. Similar to the relationship between velocity and
angular velocity, \textbf{tangential acceleration} $a_t$ along the direction of
motion is related to angular acceleration $\alpha$ by the radius $r$:
\begin{equation}
  a_t(t)=\diff vt=r\diff{\omega}t=r\alpha
\end{equation}
For \emph{uniform} circular motion, $\omega$ is constant, $\dot\omega=0$, and
therefore $a_t=0$.

By the fundamental theorem of calculus, we can of course integrate angular
acceleration to find the angular velocity (or the \emph{change} in angular
velocity) as a function of time:
\begin{equation}
  \boxed{
    \omega(t)=\int\alpha(t)\dl t+\omega_0
  }
  \quad\quad
  \boxed{
    \Delta\omega(t)=\int_{t_0}^t\alpha(t)\dl t
  }
\end{equation}
The relationships are the same as in rectilinear motion.



\subsection{Period \& Frequency}
%    \begin{tikzpicture}[scale=.75]
%      \draw[axes] (-3,0)--(3,0) node[right]{$x$};
%      \draw[axes] (0,-3)--(0,3) node[above]{$y$};
%      \draw[axes] (1,0) arc (0:38:1) node[midway,right]{$\theta$};
%      \draw circle (2.5);
%      \begin{scope}[rotate=38]
%        \draw[vector] (0,0)--(2.44,0) node[midway,above]{$\bm r$};
%        \draw[vector] (2.5,.08)--(2.5,1.5) node[above]{$\bm v$};
%        \draw[mass] (2.5,0) circle (.1);
%      \end{scope}
%    \end{tikzpicture}
For constant angular velocity $\omega$ (uniform circular motion), the motion
is strictly periodic. Its \textbf{frequency} and \textbf{period} are given by:
\begin{equation}
  \boxed{ f=\frac\omega{2\pi} }\quad\quad
  \boxed{ T=\frac{2\pi}\omega}\quad\quad
  \boxed{ f=\frac1T}
\end{equation}
Period $T$ is measured in \emph{seconds} (\si\second) and frequency $f$ is
measured in \textbf{hertz} (\si\hertz).
  



\subsection{Kinematic Equations}
For constant angular acceleration $\alpha$, the kinematic equations are just
like in rectilinear motion, but with $\theta$ replacing $x$, $\omega$ replacing
$v$, and $\alpha$ replacing $a$:
\begin{align}
  \theta &=\theta_0 + \omega_0 t + \frac12\alpha t^2\\
  \omega &=\omega_0+ \alpha t\\
  \omega^2& = \omega_0^2+ 2\alpha(\theta-\theta_0)
\end{align}
For non-constant $\alpha$, calculus will be required.




%  \textbf{Example:} An object moves in a circle with angular acceleration
%  \SI{3.0}{\radian\per\second\squared}. The radius is \SI{2.0}{\metre} and it
%  starts from rest. How long does it take for this object to finish a circle?




\subsection{Centripetal Acceleration}% \& Centripetal Force}
Even when there is no angular acceleration, there is also a component of
acceleration towards the centre of the motion,
%\begin{figure}[ht]
%  \centering
%  \begin{tikzpicture}
%    \draw[->](-3,0)--(3,0);
%    \draw[->](0,-3)--(0,3);
%    \draw circle (2.5);
%    \begin{scope}[rotate=30]
%      \draw[->,very thick,blue](2.5,0)--(2.5,1.5) node[above]{$\bm v_i$};
%      \draw[->,very thick,red] (0,0)--(2.5,0)node[pos=.6,below]{$\bm r_i$};
%      \fill (2.5,0) circle(.06);
%    \end{scope}
%    \begin{scope}[rotate=90]
%      \draw[->,very thick,blue] (2.5,0)--(2.5,1.5)node[left]{$\bm v_f$};
%      \draw[->,very thick,red] (0,0)--(2.5,0) node[midway,left]{$\bm r_f$};
%      \fill (2.5,0) circle(.06);
%    \end{scope}
%    \draw(0,1)[<->] arc(90:30:1) node[pos=.6,above]{$\Delta\theta$};
%  \end{tikzpicture}
%\end{figure}
called the \textbf{centripetal acceleration} $\bm a_c$.
%\begin{equation}
%  \boxed{
%    \bm a_c=-\frac{v^2}r\hat{\bm r}=-(\omega^2r)\hat{\bm r}
%  }
%\end{equation}
%The negative sign indicates that the direction of $\bm a_c$ %and $\bm F_c$ are
%is radially \emph{inward}, towards the centre of motion, as $\hat{\bm r}$ is
%the outward radial direction. In uniform circular motion ($\alpha=0$), where
%the period or frequency are known, the speed of the object is:
%\begin{equation}
%  v=\omega r = 2\pi rf = \frac{2\pi r}T
%\end{equation}
%Centripetal acceleration can therefore be expressed based on $T$ or $f$:
%\begin{equation}
%  \bm a_c=-(\omega^2r)\hat{\bm r}\quad\rightarrow\quad
%  \boxed{
%    \bm a_c=-\frac{4\pi^2r}{T^2}\hat{\bm r}=-4\pi^2rf^2\hat{\bm r}
%  }
%\end{equation}

Consider an object in uniform circular motion in the counterclockwise
direction with constant radius $r$ and constant speed $v$ (i.e.\ constant
angular speed $\omega=v/r$), as shown in
Fig.~\ref{fig:v-in-uniform-circ-motion}. At initial time $t_0$, the position
and velocity of the object are given by $\bm r_0=\bm r(t_0)$ and
$\bm v_0=\bm v(t_0)$. Then, at a later time $t_1=t_0+\Delta t$, the object has
moved through an angular displacement of $\Delta\theta$, and the final
position and velocity are now $\bm r_1=\bm r(t_1)$ and $\bm v_1=\bm v(t_1)$.
\begin{figure}[ht]
  \centering
  \begin{tikzpicture}
    \draw[axes] (-3,0)--(3,0);
    \draw[axes] (0,-3)--(0,3);
    \draw circle (2.5);
    \begin{scope}[rotate=30]
      \draw[vector,blue] (2.5,0)--(2.5,1.5) node[above]{$\bm v_0$};
      \draw[vector,red] (0,0)--(2.5,0) node[pos=.6,below]{$\bm r_0$};
      \fill (2.5,0) circle (.06);
    \end{scope}
    \begin{scope}[rotate=90]
      \draw[vector,blue] (2.5,0)--(2.5,1.5) node[left]{$\bm v_1$};
      \draw[vector,red] (0,0)--(2.5,0) node[midway,left]{$\bm r_1$};
      \fill (2.5,0) circle (.06);
    \end{scope}
    \draw[<-,thick] (0,1) arc (90:30:1) node[pos=.6,above]{$\Delta\theta$};
  \end{tikzpicture}
  \caption{An object in counter-clockwise uniform circular motion}
  \label{fig:v-in-uniform-circ-motion}
\end{figure}

From the definition of acceleration,
\begin{equation}
  \bm a_c=\frac{\Delta\bm v}{\Delta t}=\frac{\bm v_1-\bm v_0}{\Delta t}
\end{equation}
And the magnitude of the centripetal acceleration is
\begin{equation}
  |\bm a_c|=\frac{|\Delta\bm v|}{\Delta t}
\end{equation}
Since $|\bm r_0|=|\bm r_i|=r$ (circular motion), and $|\bm v_0|=|\bm v_1|=v$
(constant speed, uniform circular motion), the triangles formed by the
displacement vector $\Delta\bm r$ and the change in velocity $\Delta\bm v$ are
similar isosceles triangles, as shown in Fig.~\ref{fig:sim-triangles}.
\begin{figure}[ht]
  \centering
  \begin{subfigure}{.4\linewidth}
    \centering
    \begin{tikzpicture}[scale=1.5,vector]
      \draw[rotate=-60,red] (0,0)--(0,2) node[midway,below]{$\bm r_0$};
      \draw[red] (0,0)--(0,2) node[midway,left]{$\bm r_1$};
      \draw (2*sin{60},1)--(0,2) node[midway,above]{$\Delta\bm r$};
      \draw[<-,thick] (0,.8) arc (90:30:.8) node[midway,above]{$\Delta\theta$};
    \end{tikzpicture}
    \caption{Displacement}
  \end{subfigure}
  \begin{subfigure}{.4\linewidth}
    \centering
    \begin{tikzpicture}[scale=1.5,vector]
      \draw[rotate=-60,blue] (0,0)--(-2,0) node[midway,right]{$\bm v_0$};
      \draw[blue](0,0)--(-2,0) node[midway,below]{$\bm v_1$};
      \draw(-1,2*sin{60})--(-2,0) node[midway,left]{$\Delta\bm v$};
    \end{tikzpicture}
    \caption{Change in velocity}
  \end{subfigure}
  \caption{Vector diagrams for change in position and velocity are similar
    isosceles triangles}
  \label{fig:sim-triangles}
\end{figure}


%\begin{tikzpicture}
%  \begin{scope}[very thick,->]
%    \draw[rotate=-60,red](0,0)--(0,2) node[midway,below]{$r$};
%    \draw[red](0,0)--(0,2) node[midway,left]{$r$};
%    \draw (2*sin{60},1)--(0,2)node[midway,above]{$|\Delta\bm r|$};
%  \end{scope}
%\end{tikzpicture}
%\begin{tikzpicture}
%  \begin{scope}[very thick,->]
%    \draw[rotate=-60,blue](0,0)--(-2,0) node[midway,right]{$v$};
%    \draw[blue](0,0)--(-2,0) node[midway,below]{$v$};
%    \draw(-1,2*sin{60})--(-2,0) node[midway,left]{$|\Delta\bm v|$};
%  \end{scope}
%\end{tikzpicture}

That the triangles are similar means that:
\begin{equation}
  \frac{|\Delta\bm r|}r=\frac{|\Delta\bm v|}v
  \quad\rightarrow\quad
  |\Delta\bm v|=\frac vr|\Delta\bm r|
\end{equation}
The magnitude of the centripetal acceleration ($|\bm a_c|$):
\begin{equation}
  |\bm a_c|=\frac{|\Delta\bm v|}{\Delta t}
  =\frac vr\left[\frac{|\Delta\bm r|}{\Delta t}\right]=\frac{v^2}r
\end{equation}
The direction of the centripetal acceleration is easy to show using basic
geometry. When $\Delta t\rightarrow 0$, $\Delta\theta\rightarrow 0$.

Since
\begin{equation}
  2\alpha+\Delta\theta=\ang{180}
\end{equation}
when $\Delta\theta\rightarrow 0$, $\alpha\rightarrow\ang{90}$.
\begin{figure}[ht]
  \centering
  \begin{tikzpicture}
    \begin{scope}[vector]
      \draw[blue] (0,0)--(0,4) node[midway,right]{$\bm v_i$};
      \draw[blue,rotate=30] (0,0)--(0,4) node[midway,left]{$\bm v_f$};
      \draw(0,4)--(-4*sin{30},4*cos{30}) node[midway,above]{$\Delta\bm v$};
    \end{scope}
    \draw[thick] (0,1) arc (90:90+30:1) node[midway,above]{$\Delta\theta$};
    \draw[thick] (0,3.5) arc (270:195:.5)node[midway,below]{$\alpha$};
  \end{tikzpicture}
\end{figure}
The direction of $\Delta\bm v$ is perpendicular to $\bm v$, Since centripetal
acceleration is in the same direction as $\bm v$, $\bm a_c$ points towards the
centre of the circular path (i.e.\ the inwards radial direction
$-\hat{\bm r}$), giving us the equation for centripetal acceleration, which can
be expressed using the speed $v$ or the angular speed $\omega=v/r$ of the
object:
\begin{equation}
  \boxed{
    \bm a_c=-\frac{v^2}r\hat{\bm r}=-\omega^2r\hat{\bm r}
  }
  \label{eq:centripetal-acceleration}
\end{equation}



\subsection{Acceleration in General Circular Motion}

\begin{figure}[ht]
  \centering
  \begin{tikzpicture}[scale=4.2]
    \draw[dashed] (.866,-.5) arc (-30:30:1);
    \draw[vector,magenta] (1,0)--(.5,0) node[midway,below]{$\bm a_c=\omega^2r$};
    \draw[vector,cyan] (1,0)--(1,.3) node[right]{$\bm a_t=r\alpha$};
    \draw[vector] (1,0)--(.5,.3) node[left]{$\bm a$};
    \fill (1,0) circle (.02);
  \end{tikzpicture}
  \caption{The two component of acceleration in general circular motion.}
  \label{fig:circular-motion-accelerations}
\end{figure}

To summarize, in general circular motion, there are two components of
acceleration (Fig.~\ref{fig:circular-motion-accelerations}):
\begin{itemize}
\item\textbf{Centripetal acceleration} $\bm a_c$ depends on radius of curvature
  $r$ and instantaneous speed $v$. The direction of the acceleration is
  towards the centre of the circular path.
\item\textbf{Tangential acceleration} $\bm a_t$ depends on radius $r$ and
  angular acceleration $\alpha$. The direction of the acceleration is tangent
  to the circle, along the angular direction $\hat{\bm\theta}$.
\end{itemize}
The total acceleration is the vector sum of both components:
\begin{equation}
  \bm a = \bm a_c+\bm a_t
\end{equation}



\section{Dynamics of Circular Motion}

By the second law of motion $\bm F_\text{net}=m\bm a$, acceleration must be
caused by a net force along that direction. Along the angular direction
($\hat{\bm\theta}$), tangential acceleration is caused by the tangential force
$\bm F_t$:
\begin{equation}
  \boxed{
    \bm F_t=m\bm a_t=mr\alpha\hat{\bm\theta}
  }
\end{equation}
while in the radial ($\hat{\bm r}$) direction, centripetal acceleration
(directed in the $-\hat{\bm r}$ direction) is caused by a
\textbf{centripetal force}:
\begin{equation}
  \boxed{
    \bm F_c
    =m\bm a_c
    =-\frac{mv^2}r\hat{\bm r}
    =-m\left(\omega^2r\right)\hat{\bm r}
  }
\end{equation}
%  The condition for circular motion is the second law of motion:
%
%  \begin{equation}
%    \bm F_c=\sum\bm F=m\bm a_c
%  }

The forces that generate the centripetal force comes from the free-body
diagram. It may include:
\begin{itemize}[nosep]
\item Gravity
\item Friction
\item Normal force
\item Tension
\item Etc.
\end{itemize}




%\begin{frame}{Example: Horizontal Motion}
%  
%    \column{.4\textwidth}
%    \pic1{puck-on-table}
%    
%    \column{.6\textwidth}
%    \textbf{Example:} In the figure on the left, a mass $m_1$ is rolling around
%    a frictionless table with radius $R$ with a speed $v$. What is the mass of
%    $m_2$?
%  
%
%
%
%\begin{frame}{Banked Curves on Highways and Racetracks}
%  
%    \column{.35\textwidth}
%    \centering
%    \pic{.8}{banked-turn-acceleration}
%    
%    \begin{tikzpicture}[vector]
%      \fill circle (.08);
%      \draw[rotate=-30] (0,0)--(0,1) node[above]{$\bm N$};
%      \draw (0,0)--(0,-1) node[below]{$\bm F_g$};
%      \draw[rotate=60] (0,0)--(0,-1) node[right]{$\bm f$};
%    \end{tikzpicture}
%    \begin{tikzpicture}[axes]
%      \draw (0,0)--(1,0) node[right]{$x$};
%      \draw (0,0)--(0,1) node[above]{$y$};
%    \end{tikzpicture}
%make
%    \column{.65\textwidth}
%    No motion in the $y$ direction, i.e.\ no net force:
%
%    \begin{equation}
%      \sum F_y=N\cos\theta-f\sin\theta-F_g=0
%    }
%
%    Net force in the $x$ direction is the centripetal force:
%
%    \begin{equation}
%      \sum F_x=N\sin\theta +f\cos\theta = \frac{mv^2}r
%    }
%
%    Friction force $\bm f$ may be static or kinetic.
%  
%
%
%
%
%\begin{frame}{Banked Curves on Highways and Racetracks}
%  
%    \column{.35\textwidth}
%    \centering
%    \pic{.8}{banked-turn-acceleration}
%    
%    \begin{tikzpicture}[vector]
%      \fill circle (.08);
%      \draw[rotate=-30] (0,0)--(0,1)node[above]{$\bm N$};
%      \draw (0,0)--(0,-1)node[below]{$\bm F_g$};
%      \draw[rotate=60] (0,0)--(0,-1)node[right]{$\bm f$};
%    \end{tikzpicture}
%    \begin{tikzpicture}[axes]
%      \draw (0,0)--(1,0) node[right]{$x$};
%      \draw (0,0)--(0,1) node[above]{$y$};
%    \end{tikzpicture}
%
%    \column{.65\textwidth}
%    For analysis, use the simplified equation for friction $f=\mu N$ (i.e.\
%    assume either kinetic friction or maximum static friction), and weight
%    $F_g=mg$, the equations on the previous slides can be arranged as:
%
%    \vspace{-.3in}{\large
%      \begin{align*}
%        N\left(\cos\theta-\mu\sin\theta\right) &=mg\\
%        N\left(\sin\theta+\mu\cos\theta\right) &=\frac{mv^2}r
%      \end{align*}
%    }
%  
%
%
%
%\begin{frame}{Banked Curves on Highways and Racetracks}
%  
%    \column{.35\textwidth}
%    \centering
%    \pic{.8}{banked-turn-acceleration}\\
%    \begin{tikzpicture}[vector]
%      \fill circle(.08);
%      \draw[rotate=-30] (0,0)--(0,1) node[above]{$\bm N$};
%      \draw (0,0)--(0,-1) node[below]{$\bm F_g$};
%      \draw[rotate=60] (0,0)--(0,-1) node[right]{$\bm f$};
%    \end{tikzpicture}
%    \begin{tikzpicture}[axes]
%      \draw (0,0)--(1,0) node[right]{$x$};
%      \draw (0,0)--(0,1) node[above]{$y$};
%    \end{tikzpicture}
%
%    \column{.65\textwidth}
%    Dividing the two equations removes both the normal force and mass terms:
%
%    \begin{equation}
%      \frac{\sin\theta+\mu\cos\theta}{\cos\theta-\mu\sin\theta}
%      =\frac{v^2}{rg}
%    }
%
%    The \emph{maximum} velocity $v_\text{max}$ can be expressed as:
%
%    \begin{equation}
%      \boxed{v_\text{max}=
%        \sqrt{rg\frac{\sin\theta+\mu\cos\theta}{\cos\theta-\mu\sin\theta}}
%      }
%    }
%
%    Note that $v_\text{max}$ does not depend on mass.
%  
%
%
%
%
%\begin{frame}{Banked Curves on Highways and Racetracks}
%  In the limit of $\mu=0$ (frictionless case), the equation reduces to:
%
%  \begin{equation}
%    \boxed{ v_\text{max}=\sqrt{rg\tan\theta} }
%  }
%
%  And in the limit of a flat roadway with no banking ($\theta=0$,
%  $\sin\theta=0$ and $\cos\theta=1$), the equation reduces to:
%
%  \begin{equation}
%    \boxed{
%      v_{\text{max}}=\sqrt{\mu rg}
%    }
%  }
%
%
%
%
%
%%
%%
%%\begin{frame}{Another Example: Exit Ramp}
%%  \textbf{Example:} A car exits a highway on a ramp that is banked at
%%  \ang{15} to the horizontal. The exit ramp has a radius of curvature of
%%  \SI{65}{\metre}. If the conditions are extremely icy and the driver cannot
%%  depend on any friction to help make the turn, at what speed should the driver
%  travel so that the car will not skid off the ramp? What if there is friction?
%


\section{Vertical Circles}

Circular motion with a horizontal path is straightforward. However, for
vertical motion, it is generally difficult to solve by dynamics and kinematics.
Instead, use conservation of energy may be used to solve for $\bm v$.
Afterwards, we can use the equation for centripetal force to find other forces.
%\textbf{Remember:} If it is impossible to get the required centripetal
%force, then it could not continue the circular motion




\subsection{Simple Pendulum}
A simple pendulum, shown in Fig.~\ref{fig:simple-pendulum-again}, is an example
of a circular motion problem. In Section~\ref{sec:simple-pendulum-energy}, we
have already established how energy is conserved in a simple pendulum system.
\begin{figure}[ht]
  \centering
  \begin{tikzpicture}[scale=.8]
    \fill[pattern=north east lines] (-1,0) rectangle (1,0.2);
    \draw[very thick] (-1,0)--(1,0);
    \begin{scope}[rotate=20]
      \draw[thick] (0,0)--(0,-5);
      \shade[ball color=red] (0,-5) circle (0.2) node[below right]{$m$};
      \begin{scope}[vector,red]
        \draw (0,-5)--(0,-3.3) node[left]{$\bm F_T$};
        \draw[rotate around={-20:(0,-5)}] (0,-5)--(0,-6.5)
        node[below]{$\bm F_g$};
      \end{scope}
    \end{scope}
    \draw[dashed] (0,0)--(0,-5);
    \draw[dashed] (3.54,-3.54) arc (315:225:5);
  \end{tikzpicture}
  \caption{A simple pendulum is a vertical circular motion problem}
  \label{fig:simple-pendulum-again}
\end{figure}
In this system, which is defined as the pendulum bob and Earth, there are two
two forces act on the pendulum: weight $\bm F_g$, and tension $\bm F_T$. As
the pendulum swings, $\bm F_T$ is always $\perp$ to motion, therefore it
doesn't do any mechanical work. The only work is done by $\bm F_g$ as the
pendulum changes height. Since gravity is an internal force, the total energy
of the system is constant, or:
\begin{equation*}
  \Delta U_g + \Delta K = 0
\end{equation*}
However, only using conservation of energy does not immediately allow us to
find the tension force on the pendulum, nor the acceleration of the pendulum
bob. We must therefore use equations for circular motion to find the forces:
%\item Speed of the pendulum at any height is found using conservation
%  of energy
%  \begin{itemize}
%  \item 
%  \item Work is done by gravity (a conservative force) alone
%  \end{itemize}
%\item Tangential and centripetal accelerations are based on the net force
%  along the angular and radial directions
%\end{itemize}
  
\textbf{At the top of the swing}, when the pendulum is deflected by an angle
$\theta$, shown in Fig.~\ref{fig:top-swing}, velocity $v$ is zero by
definition.
\begin{figure}[ht]
  \centering
  \begin{tikzpicture}[scale=.75]
    \fill[pattern=north east lines] (-1,0) rectangle (1,0.2);
    \draw[very thick] (-1,0)--(1,0);
    \begin{scope}[rotate=45]
      \draw[thick] (0,0)--(0,-5);
      \shade[ball color=red] (0,-5) circle (.2) node[right=2.5]{$m$};
      \begin{scope}[vector,red]
        \draw[dotted] (0,-5)--(-1.1,-5)node[left]{$F_g\sin\theta$};
        \draw[dotted] (0,-5)--(0,-6.1)
        node[right,fill=yellow!20]{$F_g\cos\theta$};
        \draw (0,-5)--(0,-3.9) node[left,fill=yellow!20]{$\bm F_T$};
        \draw[rotate around={-45:(0,-5)}] (0,-5)--(0,-6.5)
        node[below]{$\bm F_g$};
      \end{scope}
    \end{scope}
    \draw[dashed] (0,0)--(0,-5);
    \draw[dashed] (3.54,-3.54) arc (315:225:5);
    \draw[axes] (0,-2) arc (270:315:2) node[midway,below]{$\theta$};
  \end{tikzpicture}
  \caption{Forces at the top of the swing of a simple pendulum}
  \label{fig:top-swing}
\end{figure}
From Eq.~\ref{eq:centripetal-acceleration}, we know that centripetal
acceleration must also be zero:
\begin{equation}
  a_c=\frac{v^2}r=0
\end{equation}
Therefore the net force along the radial direction $\hat{\bm r}$ is zero. The
tension force $F_T$ can be calculated:
\begin{equation}
  F_T=F_g\cos\theta=mg\cos\theta
\end{equation}
%    \begin{tikzpicture}[scale=.75]
%      \fill[pattern=north east lines] (-1,0) rectangle (1,0.2);
%      \draw[very thick] (-1,0)--(1,0);
%      \begin{scope}[rotate=45]
%        \draw[thick] (0,0)--(0,-5);
%        \shade[ball color=red] (0,-5) circle(.2) node[right=2.5]{$m$};
%        \begin{scope}[vector,red]
%          \draw[dotted] (0,-5)--(-1.1,-5)
%          node[left,fill=cyan!10]{$F_g\sin\theta$};
%          \draw[dotted] (0,-5)--(0,-6.1) node[right]{$F_g\cos\theta$};
%          \draw (0,-5)--(0,-3.9) node[left]{$\bm F_T$};
%          \draw[rotate around={-45:(0,-5)}] (0,-5)--(0,-6.5)
%          node[below]{$\bm F_g$};
%        \end{scope}
%      \end{scope}
%      \draw[dashed] (0,0)--(0,-5);
%      \draw[dashed] (3.54,-3.54) arc (315:225:5);
%      \draw[axes] (0,-2) arc (270:315:2) node[midway,below]{$\theta$};
%    \end{tikzpicture}
In the tangential direction $\hat{\bm\theta}$, there is still a tangential
component of gravity: $F_t=F_g\sin\theta=mg\sin\theta$, therefore, there is a
tangential acceleration with a magnitude of:
\begin{equation}
  a_t=\frac{F_t}m=g\sin\theta
\end{equation}
This is the same acceleration as an object sliding down a frictionless ramp at
an angle of $\theta$.

\textbf{At the bottom of the swing}, where $\theta=0$, as shown in
Fig.~\ref{fig:bottom-swing}, the velocity is at its maximum value,
\begin{figure}[ht]
  \centering
  \begin{tikzpicture}[scale=.8]
    \fill[pattern=north east lines] (-1,0) rectangle (1,.2);
    \draw[very thick] (-1,0)--(1,0);
    \draw[thick] (0,0)--(0,-5);
    \shade[ball color=red] (0,-5) circle (0.2) node[below right]{$m$};
    \draw[vector,red] (0,-5)--(0,-2.5) node[right]{$\bm F_T$};
    \draw[vector,red] (0,-5)--(0,-6.5) node[below]{$\bm F_g$};
    \draw[dashed] (3.54,-3.54) arc (315:225:5);
  \end{tikzpicture}
  \caption{A simple pendulum at the bottom of its swing}
  \label{fig:bottom-swing}
\end{figure}
therefore centripetal acceleration is at maximum value because:
\begin{equation}
    a_c=\frac{v^2}r
\end{equation}
At the lowest point, tension is the highest:
\begin{equation}
  F_T=F_g+F_c=m\left(g+\frac{v^2}r\right)
\end{equation}
There is no tangential acceleration because there are forces in the angular
direction:
\begin{equation}
  a_t=0
\end{equation}  




\subsection{Example Problem: Vertical Motion}
\textbf{Example:} You are playing with a yo-yo with a mass $M$. The full
length of the string is $R$. You decide to see how slowly you can swing it in
a vertical circle while keeping the string fully extended, even when the
yo-yo is at the top of its swing.
\begin{enumerate}%[a.]
\item Calculate the minimum speed at which you can swing the yo-yo while
  keeping it on a circular path.
\item If the yo-yo is at its minimum speed at the top of its swing, find the
  tension in the string when the yo-yo is at the side and at the bottom.
\end{enumerate}




%\begin{frame}{Example Problem: Vertical Motion}
%  %This is a very typical problem for vertical motion.
%  %To solve this problem, we
%  First, we draw free-body diagrams for each of the positions in the circle.
%  There are two forces acting on the yo-yo: gravity ($\bm F_g$) and tension
%  ($\bm F_T$).
%  %\footnote{We are, of
%  %course, ignoring drag and friction, but a this speed, this will not affect
%  %our answers}
%  \begin{center}
%    \begin{tikzpicture}[scale=.75]
%      \draw[thick] circle (2);
%      \begin{scope}[red]
%        \fill (0,2) circle (.1);
%        \draw[vector] (-.06,2)--+(0,-1) node[left]{$\bm F_T$};
%        \draw[vector] (.06,2)--+(0,-1.5) node[right]{$\bm F_g$};
%      \end{scope}
%      \begin{scope}[violet]
%        \fill (-2,0) circle (.1);
%        \draw[vector] (-2,0)--+(1.5,0) node[below left]{$\bm F_T$};
%        \draw[vector] (-2,0)--+(0,-1.5) node[left]{$\bm F_g$};
%      \end{scope}
%      \begin{scope}[orange]
%        \fill (0,-2) circle (.1);
%        \draw[vector] (0,-2)--+(0,1.7) node[right]{$\bm F_T$};
%        \draw[vector] (0,-2)--+(0,-1.5) node[left]{$\bm F_g$}; 
%      \end{scope}
%    \end{tikzpicture}
%  \end{center}
%  \vspace{-.2in}Since the circular motion is not uniform (i.e.\ the speed of
%  the yo-yo is not constant), we have to also use conservation of energy to
%  solve it.
%
%
%
%
%
%\begin{frame}{Example Problem: Vertical Motion}
%  \centering
%  \begin{tikzpicture}
%    \draw[dashed] circle (2);
%    \begin{scope}[red]
%      \fill (0,2) circle (.1);
%      \draw[vector] (-.06,2)--+(0,-1) node[left]{$\bm F_T$};
%      \draw[vector] (.06,2)--+(0,-1.5) node[right]{$\bm F_g$};
%      \draw[vector,black] (-.2,2)--+(-1,0) node[left]{$\bm v$};
%    \end{scope}
%
%    \node[text width=7.8cm,fill=red!10] (fc) at (6.6,2.5){
%      At the top of the circle, centripetal force is provided by both gravity
%      and string tension, i.e.:
%      
%      \vspace{-.1in}\begin{displaymath}
%        F_c = Ma_c\quad\rightarrow\quad
%        F_T+F_g = \frac{Mv^2}R
%      \end{displaymath}\par
%    };
%    \draw[axes,red] (fc) to[out=180,in=80] (0,2.2);
%    \uncover<2->{
%      \node[text width=7.8cm,fill=yellow!10] (min) at (6.6,0){
%        Since $M$, $g$ and $R$ are constant, minimum velocity $v_\text{min}$ on
%        the right side means $F_T=0$ on the left side. We are left with:
%        
%        \vspace{-.1in}\begin{displaymath}
%          Mg = \frac{Mv_\text{min}^2}R
%        \end{displaymath}\par
%      };
%    }
%    \uncover<3->{
%      \node[text width=7.8cm,fill=green!12] at (6.6,-2.4){
%        Cancelling $M$ and solving for $v_\text{min}$, we have:
%      
%        \vspace{-.12in}\begin{displaymath}
%          v^2=gR\quad\rightarrow\quad\boxed{v_\text{min} = \sqrt{gR}}
%        \end{displaymath}\par
%      };
%    }
%  \end{tikzpicture}
%
%
%
%
%
%\begin{frame}{Example Problem: Vertical Motion}
%  \centering
%  \begin{tikzpicture}
%    \fill circle (.05);
%    \draw[dashed] circle (2);
%    \begin{scope}[violet]
%      \fill (-2,0) circle (.1);
%      \draw[vector] (-2,0)--+(1.5,0) node[below left]{$\bm F_T$};
%      \draw[vector] (-2,0)--+(0,-1.5) node[right]{$\bm F_g$};
%    \end{scope}
%
%    \node[text width=5.5cm,fill=red!10] (fc) at (-5.2,2.2){
%      At the side of the circle, centripetal force is provided only by
%      tension:
%
%      \vspace{-.1in}\begin{displaymath}
%        F_c = Ma_c\quad\rightarrow\quad
%        F_T = \frac{Mv^2}R
%      \end{displaymath}
%      But we do not know the speed $v$ of the yo-yo at this location yet.
%    };
%    
%    \uncover<2->{
%      \node[text width=4.5cm,fill=yellow!15] (min) at (4.5,2.6){
%        Using conservation of energy:
%        
%        \vspace{-.2in}\begin{align*}
%          K_\text{top} + U_\text{top} &= K_\text{side}\\
%          \frac12Mv_\text{top}^2 + MgR &=\frac 12Mv_\text{side}^2
%        \end{align*}
%      };
%    }
%    
%    \uncover<3->{
%      \node[text width=4.5cm,fill=green!10] at (4.5,.4){
%        Cancelling $M$ term and solving for $v_\text{side}^2$, we have:
%
%        \vspace{-.1in}\begin{displaymath}
%          v_\text{side}^2 = v_\text{top}^2+2gR
%        \end{displaymath}\par
%      };
%    }
%
%
%    \uncover<4->{
%      \node[text width=4.5cm,fill=blue!15] at (4.5,-1.6){
%        Since $v_\text{top}^2=v_\text{min}^2=gR$ that we have just calculated,
%
%        \vspace{-.2in}\begin{displaymath}
%          v_\text{side}^2 = gR+2gR=3gR
%        \end{displaymath}\par
%      };
%    }
%
%    \uncover<5->{
%      \node[text width=5.5cm,fill=violet!15] at (-5.2,-1.1){
%        Now the final expression for tension:
%        
%        \vspace{-.2in}\begin{displaymath}
%          F_T = \frac{Mv^2}R = \frac{M(3gR)}R=\boxed{3Mg}
%        \end{displaymath}
%        Tension is 3 times the weight of the yo-yo!
%      };
%    }
%  \end{tikzpicture}
%
%
%
%
%\begin{frame}{Example Problem: Vertical Motion}
%  \centering
%  \begin{tikzpicture}
%    \fill circle (.05);
%    \draw[dashed] circle (2);
%    \begin{scope}[orange]
%      \fill (0,-2) circle (.1);
%      \draw[vector] (0,-2)--+(0,1.7) node[right]{$\bm F_T$};
%      \draw[vector] (0,-2)--+(0,-1.5) node[left]{$\bm F_g$}; 
%    \end{scope}
%
%    \node[text width=5.6cm,fill=red!10] (fc) at (-5,1.2){
%      At the bottom of the circle, tension contributes to centripetal force,
%      while gravity contributes \emph{against} it:
%
%      \vspace{-.22in}\begin{displaymath}
%        F_c = Ma_c\quad\rightarrow\quad
%        F_T-F_g = \frac{Mv^2}R
%      \end{displaymath}
%      Again, we need to find the speed of the yo-yo at this location.
%    };
%    
%    \uncover<2->{
%      \node[text width=4.6cm,fill=yellow!10] (min) at (4.5,1.7){
%        Using conservation of energy again:
%        
%        \vspace{-.2in}\begin{align*}
%          K_\text{top} + U_\text{top} &= K_\text{bottom}\\
%          \frac12Mv_\text{top}^2 + MgR &=\frac 12Mv_\text{bottom}^2
%        \end{align*}
%      };
%    }
%    
%    \uncover<3->{
%      \node[text width=4.6cm,fill=green!10] at (4.5,-.6){
%        Cancelling $M$ term and solving for $v_\text{bottom}^2$, we have:
%
%        \vspace{-.1in}\begin{displaymath}
%          v_\text{bottom}^2 = v_\text{top}^2+4gR
%        \end{displaymath}\par
%      };
%    }
%
%
%    \uncover<4->{
%      \node[text width=4.6cm,fill=blue!15] at (4.5,-2.5){
%        Recognizing that $v_\text{top}^2=gR$ like we did before:
%
%        \vspace{-.2in}\begin{displaymath}
%          v_\text{bottom}^2 = gR+4gR=5gR
%        \end{displaymath}\par
%      };
%    }
%
%    \uncover<5->{
%      \node[text width=5.6cm,fill=violet!15] at (-5,-2.2){
%        Now the final expression for tension:
%        
%        \vspace{-.2in}\begin{align*}
%          F_T &= Mg+\frac{Mv^2}R =Mg + \frac{M(5gR)}R\\
%          &=\boxed{6Mg}
%        \end{align*}
%        $F_T$ is 6 times the weight of the yo-yo
%      };
%    }
%  \end{tikzpicture}
%
%
%
%
\begin{example}
  \textbf{Example:} A cord is tied to a pail of water, and the pail is swung
  in a vertical circle of \SI{1.}\metre. What must be the minimum velocity of
  the pail be at its highest point so that no water spills out?
%  \begin{enumerate}[(A)]
%  \item\SI{3.1}{\metre\per\second}
%  \item\SI{5.6}{\metre\per\second}
%  \item\SI{20.7}{\metre\per\second}
%  \item\SI{100.5}{\metre\per\second}
\end{example}

\begin{example}
  \textbf{Example:} A roller coaster car is on a track that forms a circular
  loop, of radius $R$, in the vertical plane. If the car is to maintain contact
  with the track at the top of the loop (generally considered to be a good
  thing), what is the minimum speed that the car must have at the bottom of the
  loop? Ignore air resistance and rolling friction.
%  \begin{enumerate}[(A)]
%  \item $\sqrt{2gR}$
%  \item $\sqrt{3gR}$
%  \item $\sqrt{4gR}$
%  \item $\sqrt{5gR}$
%  \end{enumerate}
\end{example}
%
%
%
\begin{example}
  \textbf{Example:} A stone of mass $m$ is attached to a light strong string
  and whirled in a \emph{vertical} circle of radius $r$. At the exact bottom of
  the path, the tension of the string is three times the weight of the stone.
  The stone's speed at that point is:
%  \begin{enumerate}[(A)]
%  \item $2\sqrt{gR}$
%  \item $\sqrt{2gR}$
%  \item $\sqrt{3gR}$
%  \item $4gR$
\end{example}

